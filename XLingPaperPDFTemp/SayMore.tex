\documentclass[10pt]{book}
\setlength{\paperheight}{210mm}
\setlength{\paperwidth}{148mm}
\setlength{\topmargin}{0pt}
\setlength{\voffset}{-43.817244089999996pt}
\setlength{\evensidemargin}{-43.817244089999996pt}
\setlength{\oddsidemargin}{-43.817244089999996pt}
\setlength{\textwidth}{364.195275648pt}
\setlength{\textheight}{483.69685047000013pt}
\setlength{\headheight}{12.5pt}
\setlength{\headsep}{16.45275591pt}
\setlength{\footskip}{10mm}
\DeclareTextSymbol{\textsquarebracketleft}{EU1}{91}
\DeclareTextSymbol{\textsquarebracketright}{EU1}{93}
\usepackage[framemethod=TikZ]{mdframed}
\usepackage{xltxtra}
\usepackage{setspace}
\usepackage[normalem]{ulem}
\usepackage{color}
\usepackage{colortbl}
\usepackage{tabularx}
\usepackage{longtable}
\usepackage{multirow}
\usepackage{booktabs}
\usepackage{calc}
\usepackage{fancyhdr}
\usepackage{fontspec}
\usepackage{hyperref}
\hypersetup{colorlinks=true, citecolor=black, filecolor=black, linkcolor=black, urlcolor=blue, bookmarksopen=true, pdfauthor={NGONO Louis Pascal, John ROETTELE, Matthew LEE}, pdfcreator={XLingPaper version 2.31.0 (www.xlingpaper.org)}, pdftitle={SayMore: Guide d'apprenant (Français)}, pdfkeywords={Paratext 8, SIL International, United Bible Societies, Bible Translation}}
\fancypagestyle{frontmattertitle}
{\fancyhf{}
\renewcommand{\headrulewidth}{0pt}
\renewcommand{\footrulewidth}{0pt}
}\fancypagestyle{frontmatterfirstpage}
{\fancyhf{}
\fancyfoot[C]{{\XLingPaperTimesZNewZRomanFontFamily{\fontsize{9}{10.799999999999999}\selectfont \textit{\small\textit{\thepage}}}}}
\renewcommand{\headrulewidth}{0pt}
\renewcommand{\footrulewidth}{0pt}
}\fancypagestyle{frontmatter}
{\fancyhf{}
\fancyhead[LE]{{\XLingPaperTimesZNewZRomanFontFamily{\fontsize{9}{10.799999999999999}\selectfont \textit{\small\textit{\thepage}}}}}
\fancyhead[RE]{{\XLingPaperTimesZNewZRomanFontFamily{\fontsize{9}{10.799999999999999}\selectfont \textit{\small\textit{\leftmark}}}}}
\fancyhead[LO]{{\XLingPaperTimesZNewZRomanFontFamily{\fontsize{9}{10.799999999999999}\selectfont \textit{\small\textit{\leftmark}}}}}
\fancyhead[RO]{{\XLingPaperTimesZNewZRomanFontFamily{\fontsize{9}{10.799999999999999}\selectfont \textit{\small\textit{\thepage}}}}}
\renewcommand{\headrulewidth}{0pt}
\renewcommand{\footrulewidth}{0pt}
}\fancypagestyle{bodyfirstpage}
{\fancyhf{}
\fancyfoot[C]{{\XLingPaperTimesZNewZRomanFontFamily{\fontsize{10}{12}\selectfont \textit{\small\textit{\thepage}}}}}
\renewcommand{\headrulewidth}{0pt}
\renewcommand{\footrulewidth}{0pt}
}\fancypagestyle{body}
{\fancyhf{}
\fancyhead[LE]{{\XLingPaperTimesZNewZRomanFontFamily{\fontsize{10}{12}\selectfont \textit{\small\textit{\thepage}}}}}
\fancyhead[RE]{{\XLingPaperTimesZNewZRomanFontFamily{\fontsize{10}{12}\selectfont \textit{\small\textit{\leftmark}}}}}
\fancyhead[LO]{{\XLingPaperTimesZNewZRomanFontFamily{\fontsize{10}{12}\selectfont \textit{\small\textit{\rightmark}}}}}
\fancyhead[RO]{{\XLingPaperTimesZNewZRomanFontFamily{\fontsize{10}{12}\selectfont \textit{\small\textit{\thepage}}}}}
\renewcommand{\headrulewidth}{0pt}
\renewcommand{\footrulewidth}{0pt}
}\setmainfont{Times New Roman}
\font\MainFont="Times New Roman" at 10pt
\newfontfamily{\XLingPaperCharisZSILZSmallZCapsFontFamily}{Charis SIL Small Caps}
\newfontfamily{\XLingPaperCourierZNewFontFamily}{Courier New}
\newfontfamily{\XLingPaperTimesZNewZRomanFontFamily}{Times New Roman}
\definecolor{FTColorA}{HTML}{FFE6FF}
\setlength{\parindent}{0em}
\catcode`^^^^200b=\active
\def^^^^200b{\hskip0pt}
\let\origdoublepage\cleardoublepage
\newcommand{\clearemptydoublepage}{\clearpage{\pagestyle{empty}\origdoublepage}}\renewenvironment{quotation}{\list{}{\leftmargin=10mm\rightmargin=10mm}\item[]{}}{\endlist}
\clubpenalty=10000
\widowpenalty=10000
\begin{document}
\baselineskip=\glueexpr\baselineskip + 0pt plus 2pt minus 1pt\relax
\renewcommand{\footnotesize}{\fontsize{8}{9.6}\selectfont }
\newlength{\leveloneindent}
\newlength{\levelonewidth}
\newlength{\leveltwoindent}
\newlength{\leveltwowidth}
\newlength{\levelthreeindent}
\newlength{\levelthreewidth}
\newlength{\levelfourindent}
\newlength{\levelfourwidth}
\newlength{\levelfiveindent}
\newlength{\levelfivewidth}
\newlength{\levelsixindent}
\newlength{\levelsixwidth}
\newdimen\XLingPapertempdim
                \newdimen\XLingPapertempdimletter
                \newcommand{\XLingPapertableofcontents}{\immediate\openout8 = \jobname.toc\relax
\immediate\write8{<toc>}}
\newcommand{\XLingPaperaddtocontents}[1]{\immediate\write8{<tocline ref="#1" page="\thepage"/>}}
\newcommand{\XLingPaperendtableofcontents}{\immediate\write8{</toc>}\closeout8\relax
}
\newcommand{\XLingPaperdotfill}{\leaders\hbox{$\mathsurround 0pt\mkern 4.5 mu\hbox{.}\mkern 4.5 mu$}\hfill}
\newcommand{\XLingPaperdottedtocline}[4]{
\newdimen\XLingPapertempdim
\vskip0pt plus .2pt{
\leftskip#1\relax% left glue for indent
\rightskip\XLingPapertocrmarg% right glue for for right margin
\parfillskip-\rightskip% so can go into margin if need be???
\parindent#1\relax
\interlinepenalty10000
\leavevmode
\XLingPapertempdim#2\relax% numwidth
\advance\leftskip\XLingPapertempdim\null\nobreak\hskip-\leftskip{#3}\nobreak
\XLingPaperdotfill\nobreak
\hbox to\XLingPaperpnumwidth{\hfil\normalfont\normalcolor#4}
\par}}
\newlength{\XLingPaperpnumwidth}
\newlength{\XLingPapertocrmarg}
\setlength{\XLingPaperpnumwidth}{1.55em}\setlength{\XLingPapertocrmarg}{\XLingPaperpnumwidth+1em}
\newlength{\XLingPaperinterlinearsourcewidth}
\newlength{\XLingPaperinterlinearsourcegapwidth}
\settowidth{\XLingPaperinterlinearsourcegapwidth}{  }
\newlength{\XLingPaperlistinexampleindent}
\newlength{\XLingPaperisocodewidth}\setlength{\XLingPaperlistinexampleindent}{.125in+ 2.75em}
\newlength{\XLingPaperlistitemindent}
\newlength{\XLingPaperbulletlistitemwidth}\settowidth{\XLingPaperbulletlistitemwidth}{•\ }\newlength{\XLingPapersingledigitlistitemwidth}
\settowidth{\XLingPapersingledigitlistitemwidth}{8.\ }\newlength{\XLingPaperdoubledigitlistitemwidth}
\settowidth{\XLingPaperdoubledigitlistitemwidth}{88.\ }\newlength{\XLingPapertripledigitlistitemwidth}
\settowidth{\XLingPapertripledigitlistitemwidth}{888.\ }\newlength{\XLingPapersingleletterlistitemwidth}
\settowidth{\XLingPapersingleletterlistitemwidth}{m.\ }\newlength{\XLingPaperdoubleletterlistitemwidth}
\settowidth{\XLingPaperdoubleletterlistitemwidth}{mm.\ }\newlength{\XLingPapertripleletterlistitemwidth}
\settowidth{\XLingPapertripleletterlistitemwidth}{mmm.\ }\newlength{\XLingPaperromanviilistitemwidth}
\settowidth{\XLingPaperromanviilistitemwidth}{vii.\ }\newlength{\XLingPaperromanviiilistitemwidth}
\settowidth{\XLingPaperromanviiilistitemwidth}{viii.\ }\newlength{\XLingPaperromanxviiilistitemwidth}
\settowidth{\XLingPaperromanxviiilistitemwidth}{xviii.\ }\newlength{\XLingPaperspacewidth}
\settowidth{\XLingPaperspacewidth}{\ }
\newcommand{\XLingPaperneedspace}[1]{\penalty-100\begingroup
\newdimen{\XLingPaperspaceneeded}
\newdimen{\XLingPaperspaceavailable}
\setlength{\XLingPaperspaceneeded}{#1}%
\XLingPaperspaceavailable\pagegoal \advance\XLingPaperspaceavailable-\pagetotal
\ifdim \XLingPaperspaceneeded>\XLingPaperspaceavailable
\ifdim \XLingPaperspaceavailable>0pt
\vfil
\fi
\break
\fi\endgroup}
\newcommand{\XLingPaperlistitem}[4]{
\newdimen\XLingPapertempdim
\vskip0pt plus .2pt{
\leftskip#1\relax% left glue for indent
\parindent#1\relax
\interlinepenalty10000
\leavevmode
\XLingPapertempdim#2\relax% label width
\advance\leftskip\XLingPapertempdim\null\nobreak\hskip-\leftskip\hbox to\XLingPapertempdim{\hfil\normalfont\normalcolor#3\ }{#4}\nobreak
\par}}
\newcommand{\XLingPaperexample}[5]{
\newdimen\XLingPapertempdim
\vskip0pt plus .2pt{
\leftskip#1\relax% left glue for indent
\hspace*{#1}\relax
\rightskip#2\relax% right glue for indent
\interlinepenalty10000
\leavevmode
\XLingPapertempdim#3\relax% example number width
\advance\leftskip\XLingPapertempdim\null\nobreak\hskip-\leftskip\hbox to\XLingPapertempdim{\normalfont\normalcolor#4\hfil}{#5}\nobreak
\par}}
\newcommand{\XLingPaperexampleintable}[5]{
\newdimen\XLingPapertempdim
\leftskip#1\relax% left glue for indent
\hspace*{#1}\relax
\rightskip#2\relax% right glue for indent
\interlinepenalty10000
\leavevmode
\XLingPapertempdim#3\relax% example number width
\hbox to\XLingPapertempdim{\normalfont\normalcolor#4\hfil}{
\begin{tabular}
[t]{@{}l@{}}#5\end{tabular}
}\nobreak
}
\newcommand{\XLingPaperfree}[2]{\vskip0pt plus .2pt{
\leftskip#1\relax% left glue for indent
\parindent#1\relax
\interlinepenalty10000
\leavevmode{#2}\nobreak
\par}}
\newcommand{\XLingPaperlistinterlinear}[5]{\vskip0pt plus .2pt{\hspace*{#1}\hspace*{#2}
\XLingPapertempdimletter#3\relax% letter width
\advance\leftskip\XLingPapertempdimletter\null\nobreak\hskip-\leftskip\hspace*{-.3em}\hbox to\XLingPapertempdimletter{\normalfont\normalcolor#4\ \hfil}{#5}\nobreak
\par}}
\newcommand{\XLingPaperlistinterlinearintable}[5]{
\XLingPapertempdimletter#3\relax% letter width
\hspace*{-.3em}\hbox to\XLingPapertempdimletter{\normalfont\normalcolor#4\ \hfil}{
\begin{tabular}
[t]{@{}l@{}}#5\end{tabular}
}\nobreak
}

\newlength{\XLingPaperexamplefreeindent}\setlength{\XLingPaperexamplefreeindent}{-.3 em}\newskip\XLingPaperinterwordskip
\XLingPaperinterwordskip=6.66666pt plus 3.33333pt minus 2.22222pt
\def\XLingPaperintspace{\hskip\XLingPaperinterwordskip}
\def\XLingPaperraggedright{\rightskip=0pt plus1fil\pretolerance=10000}\raggedbottom
\pagestyle{fancy}
\begin{MainFont}
\XLingPapertableofcontents\pagenumbering{roman}
\pagestyle{frontmattertitle}\pagestyle{frontmattertitle}{\clearpage
\vspace*{1cm}\XLingPaperneedspace{3\baselineskip}\noindent
\fontsize{18}{21.599999999999998}\selectfont \textbf{{\centering
\vspace*{0pt}{\XeTeXpicfile "../images/SayMore5.jpg" scaled 750} \\SayMore\\}}}\par{}
{\vspace{.25in}\XLingPaperneedspace{3\baselineskip}\noindent
\fontsize{14}{16.8}\selectfont \textbf{{\centering
Guide d'apprenant (Français)\\}}}\par{}
{\clearpage
\vspace*{1cm}\XLingPaperneedspace{3\baselineskip}\noindent
\fontsize{18}{21.599999999999998}\selectfont \textbf{{\centering
\vspace*{0pt}{\XeTeXpicfile "../images/SayMore5.jpg" scaled 750} \\SayMore\\}}}\par{}
{\vspace{.25in}\XLingPaperneedspace{3\baselineskip}\noindent
\fontsize{14}{16.8}\selectfont \textbf{{\centering
Guide d'apprenant (Français)\\}}}\par{}
{\XLingPaperneedspace{3\baselineskip}\noindent
\textit{{\centering
NGONO Louis Pascal\\}}}\par{}
{\XLingPaperneedspace{3\baselineskip}\noindent
\textit{{\centering
John ROETTELE\\}}}\par{}
{\XLingPaperneedspace{3\baselineskip}\noindent
\textit{{\centering
Matthew LEE\\}}}\par{}
{\XLingPaperneedspace{3\baselineskip}\noindent
\textit{{\centering
SIL International\\}}}\par{}
{\XLingPaperneedspace{3\baselineskip}\noindent
\fontsize{10}{12}\selectfont {\centering
Dec. 2017\\}}\par{}
\clearpage
\pagestyle{frontmatter}\thispagestyle{frontmatterfirstpage}\thispagestyle{frontmatterfirstpage}{\vspace{12.2pt}\XLingPaperneedspace{3\baselineskip}\noindent
\fontsize{18}{21.599999999999998}\selectfont \textbf{{\centering
\raisebox{\baselineskip}[0pt]{\pdfbookmark[1]{Contenu}{rXLingPapContents}}\raisebox{\baselineskip}[0pt]{\protect\hypertarget{rXLingPapContents}{}}Contenu\\}}\markboth{Contenu}{Contenu}
\XLingPaperaddtocontents{rXLingPapContents}}\penalty10000\par{}
\vspace{10.8pt}\vspace{0pt}\hyperlink{SayMSuppMan}{\centering{Part I Introduction\\}}\hyperlink{sInstall}{\XLingPaperdottedtocline{0pt}{0pt}{1 Installation du logiciel Saymore}{2}
}\settowidth{\leveltwoindent}{{1 }\ }\settowidth{\leveltwowidth}{{1.1 }\thinspace\thinspace}\hyperlink{sInterface}{\XLingPaperdottedtocline{\leveltwoindent}{\leveltwowidth}{{1.1 } Changer la langue d'interface utilisateur}{3}
}\hyperlink{cTour}{\XLingPaperdottedtocline{0pt}{0pt}{2 Un Aperçu autour de SayMore}{4}
}\settowidth{\leveltwoindent}{{2 }\ }\settowidth{\leveltwowidth}{{2.1 }\thinspace\thinspace}\hyperlink{sProjectTab}{\XLingPaperdottedtocline{\leveltwoindent}{\leveltwowidth}{{2.1 } Onglet Projet}{4}
}\settowidth{\leveltwoindent}{{2 }\ }\settowidth{\leveltwowidth}{{2.2 }\thinspace\thinspace}\hyperlink{sSessionTab}{\XLingPaperdottedtocline{\leveltwoindent}{\leveltwowidth}{{2.2 } Onglet Session}{5}
}\settowidth{\leveltwoindent}{{2 }\ }\settowidth{\leveltwowidth}{{2.3 }\thinspace\thinspace}\hyperlink{sPeopleTab}{\XLingPaperdottedtocline{\leveltwoindent}{\leveltwowidth}{{2.3 } Onglet Personne}{6}
}\hyperlink{sMenus}{\XLingPaperdottedtocline{0pt}{0pt}{3 Présentation des Menus}{8}
}\settowidth{\leveltwoindent}{{3 }\ }\settowidth{\leveltwowidth}{{3.1 }\thinspace\thinspace}\hyperlink{sMenuProject}{\XLingPaperdottedtocline{\leveltwoindent}{\leveltwowidth}{{3.1 } Menu: Projet}{8}
}\settowidth{\leveltwoindent}{{3 }\ {3.1 }\ }\settowidth{\leveltwowidth}{{3.1.1 }\thinspace\thinspace}\hyperlink{sQuitSaymore}{\XLingPaperdottedtocline{\leveltwoindent}{\leveltwowidth}{{3.1.1 } Quitter}{8}
}\settowidth{\leveltwoindent}{{3 }\ }\settowidth{\leveltwowidth}{{3.2 }\thinspace\thinspace}\hyperlink{sMenuSession}{\XLingPaperdottedtocline{\leveltwoindent}{\leveltwowidth}{{3.2 } Tâches de l'onglet Sessions}{9}
}\settowidth{\leveltwoindent}{{3 }\ }\settowidth{\leveltwowidth}{{3.3 }\thinspace\thinspace}\hyperlink{sMenuPerson}{\XLingPaperdottedtocline{\leveltwoindent}{\leveltwowidth}{{3.3 } Tâches d'onglet Personnes}{9}
}\settowidth{\leveltwoindent}{{3 }\ }\settowidth{\leveltwowidth}{{3.4 }\thinspace\thinspace}\hyperlink{sMenuHelp}{\XLingPaperdottedtocline{\leveltwoindent}{\leveltwowidth}{{3.4 } Tâches d'onglet Aide}{9}
}\settowidth{\leveltwoindent}{{3 }\ {3.4 }\ }\settowidth{\leveltwowidth}{{3.4.1 }\thinspace\thinspace}\hyperlink{sHelp}{\XLingPaperdottedtocline{\leveltwoindent}{\leveltwowidth}{{3.4.1 } Ade...}{10}
}\settowidth{\leveltwoindent}{{3 }\ {3.4 }\ }\settowidth{\leveltwowidth}{{3.4.2 }\thinspace\thinspace}\hyperlink{sAbout}{\XLingPaperdottedtocline{\leveltwoindent}{\leveltwowidth}{{3.4.2 } A propos...}{10}
}\vspace{0pt}\hyperlink{pGestion}{\centering{Part II Gestion d'un projet, des sessions, ou des personnes\\}}\hyperlink{cCreateProjectOver}{\XLingPaperdottedtocline{0pt}{0pt}{4 La Creation ou l'ouverture d'un projet}{12}
}\settowidth{\leveltwoindent}{{4 }\ }\settowidth{\leveltwowidth}{{4.1 }\thinspace\thinspace}\hyperlink{sCreateProject}{\XLingPaperdottedtocline{\leveltwoindent}{\leveltwowidth}{{4.1 } Créer un projet}{12}
}\settowidth{\leveltwoindent}{{4 }\ }\settowidth{\leveltwowidth}{{4.2 }\thinspace\thinspace}\hyperlink{sOpenProject}{\XLingPaperdottedtocline{\leveltwoindent}{\leveltwowidth}{{4.2 } Ouvrir un projet existant}{13}
}\settowidth{\leveltwoindent}{{4 }\ }\settowidth{\leveltwowidth}{{4.3 }\thinspace\thinspace}\hyperlink{sOpenOtherProject}{\XLingPaperdottedtocline{\leveltwoindent}{\leveltwowidth}{{4.3 } Ouvrir un autre projet}{14}
}\hyperlink{cProjectMeta}{\XLingPaperdottedtocline{0pt}{0pt}{5 Remplir les metadonnées du projet}{15}
}\settowidth{\leveltwoindent}{{5 }\ }\settowidth{\leveltwowidth}{{5.1 }\thinspace\thinspace}\hyperlink{sAboutProj}{\XLingPaperdottedtocline{\leveltwoindent}{\leveltwowidth}{{5.1 } À propos de ce projet}{15}
}\settowidth{\leveltwoindent}{{5 }\ }\settowidth{\leveltwowidth}{{5.2 }\thinspace\thinspace}\hyperlink{sAccessProt}{\XLingPaperdottedtocline{\leveltwoindent}{\leveltwowidth}{{5.2 } Protocole d'accès}{15}
}\settowidth{\leveltwoindent}{{5 }\ }\settowidth{\leveltwowidth}{{5.3 }\thinspace\thinspace}\hyperlink{sProDescDoc}{\XLingPaperdottedtocline{\leveltwoindent}{\leveltwowidth}{{5.3 } Documents de description du projet}{16}
}\settowidth{\leveltwoindent}{{5 }\ }\settowidth{\leveltwowidth}{{5.4 }\thinspace\thinspace}\hyperlink{sOtherDocs}{\XLingPaperdottedtocline{\leveltwoindent}{\leveltwowidth}{{5.4 } Autre documents}{16}
}\settowidth{\leveltwoindent}{{5 }\ }\settowidth{\leveltwowidth}{{5.5 }\thinspace\thinspace}\hyperlink{sProProg}{\XLingPaperdottedtocline{\leveltwoindent}{\leveltwowidth}{{5.5 } Progrès du projet}{16}
}\hyperlink{cSessions}{\XLingPaperdottedtocline{0pt}{0pt}{6 Gestion des sessions}{18}
}\settowidth{\leveltwoindent}{{6 }\ }\settowidth{\leveltwowidth}{{6.1 }\thinspace\thinspace}\hyperlink{sCreateSession}{\XLingPaperdottedtocline{\leveltwoindent}{\leveltwowidth}{{6.1 } Ajouter une Session}{18}
}\settowidth{\leveltwoindent}{{6 }\ {6.1 }\ }\settowidth{\leveltwowidth}{{6.1.1 }\thinspace\thinspace}\hyperlink{sNewFromDevice}{\XLingPaperdottedtocline{\leveltwoindent}{\leveltwowidth}{{6.1.1 } Nouveau à partir d'un appareil...}{18}
}\settowidth{\leveltwoindent}{{6 }\ {6.1 }\ }\settowidth{\leveltwowidth}{{6.1.2 }\thinspace\thinspace}\hyperlink{sNewFromRecording}{\XLingPaperdottedtocline{\leveltwoindent}{\leveltwowidth}{{6.1.2 } Nouveau d'un enregistrement...}{19}
}\settowidth{\leveltwoindent}{{6 }\ }\settowidth{\leveltwowidth}{{6.2 }\thinspace\thinspace}\hyperlink{sDeleteSession}{\XLingPaperdottedtocline{\leveltwoindent}{\leveltwowidth}{{6.2 } Supprimer une session}{20}
}\settowidth{\leveltwoindent}{{6 }\ }\settowidth{\leveltwowidth}{{6.3 }\thinspace\thinspace}\hyperlink{sSessMeta1}{\XLingPaperdottedtocline{\leveltwoindent}{\leveltwowidth}{{6.3 } Session Metadata}{20}
}\hyperlink{cPeople}{\XLingPaperdottedtocline{0pt}{0pt}{7 Gestion des personnes}{23}
}\settowidth{\leveltwoindent}{{7 }\ }\settowidth{\leveltwowidth}{{7.1 }\thinspace\thinspace}\hyperlink{sNewPerson}{\XLingPaperdottedtocline{\leveltwoindent}{\leveltwowidth}{{7.1 } Ajouter une Nouvelle Personne}{23}
}\settowidth{\leveltwoindent}{{7 }\ }\settowidth{\leveltwowidth}{{7.2 }\thinspace\thinspace}\hyperlink{sInformedConsent}{\XLingPaperdottedtocline{\leveltwoindent}{\leveltwowidth}{{7.2 } Consentement éclairé}{23}
}\settowidth{\leveltwoindent}{{7 }\ }\settowidth{\leveltwowidth}{{7.3 }\thinspace\thinspace}\hyperlink{sDeletePerson}{\XLingPaperdottedtocline{\leveltwoindent}{\leveltwowidth}{{7.3 } Supprimer une personne}{24}
}\vspace{0pt}\hyperlink{pTranscribe}{\centering{Part III Transcription des fichiers audios et vidéos\\}}\hyperlink{cTranscribe}{\XLingPaperdottedtocline{0pt}{0pt}{8 Survol du processus de la transcription}{26}
}\settowidth{\leveltwoindent}{{8 }\ }\settowidth{\leveltwowidth}{{8.1 }\thinspace\thinspace}\hyperlink{sMetaMaybe}{\XLingPaperdottedtocline{\leveltwoindent}{\leveltwowidth}{{8.1 } Metadonnées du session.}{27}
}\settowidth{\leveltwoindent}{{8 }\ }\settowidth{\leveltwowidth}{{8.2 }\thinspace\thinspace}\hyperlink{sAddContributors}{\XLingPaperdottedtocline{\leveltwoindent}{\leveltwowidth}{{8.2 } Ajouter les contributeurs de session}{27}
}\settowidth{\leveltwoindent}{{8 }\ }\settowidth{\leveltwowidth}{{8.3 }\thinspace\thinspace}\hyperlink{sStartAnnot}{\XLingPaperdottedtocline{\leveltwoindent}{\leveltwowidth}{{8.3 } Faire l'annotation}{29}
}\settowidth{\leveltwoindent}{{8 }\ {8.3 }\ }\settowidth{\leveltwowidth}{{8.3.1 }\thinspace\thinspace}\hyperlink{sVidtoAudio}{\XLingPaperdottedtocline{\leveltwoindent}{\leveltwowidth}{{8.3.1 } Créer une version audio du vidéo.}{29}
}\settowidth{\leveltwoindent}{{8 }\ {8.3 }\ }\settowidth{\leveltwowidth}{{8.3.2 }\thinspace\thinspace}\hyperlink{sAutoSeg}{\XLingPaperdottedtocline{\leveltwoindent}{\leveltwowidth}{{8.3.2 } Segmentation Automatique}{29}
}\settowidth{\leveltwoindent}{{8 }\ {8.3 }\ }\settowidth{\leveltwowidth}{{8.3.3 }\thinspace\thinspace}\hyperlink{sAdjustSeg}{\XLingPaperdottedtocline{\leveltwoindent}{\leveltwowidth}{{8.3.3 } Changer la Segmentation}{31}
}\settowidth{\leveltwoindent}{{8 }\ {8.3 }\ }\settowidth{\leveltwowidth}{{8.3.4 }\thinspace\thinspace}\hyperlink{sManSeg}{\XLingPaperdottedtocline{\leveltwoindent}{\leveltwowidth}{{8.3.4 } Segmentation Manuelle}{31}
}\settowidth{\leveltwoindent}{{8 }\ }\settowidth{\leveltwowidth}{{8.4 }\thinspace\thinspace}\hyperlink{sCarefulSpeech}{\XLingPaperdottedtocline{\leveltwoindent}{\leveltwowidth}{{8.4 } Transcription en discours soigneux.}{32}
}\settowidth{\leveltwoindent}{{8 }\ }\settowidth{\leveltwowidth}{{8.5 }\thinspace\thinspace}\hyperlink{sOralTransl}{\XLingPaperdottedtocline{\leveltwoindent}{\leveltwowidth}{{8.5 } Transcription Orale}{34}
}\settowidth{\leveltwoindent}{{8 }\ }\settowidth{\leveltwowidth}{{8.6 }\thinspace\thinspace}\hyperlink{sWrittenTrans}{\XLingPaperdottedtocline{\leveltwoindent}{\leveltwowidth}{{8.6 } Rédiger la Transcription}{36}
}\settowidth{\leveltwoindent}{{8 }\ }\settowidth{\leveltwowidth}{{8.7 }\thinspace\thinspace}\hyperlink{sWritTransl}{\XLingPaperdottedtocline{\leveltwoindent}{\leveltwowidth}{{8.7 } Rédiger la traduction}{38}
}\vspace{0pt}\hyperlink{pArchiving}{\centering{Part IV Archivage\\}}\hyperlink{cExport}{\XLingPaperdottedtocline{0pt}{0pt}{9 Exportation des données}{40}
}\settowidth{\leveltwoindent}{{9 }\ }\settowidth{\leveltwowidth}{{9.1 }\thinspace\thinspace}\hyperlink{sSubtitles}{\XLingPaperdottedtocline{\leveltwoindent}{\leveltwowidth}{{9.1 } Exporter des sous-titres.}{40}
}\settowidth{\leveltwoindent}{{9 }\ }\settowidth{\leveltwowidth}{{9.2 }\thinspace\thinspace}\hyperlink{sAudacityLabels}{\XLingPaperdottedtocline{\leveltwoindent}{\leveltwowidth}{{9.2 } Exporter des étiquettes Audacity.}{40}
}\settowidth{\leveltwoindent}{{9 }\ }\settowidth{\leveltwowidth}{{9.3 }\thinspace\thinspace}\hyperlink{sExportSession}{\XLingPaperdottedtocline{\leveltwoindent}{\leveltwowidth}{{9.3 } Exporter des sessions}{40}
}\settowidth{\leveltwoindent}{{9 }\ }\settowidth{\leveltwowidth}{{9.4 }\thinspace\thinspace}\hyperlink{sExportPeople}{\XLingPaperdottedtocline{\leveltwoindent}{\leveltwowidth}{{9.4 } Exporter des personnes}{41}
}\hyperlink{cArchive}{\XLingPaperdottedtocline{0pt}{0pt}{10 Archivage}{42}
}\settowidth{\leveltwoindent}{{10 }\ }\settowidth{\leveltwowidth}{{10.1 }\thinspace\thinspace}\hyperlink{sArchiveRamp}{\XLingPaperdottedtocline{\leveltwoindent}{\leveltwowidth}{{10.1 } Archive avec RAMP (SIL) ...}{42}
}\settowidth{\leveltwoindent}{{10 }\ }\settowidth{\leveltwowidth}{{10.2 }\thinspace\thinspace}\hyperlink{sIMDI}{\XLingPaperdottedtocline{\leveltwoindent}{\leveltwowidth}{{10.2 } Achiver avec IMDI}{42}
}\clearpage
\pagenumbering{arabic}{\clearpage
\XLingPaperneedspace{3\baselineskip}\noindent
\fontsize{18}{21.599999999999998}\selectfont \textbf{{\centering
\thispagestyle{empty}\raisebox{\baselineskip}[0pt]{\pdfbookmark[1]{Part I Introduction}{SayMSuppMan}}\raisebox{\baselineskip}[0pt]{\protect\hypertarget{SayMSuppMan}{}}Part I\\}}}\par{}
\vspace{10.8pt}{\XLingPaperneedspace{3\baselineskip}\noindent
\fontsize{18}{21.599999999999998}\selectfont \textbf{{\centering
Introduction\\}}}\par{}
\vspace{21.6pt}
\begin{mdframed}
[backgroundcolor=FTColorA,skipabove=3pt,skipbelow=3pt,innermargin=2cm,outermargin=2cm,innertopmargin=.03in,innerbottommargin=.03in,innerleftmargin=.125in,innerrightmargin=.125in,align=left]\vspace{0pt}\indent Remarque: Ce document n'est q'une premiere ebauche. Le document final sera disponible sur \href{Outilingua.net}{\textcolor[rgb]{0,0,1}{\uline{Outilingua.net}}}.\par{}\end{mdframed}
\vspace{0pt}\indent En tant que préparateur des documents relatifs à l'organisation et au fonctionnement de langue, vous accumulez rapidement un grand nombre d'enregistrements source et de modification basés sur eux. Vous devez gérer ces enregistrements, documenter le consentement éclairé, transcrire, traduire, entrer des métadonnées et, enfin, soumettre à une archive numérique. En cours de route, vous devez garder tous ces fichiers bien organisés et étiquetés. Vous aurez envie de garder une trace des objectifs du projet afin d'émerger avec la couverture souhaitée dans des domaines tels que le genre, la spontanéité et les rôles sociaux de l'orateur. Vous aurez besoin d'aide pour suivre l'emplacement de chaque session dans votre flux de travail. Ainsi, SayMore peut vous aider à faire tout cela\par{}\vspace{6pt}\vspace{0pt}\indent {\textbf{En résumé, SayMore est un outil qui vous aide à organiser, transcrire, et archiver du vidéo, de l'audio, des images et divers autres fichiers avec leurs métadonnées.}}\par{}\vspace{6pt}\clearpage
\thispagestyle{bodyfirstpage}\markboth{Installation du logiciel Saymore}{Installation du logiciel Saymore}
\XLingPaperaddtocontents{sInstall}{\XLingPaperneedspace{3\baselineskip}\noindent
\fontsize{18}{21.599999999999998}\selectfont \textbf{{\centering
\raisebox{\baselineskip}[0pt]{\protect\hypertarget{sInstall}{}}\raisebox{\baselineskip}[0pt]{\pdfbookmark[1]{1 Installation du logiciel Saymore}{sInstall}}1\\}}}\par{}
\vspace{10.8pt}{\XLingPaperneedspace{3\baselineskip}\noindent
\fontsize{18}{21.599999999999998}\selectfont \textbf{{\centering
Installation du logiciel Saymore\\}}}\par{}
\vspace{21.6pt}\vspace{0pt}\indent Saymore peut être téléchargé par le lien \href{https://software.sil.org/saymore/download/ }{\textcolor[rgb]{0,0,1}{\uline{https://software.sil.org/saymore/download/}}} . Il est à noter que SayMore exige des conditions techniques tels que: Windows 7, 8 ou encore 10, également PDF reader et Microsoft Net 4.6.\par{}\vspace{6pt}\vspace{0pt}\indent Si vous allez hors connexion, SayMore peut convertir les fichiers multimédias de ceux de votre appareil pour un archivage à long terme. Pour certaines de ces conversions, il utilise un outil Open Source nommé {\textbf{FFmpeg}}. Normalement, il sera téléchargé la première fois que vous en aurez besoin. Mais si vous allez utiliser SayMore hors Internet à moindre coût, assurez-vous de faire une conversion de test avant de quitter une connexion Internet.\par{}\vspace{6pt}\vspace{0pt}\indent Après l'installation, vous allez voir:\par{}\vspace{6pt}\vspace{10pt plus 2pt minus 1pt}\setbox0=\vbox{\protect\raggedright\leavevmode
\vspace*{0pt}{\XeTeXpicfile "../images/fr/Ouvrir_Creer un nouveau_fr.png" scaled 750}\\[0pt]\protect\hypertarget{fSplashSayMore}{}\XLingPaperaddtocontents{fSplashSayMore}\textit{{Figure }}\textit{{1.1}}\textit{{ A la première ouverture de Saymore\\}}}\box0\par{}\vspace{10pt plus 2pt minus 1pt}\vspace{0pt}\vspace{6pt}\vspace{0pt}\indent Dans cette fenêtre, vous pouvez rouvrir un projet, chercher un projet existant dans un support USB ... , et créer nouveau, ou un projet vide.\par{}\vspace{6pt}{\XLingPaperneedspace{3\baselineskip}
\noindent\rule{\textwidth}{1pt}
{}\penalty10000\vspace{3pt}\XLingPaperneedspace{3\baselineskip}\noindent
\fontsize{12}{14.399999999999999}\selectfont \textbf{{\noindent
\raisebox{\baselineskip}[0pt]{\pdfbookmark[2]{{1.1 } Changer la langue d'interface utilisateur}{sInterface}}\raisebox{\baselineskip}[0pt]{\protect\hypertarget{sInterface}{}}{1.1 }Changer la langue d'interface utilisateur}}\markright{Changer la langue d'interface utilisateur}
\XLingPaperaddtocontents{sInterface}}\par{}
\penalty10000\vspace{10pt}\penalty10000\vspace{0pt}\indent Changez la langue d'interface utilisateur:\\\par{}\vspace{6pt}\vspace{10pt plus 2pt minus 1pt}\setbox0=\vbox{\protect\raggedright\leavevmode
\vspace*{0pt}{\XeTeXpicfile "../images/fr/Langue_d_interface_utilisateur_fr.png" scaled 750}\\[0pt]\protect\hypertarget{flangue}{}\XLingPaperaddtocontents{flangue}\textit{{Figure }}\textit{{1.2}}\textit{{ Langue d'Interface utilisateur menu\\}}}\box0\par{}\vspace{10pt plus 2pt minus 1pt}\vspace{0pt}\indent Si votre langue n'est pas là, ou vous voulez changez la translation d'interface utilisateur:\par{}{\parskip .5pt plus 1pt minus 1pt

\vspace{\baselineskip}

{\setlength{\XLingPapertempdim}{\XLingPaperbulletlistitemwidth+6em}\leftskip\XLingPapertempdim\relax
\interlinepenalty10000
\XLingPaperlistitem{6em}{\XLingPaperbulletlistitemwidth}{•}{Cliquez sur la {\textup{\textmd{\textcolor[rgb]{0,0.2,0.8}{\uline{Je veux localiser SayMore pour une autre langue...}}}}} Cet ourvrir le finetre pour changez, mais en anglais.}\vspace{3pt}}
{\setlength{\XLingPapertempdim}{\XLingPaperbulletlistitemwidth+6em}\leftskip\XLingPapertempdim\relax
\interlinepenalty10000
\XLingPaperlistitem{6em}{\XLingPaperbulletlistitemwidth}{•}{C'est aussi possible de téléchargez une autre langue d'interface et aussi un autre Accès au fichier de Protocole par un autre utilisateur.}}
\vspace{\baselineskip}
}
\begin{mdframed}
[backgroundcolor=FTColorA,skipabove=3pt,skipbelow=3pt,innermargin=2cm,outermargin=2cm,innertopmargin=.03in,innerbottommargin=.03in,innerleftmargin=.125in,innerrightmargin=.125in,align=left]\vspace{0pt}\indent Si toutes les sélections n'ont pas été modifiées, fermer SayMore et le rouvrir\par{}\end{mdframed}
\vspace{0pt}\clearpage
\thispagestyle{bodyfirstpage}\markboth{Un Aperçu autour de SayMore}{Un Aperçu autour de SayMore}
\XLingPaperaddtocontents{cTour}{\XLingPaperneedspace{3\baselineskip}\noindent
\fontsize{18}{21.599999999999998}\selectfont \textbf{{\centering
\raisebox{\baselineskip}[0pt]{\protect\hypertarget{cTour}{}}\raisebox{\baselineskip}[0pt]{\pdfbookmark[1]{2 Un Aperçu autour de SayMore}{cTour}}2\\}}}\par{}
\vspace{10.8pt}{\XLingPaperneedspace{3\baselineskip}\noindent
\fontsize{18}{21.599999999999998}\selectfont \textbf{{\centering
Un Aperçu autour de SayMore\\}}}\par{}
\vspace{21.6pt}\vspace{0pt}\indent SayMore ne contient que quatre menus et trois onglets. Les menus {\textbf{Session}} et {\textbf{Personne}} ont la couleur grise quand l'onglet {\textbf{Project}} est activé.\par{}\vspace{6pt}\vspace{10pt plus 2pt minus 1pt}\setbox0=\vbox{\protect\raggedright\leavevmode
\vspace*{0pt}{\XeTeXpicfile "../images/fr/MenusetOnglet_fr.png" scaled 750}\\[0pt]\protect\hypertarget{fOnglets}{}\XLingPaperaddtocontents{fOnglets}\textit{{Figure }}\textit{{2.1}}\textit{{ Les menus et les onglets\\}}}\box0\par{}\vspace{10pt plus 2pt minus 1pt}\vspace{0pt}\indent Les prochaines sections vous aideront à trouver les mises en page\par{}\vspace{6pt}{\XLingPaperneedspace{3\baselineskip}
\noindent\rule{\textwidth}{1pt}
{}\penalty10000\vspace{3pt}\XLingPaperneedspace{3\baselineskip}\noindent
\fontsize{12}{14.399999999999999}\selectfont \textbf{{\noindent
\raisebox{\baselineskip}[0pt]{\pdfbookmark[2]{{2.1 } Onglet Projet}{sProjectTab}}\raisebox{\baselineskip}[0pt]{\protect\hypertarget{sProjectTab}{}}{2.1 }Onglet Projet}}\markright{Onglet Projet}
\XLingPaperaddtocontents{sProjectTab}}\par{}
\penalty10000\vspace{10pt}\penalty10000\vspace{0pt}\indent Cet onglet projet nous permet spécifier et de conserver la description du projet\par{}\vspace{6pt}\vspace{10pt plus 2pt minus 1pt}\setbox0=\vbox{\protect\raggedright\leavevmode
\vspace*{0pt}{\XeTeXpicfile "../images/fr/ongletProjet_fr.png" scaled 750}\\[0pt]\protect\hypertarget{f-NeedsALabel-.xlingpaper.1..styledPaper.1..lingPaper.1..part.1..chapter.2..section1.1..figure.2.}{}\XLingPaperaddtocontents{f-NeedsALabel-.xlingpaper.1..styledPaper.1..lingPaper.1..part.1..chapter.2..section1.1..figure.2.}\textit{{Figure }}\textit{{2.2}}\textit{{ Fenetre de Onglet Projet\\}}}\box0\par{}\vspace{10pt plus 2pt minus 1pt}\vspace{0pt}\indent Dans cet onglet projet, nous avons un petit menu qui permet de modifier le projet actuel:\par{}\vspace{6pt}\vspace{10pt plus 2pt minus 1pt}\setbox0=\vbox{\protect\centering \leavevmode
\vspace*{0pt}{\XeTeXpicfile "../images/fr/ongletProjet_petitmenu_fr.png" scaled 750}\\[0pt]\protect\hypertarget{f-NeedsALabel-.xlingpaper.1..styledPaper.1..lingPaper.1..part.1..chapter.2..section1.1..figure.4.}{}\XLingPaperaddtocontents{f-NeedsALabel-.xlingpaper.1..styledPaper.1..lingPaper.1..part.1..chapter.2..section1.1..figure.4.}\textit{{Figure }}\textit{{2.3}}\textit{{ Onglet Projet petit menu\\}}}\box0\par{}\vspace{10pt plus 2pt minus 1pt}{\parskip .5pt plus 1pt minus 1pt

\vspace{\baselineskip}

{\setlength{\XLingPapertempdim}{\XLingPaperbulletlistitemwidth+6em}\leftskip\XLingPapertempdim\relax
\interlinepenalty10000
\XLingPaperlistitem{6em}{\XLingPaperbulletlistitemwidth}{•}{{\textbf{A propos de ce projet}}\\{\textit{Ici, on peut mettre les informations se rapportant autour de votre projet, à savoir le titre, la description, la localisation etc...}}}\vspace{3pt}}
{\setlength{\XLingPapertempdim}{\XLingPaperbulletlistitemwidth+6em}\leftskip\XLingPapertempdim\relax
\interlinepenalty10000
\XLingPaperlistitem{6em}{\XLingPaperbulletlistitemwidth}{•}{{\textbf{Protocole d'accès}}\\{\textit{ici ça nous permet de déterminer qui peut avoir accès au archives}}}\vspace{3pt}}
{\setlength{\XLingPapertempdim}{\XLingPaperbulletlistitemwidth+6em}\leftskip\XLingPapertempdim\relax
\interlinepenalty10000
\XLingPaperlistitem{6em}{\XLingPaperbulletlistitemwidth}{•}{{\textbf{Document de description}}\\{\textit{Ici ça nous permet d'ajouter les documents qui décrivent le projet}}}\vspace{3pt}}
{\setlength{\XLingPapertempdim}{\XLingPaperbulletlistitemwidth+6em}\leftskip\XLingPapertempdim\relax
\interlinepenalty10000
\XLingPaperlistitem{6em}{\XLingPaperbulletlistitemwidth}{•}{{\textbf{Autres documents}}\\{\textit{Ici ça nous permet d'ajouter les documents qui ne s'accordent pas ailleurs. Ex: information sur le financement des projets}}}\vspace{3pt}}
{\setlength{\XLingPapertempdim}{\XLingPaperbulletlistitemwidth+6em}\leftskip\XLingPapertempdim\relax
\interlinepenalty10000
\XLingPaperlistitem{6em}{\XLingPaperbulletlistitemwidth}{•}{{\textbf{Progrès}}\\{\textit{Ici permet de regarder le progrès du projet, on peut également copier, enregistrer ou imprimer}}.}}
\vspace{\baselineskip}
}\vspace{0pt}{\XLingPaperneedspace{3\baselineskip}
\noindent\rule{\textwidth}{1pt}
{}\penalty10000\vspace{3pt}\XLingPaperneedspace{3\baselineskip}\noindent
\fontsize{12}{14.399999999999999}\selectfont \textbf{{\noindent
\raisebox{\baselineskip}[0pt]{\pdfbookmark[2]{{2.2 } Onglet Session}{sSessionTab}}\raisebox{\baselineskip}[0pt]{\protect\hypertarget{sSessionTab}{}}{2.2 }Onglet Session}}\markright{Onglet Session}
\XLingPaperaddtocontents{sSessionTab}}\par{}
\penalty10000\vspace{10pt}\penalty10000\vspace{0pt}\indent L'onglet Sessions est l'endroit où vous trouverez les informations relatives à chaque session ainsi que ce qui a été fait dans chaque session.\par{}\vspace{6pt}\vspace{0pt}\indent La fenêtre de l'onglet Session:\\\par{}\vspace{6pt}\vspace{10pt plus 2pt minus 1pt}\setbox0=\vbox{\protect\raggedright\leavevmode
\vspace*{0pt}{\XeTeXpicfile "../images/fr/TAB_Sessions_avecExamples_fr.png" scaled 750}\\[0pt]\protect\hypertarget{f-NeedsALabel-.xlingpaper.1..styledPaper.1..lingPaper.1..part.1..chapter.2..section1.2..figure.2.}{}\XLingPaperaddtocontents{f-NeedsALabel-.xlingpaper.1..styledPaper.1..lingPaper.1..part.1..chapter.2..section1.2..figure.2.}\textit{{Figure }}\textit{{2.4}}\textit{{ Sessions tab avec example\\}}}\box0\par{}\vspace{10pt plus 2pt minus 1pt}\vspace{0pt}\indent Le côté gauche de la fenêtre est l'endroit où vous verrez vos sessions. Des colonnes permettent d'afficher l'ID, le titre, les étapes et le statut de la session. Vous pouvez activer d'autres colonnes facultatives (Date, Genre, Emplacement) en cliquant sur \vspace*{0pt}{\XeTeXpicfile "../images/en/TAB_Sessions_ADDCOLUMNS.png" scaled 750} en sélectionnant ceux que vous voulez voir.\par{}\vspace{6pt}\vspace{0pt}\indent La fenêtre supérieure droite montre quels fichiers composent cette session. La fenêtre en bas à droite affiche les informations et le contenu à remplir en fonction de ce que vous avez sélectionné dans la fenêtre en haut à droite. Si vous souhaitez ajouter de la documentation ou ajouter le fichier audio pour cette session, cliquez sur \vspace*{0pt}{\XeTeXpicfile "../images/fr/ajouterdesfichiers_fr.png" scaled 750} et choisir les fichiers désirables.\par{}{\XLingPaperneedspace{3\baselineskip}
\noindent\rule{\textwidth}{1pt}
{}\penalty10000\vspace{3pt}\XLingPaperneedspace{3\baselineskip}\noindent
\fontsize{12}{14.399999999999999}\selectfont \textbf{{\noindent
\raisebox{\baselineskip}[0pt]{\pdfbookmark[2]{{2.3 } Onglet Personne}{sPeopleTab}}\raisebox{\baselineskip}[0pt]{\protect\hypertarget{sPeopleTab}{}}{2.3 }Onglet Personne}}\markright{Onglet Personne}
\XLingPaperaddtocontents{sPeopleTab}}\par{}
\penalty10000\vspace{10pt}\penalty10000\vspace{0pt}\indent L'onglet Personnes vous permet de suivre tous vos participants. Voici où vous documenterez leurs informations biologiques, contributions et notes :\par{}\vspace{6pt}\vspace{10pt plus 2pt minus 1pt}\setbox0=\vbox{\protect\raggedright\leavevmode
\vspace*{0pt}{\XeTeXpicfile "../images/fr/OngletPersonnes_avecExamples_fr.png" scaled 750}\\[0pt]\protect\hypertarget{f-NeedsALabel-.xlingpaper.1..styledPaper.1..lingPaper.1..part.1..chapter.2..section1.3..figure.2.}{}\XLingPaperaddtocontents{f-NeedsALabel-.xlingpaper.1..styledPaper.1..lingPaper.1..part.1..chapter.2..section1.3..figure.2.}\textit{{Figure }}\textit{{2.5}}\textit{{ Onglet Personne avec example\\}}}\box0\par{}\vspace{10pt plus 2pt minus 1pt}\vspace{0pt}\indent Aussi vous pouvez ajouter des fichiers que vous avez besoin ici en cliquant \vspace*{0pt}{\XeTeXpicfile "../images/fr/ajouterdesfichiers_fr.png" scaled 750} et choisir les fichiers désirables.\par{}\clearpage
\thispagestyle{bodyfirstpage}\markboth{Présentation des Menus}{Présentation des Menus}
\XLingPaperaddtocontents{sMenus}{\XLingPaperneedspace{3\baselineskip}\noindent
\fontsize{18}{21.599999999999998}\selectfont \textbf{{\centering
\raisebox{\baselineskip}[0pt]{\protect\hypertarget{sMenus}{}}\raisebox{\baselineskip}[0pt]{\pdfbookmark[1]{3 Présentation des Menus}{sMenus}}3\\}}}\par{}
\vspace{10.8pt}{\XLingPaperneedspace{3\baselineskip}\noindent
\fontsize{18}{21.599999999999998}\selectfont \textbf{{\centering
Présentation des Menus\\}}}\par{}
\vspace{21.6pt}\vspace{0pt}\vspace{6pt}{\XLingPaperneedspace{3\baselineskip}
\noindent\rule{\textwidth}{1pt}
{}\penalty10000\vspace{3pt}\XLingPaperneedspace{3\baselineskip}\noindent
\fontsize{12}{14.399999999999999}\selectfont \textbf{{\noindent
\raisebox{\baselineskip}[0pt]{\pdfbookmark[2]{{3.1 } Menu: Projet}{sMenuProject}}\raisebox{\baselineskip}[0pt]{\protect\hypertarget{sMenuProject}{}}{3.1 }Menu: Projet}}\markright{Menu: Projet}
\XLingPaperaddtocontents{sMenuProject}}\par{}
\penalty10000\vspace{10pt}\penalty10000\vspace{0pt}\indent Quand vous ouvrez le menu projet, vous avez cette présentation:\par{}\vspace{6pt}\vspace{10pt plus 2pt minus 1pt}\setbox0=\vbox{\protect\centering \leavevmode
\vspace*{0pt}{\XeTeXpicfile "../images/fr/mnuProjet_fr.png" scaled 750}\\[0pt]\protect\hypertarget{fMenuProjet}{}\XLingPaperaddtocontents{fMenuProjet}\textit{{Figure }}\textit{{3.1}}\textit{{ Menu Projet\\}}}\box0\par{}\vspace{10pt plus 2pt minus 1pt}
\begin{mdframed}
[backgroundcolor=FTColorA,skipabove=3pt,skipbelow=3pt,innermargin=2cm,outermargin=2cm,innertopmargin=.03in,innerbottommargin=.03in,innerleftmargin=.125in,innerrightmargin=.125in,align=left]\vspace{0pt}\indent Remarque:\par{}{\parskip .5pt plus 1pt minus 1pt

\vspace{\baselineskip}

{\setlength{\XLingPapertempdim}{\XLingPaperbulletlistitemwidth+6em}\leftskip\XLingPapertempdim\relax
\interlinepenalty10000
\XLingPaperlistitem{6em}{\XLingPaperbulletlistitemwidth}{•}{Comme vous pouvez le constater il n'y a pas d'option sauvegarde, parce que SayMore le fait pour vous}\vspace{3pt}}
{\setlength{\XLingPapertempdim}{\XLingPaperbulletlistitemwidth+6em}\leftskip\XLingPapertempdim\relax
\interlinepenalty10000
\XLingPaperlistitem{6em}{\XLingPaperbulletlistitemwidth}{•}{Il n'y a pas de commande supprimer le projet. Vous pouvez supprimer le projet dans vos dossiers Windows Explorer.}}
\vspace{\baselineskip}
}\end{mdframed}
{\XLingPaperneedspace{3\baselineskip}
\noindent\rule{\textwidth}{.4pt}
{}\penalty10000\vspace{3pt}\XLingPaperneedspace{3\baselineskip}\noindent
\fontsize{10}{12}\selectfont \textbf{{\noindent
\raisebox{\baselineskip}[0pt]{\pdfbookmark[3]{{3.1.1 } Quitter}{sQuitSaymore}}\raisebox{\baselineskip}[0pt]{\protect\hypertarget{sQuitSaymore}{}}{3.1.1 }Quitter}}\markright{Quitter}
\XLingPaperaddtocontents{sQuitSaymore}}\par{}
\penalty10000\vspace{10pt}\penalty10000\vspace{0pt}\indent Cela ferme l'application\par{}{\XLingPaperneedspace{3\baselineskip}
\noindent\rule{\textwidth}{1pt}
{}\penalty10000\vspace{3pt}\XLingPaperneedspace{3\baselineskip}\noindent
\fontsize{12}{14.399999999999999}\selectfont \textbf{{\noindent
\raisebox{\baselineskip}[0pt]{\pdfbookmark[2]{{3.2 } Tâches de l'onglet Sessions}{sMenuSession}}\raisebox{\baselineskip}[0pt]{\protect\hypertarget{sMenuSession}{}}{3.2 }Tâches de l'onglet Sessions}}\markright{Tâches de l'onglet Sessions}
\XLingPaperaddtocontents{sMenuSession}}\par{}
\penalty10000\vspace{10pt}\penalty10000\vspace{0pt}\indent Quand cliques sur l'onglet session, tu as une nouvelle fenêtre qui s'affiche et aussi le menu Session disponible\\En cliquant le menu session tu peux voir:\par{}\vspace{6pt}\vspace{10pt plus 2pt minus 1pt}\setbox0=\vbox{\protect\centering \leavevmode
\vspace*{0pt}{\XeTeXpicfile "../images/fr/mnuSession_fr.png" scaled 750}\\[0pt]\protect\hypertarget{fMenuSession}{}\XLingPaperaddtocontents{fMenuSession}\textit{{Figure }}\textit{{3.2}}\textit{{ Menu Session\\}}}\box0\par{}\vspace{10pt plus 2pt minus 1pt}\vspace{0pt}{\XLingPaperneedspace{3\baselineskip}
\noindent\rule{\textwidth}{1pt}
{}\penalty10000\vspace{3pt}\XLingPaperneedspace{3\baselineskip}\noindent
\fontsize{12}{14.399999999999999}\selectfont \textbf{{\noindent
\raisebox{\baselineskip}[0pt]{\pdfbookmark[2]{{3.3 } Tâches d'onglet Personnes}{sMenuPerson}}\raisebox{\baselineskip}[0pt]{\protect\hypertarget{sMenuPerson}{}}{3.3 }Tâches d'onglet Personnes}}\markright{Tâches d'onglet Personnes}
\XLingPaperaddtocontents{sMenuPerson}}\par{}
\penalty10000\vspace{10pt}\penalty10000\vspace{0pt}\indent Quand tu cliques l'onglet {\textbf{personne}} tu peux sélectionner le menu {\textbf{Personne}}\par{}\vspace{6pt}\vspace{10pt plus 2pt minus 1pt}\setbox0=\vbox{\protect\centering \leavevmode
\vspace*{0pt}{\XeTeXpicfile "../images/fr/mnuPersonne_fr.png" scaled 750}\\[0pt]\protect\hypertarget{fMenuPersonne}{}\XLingPaperaddtocontents{fMenuPersonne}\textit{{Figure }}\textit{{3.3}}\textit{{ Menu Personne\\}}}\box0\par{}\vspace{10pt plus 2pt minus 1pt}{\XLingPaperneedspace{3\baselineskip}
\noindent\rule{\textwidth}{1pt}
{}\penalty10000\vspace{3pt}\XLingPaperneedspace{3\baselineskip}\noindent
\fontsize{12}{14.399999999999999}\selectfont \textbf{{\noindent
\raisebox{\baselineskip}[0pt]{\pdfbookmark[2]{{3.4 } Tâches d'onglet Aide}{sMenuHelp}}\raisebox{\baselineskip}[0pt]{\protect\hypertarget{sMenuHelp}{}}{3.4 }Tâches d'onglet Aide}}\markright{Tâches d'onglet Aide}
\XLingPaperaddtocontents{sMenuHelp}}\par{}
\penalty10000\vspace{10pt}\penalty10000\vspace{0pt}\indent C'est dans cet onglet qu'on peux trouver l'aide pour l'usage de SayMore. Quand vous cliquez sur le menu aide vous pouvez voir:\par{}\vspace{6pt}\vspace{10pt plus 2pt minus 1pt}\setbox0=\vbox{\protect\centering \leavevmode
\vspace*{0pt}{\XeTeXpicfile "../images/fr/mnuAide_fr.png" scaled 750}\\[0pt]\protect\hypertarget{f-NeedsALabel-.xlingpaper.1..styledPaper.1..lingPaper.1..part.1..chapter.3..section1.4..figure.2.}{}\XLingPaperaddtocontents{f-NeedsALabel-.xlingpaper.1..styledPaper.1..lingPaper.1..part.1..chapter.3..section1.4..figure.2.}\textit{{Figure }}\textit{{3.4}}\textit{{ Menu Aide\\}}}\box0\par{}\vspace{10pt plus 2pt minus 1pt}{\XLingPaperneedspace{3\baselineskip}
\noindent\rule{\textwidth}{.4pt}
{}\penalty10000\vspace{3pt}\XLingPaperneedspace{3\baselineskip}\noindent
\fontsize{10}{12}\selectfont \textbf{{\noindent
\raisebox{\baselineskip}[0pt]{\pdfbookmark[3]{{3.4.1 } Ade...}{sHelp}}\raisebox{\baselineskip}[0pt]{\protect\hypertarget{sHelp}{}}{3.4.1 }Ade...}}\markright{Ade...}
\XLingPaperaddtocontents{sHelp}}\par{}
\penalty10000\vspace{10pt}\penalty10000\vspace{0pt}\indent Quand vous cliquez {\textbf{Aide...}} il ouvre une nouvelle fenêtre:\par{}\vspace{6pt}\vspace{10pt plus 2pt minus 1pt}\setbox0=\vbox{\protect\raggedright\leavevmode
\vspace*{0pt}{\XeTeXpicfile "../images/en/SayMore_HelpWindow_en.png" scaled 750}\\[0pt]\protect\hypertarget{f-NeedsALabel-.xlingpaper.1..styledPaper.1..lingPaper.1..part.1..chapter.2..section1.4..section2.2..figure.1.}{}\XLingPaperaddtocontents{f-NeedsALabel-.xlingpaper.1..styledPaper.1..lingPaper.1..part.1..chapter.2..section1.4..section2.2..figure.1.}\textit{{Figure }}\textit{{3.5}}\textit{{ SayMore Help Window\\}}}\box0\par{}\vspace{10pt plus 2pt minus 1pt}\vspace{0pt}\indent À partir de cette fenêtre, vous pouvez effectuer une recherche en utilisant l'arborescence du contenu, l'index (Recherche par mot-clé) et la recherche (rechercher un mot)\par{}{\XLingPaperneedspace{3\baselineskip}
\noindent\rule{\textwidth}{.4pt}
{}\penalty10000\vspace{3pt}\XLingPaperneedspace{3\baselineskip}\noindent
\fontsize{10}{12}\selectfont \textbf{{\noindent
\raisebox{\baselineskip}[0pt]{\pdfbookmark[3]{{3.4.2 } A propos...}{sAbout}}\raisebox{\baselineskip}[0pt]{\protect\hypertarget{sAbout}{}}{3.4.2 }A propos...}}\markright{A propos...}
\XLingPaperaddtocontents{sAbout}}\par{}
\penalty10000\vspace{10pt}\penalty10000\vspace{0pt}\indent Quand vous cliquez sur "A Propos..." une fenêtre s'ouvre en détaillant ce qui a été fait par SayMore et remercie ceux qui ont conçu SayMore. Il contient également des liens vers les composants / bibliothèques Open Source utilisés dans SayMore\par{}\clearpage
{\clearpage
\XLingPaperneedspace{3\baselineskip}\noindent
\fontsize{18}{21.599999999999998}\selectfont \textbf{{\centering
\thispagestyle{empty}\raisebox{\baselineskip}[0pt]{\pdfbookmark[1]{Part II Gestion d'un projet, des sessions, ou des personnes}{pGestion}}\raisebox{\baselineskip}[0pt]{\protect\hypertarget{pGestion}{}}Part II\\}}}\par{}
\vspace{10.8pt}{\XLingPaperneedspace{3\baselineskip}\noindent
\fontsize{18}{21.599999999999998}\selectfont \textbf{{\centering
Gestion d'un projet, des sessions, ou des personnes\\}}}\par{}
\vspace{21.6pt}\vspace{0pt}\vspace{6pt}\clearpage
\thispagestyle{bodyfirstpage}\markboth{La Creation ou l'ouverture d'un projet}{La Creation ou l'ouverture d'un projet}
\XLingPaperaddtocontents{cCreateProjectOver}{\XLingPaperneedspace{3\baselineskip}\noindent
\fontsize{18}{21.599999999999998}\selectfont \textbf{{\centering
\raisebox{\baselineskip}[0pt]{\protect\hypertarget{cCreateProjectOver}{}}\raisebox{\baselineskip}[0pt]{\pdfbookmark[1]{4 La Creation ou l'ouverture d'un projet}{cCreateProjectOver}}4\\}}}\par{}
\vspace{10.8pt}{\XLingPaperneedspace{3\baselineskip}\noindent
\fontsize{18}{21.599999999999998}\selectfont \textbf{{\centering
La Creation ou l'ouverture d'un projet\\}}}\par{}
\vspace{21.6pt}\vspace{0pt}\indent Avant de travailler dans SayMore, vous devez créer un projet. Vous devez créer un projet distinct pour chaque langue que vous analysez, mais vous n'avez pas besoin de créer un nouveau projet pour chaque fichier analysé.\par{}\vspace{6pt}{\XLingPaperneedspace{3\baselineskip}
\noindent\rule{\textwidth}{1pt}
{}\penalty10000\vspace{3pt}\XLingPaperneedspace{3\baselineskip}\noindent
\fontsize{12}{14.399999999999999}\selectfont \textbf{{\noindent
\raisebox{\baselineskip}[0pt]{\pdfbookmark[2]{{4.1 } Créer un projet}{sCreateProject}}\raisebox{\baselineskip}[0pt]{\protect\hypertarget{sCreateProject}{}}{4.1 }Créer un projet}}\markright{Créer un projet}
\XLingPaperaddtocontents{sCreateProject}}\par{}
\penalty10000\vspace{10pt}\penalty10000\vspace{0pt}\indent Quand vous cliquez sur le menu projet, vous allez dans {\textbf{Ouvrir / Créer un Projet}} :\par{}\vspace{6pt}\vspace{10pt plus 2pt minus 1pt}\setbox0=\vbox{\protect\raggedright\leavevmode
\vspace*{0pt}{\XeTeXpicfile "../images/fr/Ouvrir_Creer un nouveau_fr.png" scaled 750}\\[0pt]\protect\hypertarget{f-NeedsALabel-.xlingpaper.1..styledPaper.1..lingPaper.1..part.1..chapter.2..section1.1..section2.1..figure.2.}{}\XLingPaperaddtocontents{f-NeedsALabel-.xlingpaper.1..styledPaper.1..lingPaper.1..part.1..chapter.2..section1.1..section2.1..figure.2.}\textit{{Figure }}\textit{{4.1}}\textit{{ Ouvrir / Créer un Projet\\}}}\box0\par{}\vspace{10pt plus 2pt minus 1pt}\vspace{0pt}\indent En cliquant sur Créer un nouveau projet, vous avez cette fenêtre pour nommer le projet.\par{}\vspace{6pt}\vspace{10pt plus 2pt minus 1pt}\setbox0=\vbox{\protect\centering \leavevmode
\vspace*{0pt}{\XeTeXpicfile "../images/fr/NomDeNouveauProjet_fr.png" scaled 750}\\[0pt]\protect\hypertarget{fNewProFr}{}\XLingPaperaddtocontents{fNewProFr}\textit{{Figure }}\textit{{4.2}}\textit{{ Nommer le nouveau projet\\}}}\box0\par{}\vspace{10pt plus 2pt minus 1pt}\vspace{0pt}\indent Une fois que vous cliquez sur OK, SayMore s'ouvre avec un nouveau projet vide. Pour continuer à aller à :\\\hyperlink{cProjectMeta}{5} \hyperlink{cProjectMeta}{Remplir les metadonnées du projet}\par{}{\XLingPaperneedspace{3\baselineskip}
\noindent\rule{\textwidth}{1pt}
{}\penalty10000\vspace{3pt}\XLingPaperneedspace{3\baselineskip}\noindent
\fontsize{12}{14.399999999999999}\selectfont \textbf{{\noindent
\raisebox{\baselineskip}[0pt]{\pdfbookmark[2]{{4.2 } Ouvrir un projet existant}{sOpenProject}}\raisebox{\baselineskip}[0pt]{\protect\hypertarget{sOpenProject}{}}{4.2 }Ouvrir un projet existant}}\markright{Ouvrir un projet existant}
\XLingPaperaddtocontents{sOpenProject}}\par{}
\penalty10000\vspace{10pt}\penalty10000\vspace{0pt}\indent Quand vous ouvrez SayMore, la boîte de dialogue suivante s'affiche.\par{}\vspace{6pt}\vspace{10pt plus 2pt minus 1pt}\setbox0=\vbox{\protect\raggedright\leavevmode
\vspace*{0pt}{\XeTeXpicfile "../images/fr/Ouvrir_Creer un nouveau_fr.png" scaled 750}\\[0pt]\protect\hypertarget{fOuvrirExistant}{}\XLingPaperaddtocontents{fOuvrirExistant}\textit{{Figure }}\textit{{4.3}}\textit{{ Ouvrir / Créer un Projet\\}}}\box0\par{}\vspace{10pt plus 2pt minus 1pt}\vspace{0pt}\indent Cliquez sur le nom de votre projet pour l'ouvrir.\\{\textit{Le projet s'affiche.}}\par{}{\XLingPaperneedspace{3\baselineskip}
\noindent\rule{\textwidth}{1pt}
{}\penalty10000\vspace{3pt}\XLingPaperneedspace{3\baselineskip}\noindent
\fontsize{12}{14.399999999999999}\selectfont \textbf{{\noindent
\raisebox{\baselineskip}[0pt]{\pdfbookmark[2]{{4.3 } Ouvrir un autre projet}{sOpenOtherProject}}\raisebox{\baselineskip}[0pt]{\protect\hypertarget{sOpenOtherProject}{}}{4.3 }Ouvrir un autre projet}}\markright{Ouvrir un autre projet}
\XLingPaperaddtocontents{sOpenOtherProject}}\par{}
\penalty10000\vspace{10pt}\penalty10000\vspace{0pt}\indent Si vous travaillez dans un projet et désirez ouvrir un autre projet :\par{}\vspace{6pt}\vspace{0pt}\indent Du menu {\textbf{Projet}}, cliquez {\textbf{Ouvrir / Créer un Projet}} :\par{}\vspace{6pt}\vspace{10pt plus 2pt minus 1pt}\setbox0=\vbox{\protect\raggedright\leavevmode
\vspace*{0pt}{\XeTeXpicfile "../images/fr/Ouvrir_Creer un nouveau_fr.png" scaled 750}\\[0pt]\protect\hypertarget{fOuvrirAutre}{}\XLingPaperaddtocontents{fOuvrirAutre}\textit{{Figure }}\textit{{4.4}}\textit{{ Ouvrir / Créer un Projet\\}}}\box0\par{}\vspace{10pt plus 2pt minus 1pt}\vspace{0pt}\indent Cliquez sur le nom de votre projet pour l'ouvrir.\\{\textit{Le projet s'affiche.}}\par{}\clearpage
\thispagestyle{bodyfirstpage}\markboth{Remplir les metadonnées du projet}{Remplir les metadonnées du projet}
\XLingPaperaddtocontents{cProjectMeta}{\XLingPaperneedspace{3\baselineskip}\noindent
\fontsize{18}{21.599999999999998}\selectfont \textbf{{\centering
\raisebox{\baselineskip}[0pt]{\protect\hypertarget{cProjectMeta}{}}\raisebox{\baselineskip}[0pt]{\pdfbookmark[1]{5 Remplir les metadonnées du projet}{cProjectMeta}}5\\}}}\par{}
\vspace{10.8pt}{\XLingPaperneedspace{3\baselineskip}\noindent
\fontsize{18}{21.599999999999998}\selectfont \textbf{{\centering
Remplir les metadonnées du projet\\}}}\par{}
\vspace{21.6pt}\vspace{0pt}\indent Lorsque vous créez un nouveau projet ou même ouvrez un projet existant, il est important de vous assurer que les éléments de l'onglet Projet sont remplis autant que possible. Ce sera également le premier onglet ouvert lorsque vous créez un nouveau projet. L'onglet Projet est conçu pour collecter les métadonnées qui concernent l'ensemble du projet. Ces métadonnées et documents sont essentiels pour la découverte et l'utilisation de l'archive.\par{}\vspace{6pt}{\XLingPaperneedspace{3\baselineskip}
\noindent\rule{\textwidth}{1pt}
{}\penalty10000\vspace{3pt}\XLingPaperneedspace{3\baselineskip}\noindent
\fontsize{12}{14.399999999999999}\selectfont \textbf{{\noindent
\raisebox{\baselineskip}[0pt]{\pdfbookmark[2]{{5.1 } À propos de ce projet}{sAboutProj}}\raisebox{\baselineskip}[0pt]{\protect\hypertarget{sAboutProj}{}}{5.1 }À propos de ce projet}}\markright{À propos de ce projet}
\XLingPaperaddtocontents{sAboutProj}}\par{}
\penalty10000\vspace{10pt}\penalty10000{\parskip .5pt plus 1pt minus 1pt

{\setlength{\XLingPapertempdim}{\XLingPaperbulletlistitemwidth+6em}\leftskip\XLingPapertempdim\relax
\interlinepenalty10000
\XLingPaperlistitem{6em}{\XLingPaperbulletlistitemwidth}{•}{{\textbf{Title}} -}\vspace{3pt}}
{\setlength{\XLingPapertempdim}{\XLingPaperbulletlistitemwidth+6em}\leftskip\XLingPapertempdim\relax
\interlinepenalty10000
\XLingPaperlistitem{6em}{\XLingPaperbulletlistitemwidth}{•}{{\textbf{Description}} -}\vspace{3pt}}
{\setlength{\XLingPapertempdim}{\XLingPaperbulletlistitemwidth+6em}\leftskip\XLingPapertempdim\relax
\interlinepenalty10000
\XLingPaperlistitem{6em}{\XLingPaperbulletlistitemwidth}{•}{{\textbf{Vernaculaire}} -}\vspace{3pt}}
{\setlength{\XLingPapertempdim}{\XLingPaperbulletlistitemwidth+6em}\leftskip\XLingPapertempdim\relax
\interlinepenalty10000
\XLingPaperlistitem{6em}{\XLingPaperbulletlistitemwidth}{•}{{\textbf{Emlacement / Adresse}} -}\vspace{3pt}}
{\setlength{\XLingPapertempdim}{\XLingPaperbulletlistitemwidth+6em}\leftskip\XLingPapertempdim\relax
\interlinepenalty10000
\XLingPaperlistitem{6em}{\XLingPaperbulletlistitemwidth}{•}{{\textbf{Région}} -}\vspace{3pt}}
{\setlength{\XLingPapertempdim}{\XLingPaperbulletlistitemwidth+6em}\leftskip\XLingPapertempdim\relax
\interlinepenalty10000
\XLingPaperlistitem{6em}{\XLingPaperbulletlistitemwidth}{•}{{\textbf{Pays}} -}\vspace{3pt}}
{\setlength{\XLingPapertempdim}{\XLingPaperbulletlistitemwidth+6em}\leftskip\XLingPapertempdim\relax
\interlinepenalty10000
\XLingPaperlistitem{6em}{\XLingPaperbulletlistitemwidth}{•}{{\textbf{Continent}} -}\vspace{3pt}}
{\setlength{\XLingPapertempdim}{\XLingPaperbulletlistitemwidth+6em}\leftskip\XLingPapertempdim\relax
\interlinepenalty10000
\XLingPaperlistitem{6em}{\XLingPaperbulletlistitemwidth}{•}{{\textbf{Personne à contacter}} -}\vspace{3pt}}
{\setlength{\XLingPapertempdim}{\XLingPaperbulletlistitemwidth+6em}\leftskip\XLingPapertempdim\relax
\interlinepenalty10000
\XLingPaperlistitem{6em}{\XLingPaperbulletlistitemwidth}{•}{{\textbf{Titre du project de financement}} -}\vspace{3pt}}
{\setlength{\XLingPapertempdim}{\XLingPaperbulletlistitemwidth+6em}\leftskip\XLingPapertempdim\relax
\interlinepenalty10000
\XLingPaperlistitem{6em}{\XLingPaperbulletlistitemwidth}{•}{{\textbf{Disponible le}} -}\vspace{3pt}}
{\setlength{\XLingPapertempdim}{\XLingPaperbulletlistitemwidth+6em}\leftskip\XLingPapertempdim\relax
\interlinepenalty10000
\XLingPaperlistitem{6em}{\XLingPaperbulletlistitemwidth}{•}{{\textbf{Titulaire de droits}} -}\vspace{3pt}}
{\setlength{\XLingPapertempdim}{\XLingPaperbulletlistitemwidth+6em}\leftskip\XLingPapertempdim\relax
\interlinepenalty10000
\XLingPaperlistitem{6em}{\XLingPaperbulletlistitemwidth}{•}{{\textbf{Déposant}} -}}
\vspace{\baselineskip}
}{\XLingPaperneedspace{3\baselineskip}
\noindent\rule{\textwidth}{1pt}
{}\penalty10000\vspace{3pt}\XLingPaperneedspace{3\baselineskip}\noindent
\fontsize{12}{14.399999999999999}\selectfont \textbf{{\noindent
\raisebox{\baselineskip}[0pt]{\pdfbookmark[2]{{5.2 } Protocole d'accès}{sAccessProt}}\raisebox{\baselineskip}[0pt]{\protect\hypertarget{sAccessProt}{}}{5.2 }Protocole d'accès}}\markright{Protocole d'accès}
\XLingPaperaddtocontents{sAccessProt}}\par{}
\penalty10000\vspace{10pt}\penalty10000\vspace{0pt}\indent L'accès est la façon dont vous déterminez qui a l'autorisation d'obtenir ou de voir vos données archivées. Le protocole d'accès est utilisé pour déterminer l'accès à attribuer à un élément dans les archives. C'est ici que vous choisissez le protocole d'accès pour le projet.\par{}{\parskip .5pt plus 1pt minus 1pt

\vspace{\baselineskip}

{\setlength{\XLingPapertempdim}{\XLingPaperbulletlistitemwidth+6em}\leftskip\XLingPapertempdim\relax
\interlinepenalty10000
\XLingPaperlistitem{6em}{\XLingPaperbulletlistitemwidth}{•}{Si votre organization ditez contre:\\SIL members or others with permission to use REAP/RAMP, choose:}{\setlength{\XLingPaperlistitemindent}{\XLingPaperbulletlistitemwidth + 6em}
{\setlength{\XLingPapertempdim}{\XLingPaperbulletlistitemwidth+\XLingPaperlistitemindent}\leftskip\XLingPapertempdim\relax
\interlinepenalty10000
\XLingPaperlistitem{\XLingPaperlistitemindent}{\XLingPaperbulletlistitemwidth}{•}{REAP}}}\vspace{3pt}}
{\setlength{\XLingPapertempdim}{\XLingPaperbulletlistitemwidth+6em}\leftskip\XLingPapertempdim\relax
\interlinepenalty10000
\XLingPaperlistitem{6em}{\XLingPaperbulletlistitemwidth}{•}{Tous les autres, choissiez un de suivre:}{\setlength{\XLingPaperlistitemindent}{\XLingPaperbulletlistitemwidth + 6em}
{\setlength{\XLingPapertempdim}{\XLingPaperbulletlistitemwidth+\XLingPaperlistitemindent}\leftskip\XLingPapertempdim\relax
\interlinepenalty10000
\XLingPaperlistitem{\XLingPaperlistitemindent}{\XLingPaperbulletlistitemwidth}{•}{AILCA}\vspace{3pt}}
{\setlength{\XLingPapertempdim}{\XLingPaperbulletlistitemwidth+\XLingPaperlistitemindent}\leftskip\XLingPapertempdim\relax
\interlinepenalty10000
\XLingPaperlistitem{\XLingPaperlistitemindent}{\XLingPaperbulletlistitemwidth}{•}{AILLA}\vspace{3pt}}
{\setlength{\XLingPapertempdim}{\XLingPaperbulletlistitemwidth+\XLingPaperlistitemindent}\leftskip\XLingPapertempdim\relax
\interlinepenalty10000
\XLingPaperlistitem{\XLingPaperlistitemindent}{\XLingPaperbulletlistitemwidth}{•}{ANLA}\vspace{3pt}}
{\setlength{\XLingPapertempdim}{\XLingPaperbulletlistitemwidth+\XLingPaperlistitemindent}\leftskip\XLingPapertempdim\relax
\interlinepenalty10000
\XLingPaperlistitem{\XLingPaperlistitemindent}{\XLingPaperbulletlistitemwidth}{•}{ELAR}\vspace{3pt}}
{\setlength{\XLingPapertempdim}{\XLingPaperbulletlistitemwidth+\XLingPaperlistitemindent}\leftskip\XLingPapertempdim\relax
\interlinepenalty10000
\XLingPaperlistitem{\XLingPaperlistitemindent}{\XLingPaperbulletlistitemwidth}{•}{TLA}\vspace{3pt}}
{\setlength{\XLingPapertempdim}{\XLingPaperbulletlistitemwidth+\XLingPaperlistitemindent}\leftskip\XLingPapertempdim\relax
\interlinepenalty10000
\XLingPaperlistitem{\XLingPaperlistitemindent}{\XLingPaperbulletlistitemwidth}{•}{Custom}}}}
\vspace{\baselineskip}
}\vspace{0pt}\indent Si vous choisissez "Personnalisé", une boîte vide s'affiche dans laquelle vous pouvez saisir des options d'accès personnalisées séparées par des virgules. Cela vous permet de définir différentes sessions enregistrées pour différents protocoles d'accès.\par{}{\XLingPaperneedspace{3\baselineskip}
\noindent\rule{\textwidth}{1pt}
{}\penalty10000\vspace{3pt}\XLingPaperneedspace{3\baselineskip}\noindent
\fontsize{12}{14.399999999999999}\selectfont \textbf{{\noindent
\raisebox{\baselineskip}[0pt]{\pdfbookmark[2]{{5.3 } Documents de description du projet}{sProDescDoc}}\raisebox{\baselineskip}[0pt]{\protect\hypertarget{sProDescDoc}{}}{5.3 }Documents de description du projet}}\markright{Documents de description du projet}
\XLingPaperaddtocontents{sProDescDoc}}\par{}
\penalty10000\vspace{10pt}\penalty10000\vspace{0pt}\indent C’est où vous ajoutez des documents décrivant le projet et le corpus. Lorsque le projet est archivé avec IMDI, les fichiers sont exportés vers une session spéciale nommée « Documents descriptifs de projet ».\par{}{\XLingPaperneedspace{3\baselineskip}
\noindent\rule{\textwidth}{1pt}
{}\penalty10000\vspace{3pt}\XLingPaperneedspace{3\baselineskip}\noindent
\fontsize{12}{14.399999999999999}\selectfont \textbf{{\noindent
\raisebox{\baselineskip}[0pt]{\pdfbookmark[2]{{5.4 } Autre documents}{sOtherDocs}}\raisebox{\baselineskip}[0pt]{\protect\hypertarget{sOtherDocs}{}}{5.4 }Autre documents}}\markright{Autre documents}
\XLingPaperaddtocontents{sOtherDocs}}\par{}
\penalty10000\vspace{10pt}\penalty10000\vspace{0pt}\indent C’est où vous ajoutez tous les autres documents au niveau du projet qui n’appartient pas dans les Documents de Description. Lorsque le projet est archivé avec IMDI ici, les fichiers sont exportés vers une session spéciale nommée « Other Project Documents ».\par{}{\XLingPaperneedspace{3\baselineskip}
\noindent\rule{\textwidth}{1pt}
{}\penalty10000\vspace{3pt}\XLingPaperneedspace{3\baselineskip}\noindent
\fontsize{12}{14.399999999999999}\selectfont \textbf{{\noindent
\raisebox{\baselineskip}[0pt]{\pdfbookmark[2]{{5.5 } Progrès du projet}{sProProg}}\raisebox{\baselineskip}[0pt]{\protect\hypertarget{sProProg}{}}{5.5 }Progrès du projet}}\markright{Progrès du projet}
\XLingPaperaddtocontents{sProProg}}\par{}
\penalty10000\vspace{10pt}\penalty10000\vspace{0pt}\indent Cette fenêtre est où vous pouvez suivre l’état d’avancement des projets :\par{}\vspace{6pt}\vspace{10pt plus 2pt minus 1pt}\setbox0=\vbox{\protect\raggedright\leavevmode
\vspace*{0pt}{\XeTeXpicfile "../images/fr/Progress_fr.png" scaled 750}\\[0pt]\protect\hypertarget{fProgresFr}{}\XLingPaperaddtocontents{fProgresFr}\textit{{Figure }}\textit{{5.1}}\textit{{ Progres\\}}}\box0\par{}\vspace{10pt plus 2pt minus 1pt}\vspace{0pt}\indent Comme vous pouvez le voir dans l’image ci-dessus, c’est un graphique très utile qui peut être copié, pour être collées dans un rapport, enregistré comme un fichier {\textbf{HTML}} individuel ou imprimé.\par{}\clearpage
\thispagestyle{bodyfirstpage}\markboth{Gestion des sessions}{Gestion des sessions}
\XLingPaperaddtocontents{cSessions}{\XLingPaperneedspace{3\baselineskip}\noindent
\fontsize{18}{21.599999999999998}\selectfont \textbf{{\centering
\raisebox{\baselineskip}[0pt]{\protect\hypertarget{cSessions}{}}\raisebox{\baselineskip}[0pt]{\pdfbookmark[1]{6 Gestion des sessions}{cSessions}}6\\}}}\par{}
\vspace{10.8pt}{\XLingPaperneedspace{3\baselineskip}\noindent
\fontsize{18}{21.599999999999998}\selectfont \textbf{{\centering
Gestion des sessions\\}}}\par{}
\vspace{21.6pt}\vspace{0pt}\indent Si vous venez de créer un nouveau projet, l'onglet sessions sera vide. Sinon, vous verrez toutes vos sessions précédentes et ce qui a été enregistré dans chaque session.\par{}\vspace{6pt}{\XLingPaperneedspace{3\baselineskip}
\noindent\rule{\textwidth}{1pt}
{}\penalty10000\vspace{3pt}\XLingPaperneedspace{3\baselineskip}\noindent
\fontsize{12}{14.399999999999999}\selectfont \textbf{{\noindent
\raisebox{\baselineskip}[0pt]{\pdfbookmark[2]{{6.1 } Ajouter une Session}{sCreateSession}}\raisebox{\baselineskip}[0pt]{\protect\hypertarget{sCreateSession}{}}{6.1 }Ajouter une Session}}\markright{Ajouter une Session}
\XLingPaperaddtocontents{sCreateSession}}\par{}
\penalty10000\vspace{10pt}\penalty10000\vspace{0pt}\indent {\textbf{Nouveau}} vous permet de créer une nouvelle session vide. Vous devez ensuite ajouter manuellement tous les fichiers que vous souhaitez associer à la nouvelle session. Ceci est fait sur le côté droit de la fenêtre de l'onglet Sessions. Pour créer une nouvelle session, vous pouvez utiliser le menu déroulant Session ou, en bas de la fenêtre de l'onglet Session, des boutons:\par{}\vspace{6pt}\vspace{10pt plus 2pt minus 1pt}\setbox0=\vbox{\protect\centering \leavevmode
\vspace*{0pt}{\XeTeXpicfile "../images/fr/Onglet_Sessions_Buttons_fr.png" scaled 750}\\[0pt]\protect\hypertarget{f3nouveau}{}\XLingPaperaddtocontents{f3nouveau}\textit{{Figure }}\textit{{6.1}}\textit{{ Buttons de Onglet Sessions\\}}}\box0\par{}\vspace{10pt plus 2pt minus 1pt}\vspace{0pt}\indent Vous pouvez ajouter une session vide ou une session avec le contenu d'un nouvel enregistrement. De plus, vous pouvez ajouter plusieurs sessions en téléchargeant les enregistrements du jour directement sur les appareils (caméra, enregistreur audio, etc.). Dans ce cas, SayMore crée des ID pour chaque session. Chaque ID deviendra le nom de son dossier, ainsi que la première partie de la plupart des noms de fichiers dans le dossier. Ceux-ci sont automatiquement mis à jour si vous modifiez ultérieurement cette métadonnée Id.\par{}\vspace{6pt}
\begin{mdframed}
[backgroundcolor=FTColorA,skipabove=3pt,skipbelow=3pt,innermargin=2cm,outermargin=2cm,innertopmargin=.03in,innerbottommargin=.03in,innerleftmargin=.125in,innerrightmargin=.125in,align=left]\vspace{0pt}{\parskip .5pt plus 1pt minus 1pt

\vspace{\baselineskip}

{\setlength{\XLingPapertempdim}{\XLingPaperbulletlistitemwidth+6em}\leftskip\XLingPapertempdim\relax
\interlinepenalty10000
\XLingPaperlistitem{6em}{\XLingPaperbulletlistitemwidth}{•}{Vous pouvez ajouter d'autres fichiers, et non une nouvelle session}}
\vspace{\baselineskip}
}\end{mdframed}
{\XLingPaperneedspace{3\baselineskip}
\noindent\rule{\textwidth}{.4pt}
{}\penalty10000\vspace{3pt}\XLingPaperneedspace{3\baselineskip}\noindent
\fontsize{10}{12}\selectfont \textbf{{\noindent
\raisebox{\baselineskip}[0pt]{\pdfbookmark[3]{{6.1.1 } Nouveau à partir d'un appareil...}{sNewFromDevice}}\raisebox{\baselineskip}[0pt]{\protect\hypertarget{sNewFromDevice}{}}{6.1.1 }Nouveau à partir d'un appareil...}}\markright{Nouveau à partir d'un appareil...}
\XLingPaperaddtocontents{sNewFromDevice}}\par{}
\penalty10000\vspace{10pt}\penalty10000\vspace{0pt}\indent {\textbf{Nouveau à Partir d'un Périphérique }}Vous permet de créer une nouvelle session et ouvre\par{}\vspace{6pt}\vspace{0pt}\indent Quand vous avez sélectionner un dossier vous pouvez le voir:\par{}\vspace{6pt}\vspace{0pt}\indent Ici vous pouvez sélectionner tout ou partie des fichiers audio. Avec la boîte de médias, vous pouvez lire le fichier en sur brillance pour vous assurer que c'est le fichier désiré. SayMore se souviendra du chemin vers ce fichier, donc la prochaine fois que vous cliquerez sur {\textbf{Nouvelle session à partir d'un appareil}} vous verrez ce dossier toujours sélectionné. Chaque fichier audio sélectionné sera créé dans sa propre session. Comme vous pouvez le voir dans {\textit{Figure}} \hyperlink{fnewsesapp}{6.2} qu'il y a trois fichiers sélectionnés, donc 3 nouvelles sessions seront créées.\par{}\vspace{6pt}\vspace{10pt plus 2pt minus 1pt}\setbox0=\vbox{\protect\raggedright\leavevmode
\vspace*{0pt}{\XeTeXpicfile "../images/fr/NouvellesSessions_d_appareil_fr.png" scaled 750}\\[0pt]\protect\hypertarget{fnewsesapp}{}\XLingPaperaddtocontents{fnewsesapp}\textit{{Figure }}\textit{{6.2}}\textit{{ Nouvelles sessions à partir d'appareil\\}}}\box0\par{}\vspace{10pt plus 2pt minus 1pt}\vspace{0pt}\vspace{6pt}
\begin{mdframed}
[backgroundcolor=FTColorA,skipabove=3pt,skipbelow=3pt,innermargin=2cm,outermargin=2cm,innertopmargin=.03in,innerbottommargin=.03in,innerleftmargin=.125in,innerrightmargin=.125in,align=left]\vspace{0pt}\indent Si SayMore ne peut pas accéder au dossier où se trouvaient les fichiers la dernière fois que vous avez utilisé cette fonction, un avertissement \vspace*{0pt}{\XeTeXpicfile "../images/en/WARNING_ICON.png" scaled 750} Le message apparaît. Naviguez simplement dans le dossier souhaité à nouveau.\par{}\end{mdframed}
{\XLingPaperneedspace{3\baselineskip}
\noindent\rule{\textwidth}{.4pt}
{}\penalty10000\vspace{3pt}\XLingPaperneedspace{3\baselineskip}\noindent
\fontsize{10}{12}\selectfont \textbf{{\noindent
\raisebox{\baselineskip}[0pt]{\pdfbookmark[3]{{6.1.2 } Nouveau d'un enregistrement...}{sNewFromRecording}}\raisebox{\baselineskip}[0pt]{\protect\hypertarget{sNewFromRecording}{}}{6.1.2 }Nouveau d'un enregistrement...}}\markright{Nouveau d'un enregistrement...}
\XLingPaperaddtocontents{sNewFromRecording}}\par{}
\penalty10000\vspace{10pt}\penalty10000\vspace{0pt}\indent {\textbf{Nouveau à partir d'un enregistrement...}} Vous permet d'utiliser le microphone de l'ordinateur pour enregistrer une session. Si vous avez un microphone externe connecté, il enregistrera à partir de là. Cliquez sur le bouton Enregistrer et une fois terminé, cliquez sur le bouton Arrêter. Vous pouvez immédiatement entendre la lecture en appuyant sur Lire l'enregistrement. Si vous êtes satisfait de l'enregistrement, cliquez sur OK.\par{}\vspace{6pt}\vspace{10pt plus 2pt minus 1pt}\setbox0=\vbox{\protect\centering \leavevmode
\vspace*{0pt}{\XeTeXpicfile "../images/fr/Nouveau_d_un_enregistrement_fr.png" scaled 750}\\[0pt]\protect\hypertarget{fEnregistrer}{}\XLingPaperaddtocontents{fEnregistrer}\textit{{Figure }}\textit{{6.3}}\textit{{ Enregistreur Nouveau\\}}}\box0\par{}\vspace{10pt plus 2pt minus 1pt}
\begin{mdframed}
[backgroundcolor=FTColorA,skipabove=3pt,skipbelow=3pt,innermargin=2cm,outermargin=2cm,innertopmargin=.03in,innerbottommargin=.03in,innerleftmargin=.125in,innerrightmargin=.125in,align=left]\vspace{0pt}\indent {\textit{\textbf{Nous plaidons avec votre consentement pour éviter l'usage des microphones intégré de votre ordinateur portable. Si vous êtes prêt l'utilité pour des recherches phonétiques dans le future, un bon casque USB peut être utilisé à un coût raisonnable. Il est également à que le Zoom H2 peut également être branché et utilisé comme microphone.}}}\par{}\end{mdframed}
{\XLingPaperneedspace{3\baselineskip}
\noindent\rule{\textwidth}{1pt}
{}\penalty10000\vspace{3pt}\XLingPaperneedspace{3\baselineskip}\noindent
\fontsize{12}{14.399999999999999}\selectfont \textbf{{\noindent
\raisebox{\baselineskip}[0pt]{\pdfbookmark[2]{{6.2 } Supprimer une session}{sDeleteSession}}\raisebox{\baselineskip}[0pt]{\protect\hypertarget{sDeleteSession}{}}{6.2 }Supprimer une session}}\markright{Supprimer une session}
\XLingPaperaddtocontents{sDeleteSession}}\par{}
\penalty10000\vspace{10pt}\penalty10000\vspace{0pt}\indent Supprimer une session.... Vous permet de supprimer une session que vous avez sélectionné dans\par{}{\XLingPaperneedspace{3\baselineskip}
\noindent\rule{\textwidth}{1pt}
{}\penalty10000\vspace{3pt}\XLingPaperneedspace{3\baselineskip}\noindent
\fontsize{12}{14.399999999999999}\selectfont \textbf{{\noindent
\raisebox{\baselineskip}[0pt]{\pdfbookmark[2]{{6.3 } Session Metadata}{sSessMeta1}}\raisebox{\baselineskip}[0pt]{\protect\hypertarget{sSessMeta1}{}}{6.3 }Session Metadata}}\markright{Session Metadata}
\XLingPaperaddtocontents{sSessMeta1}}\par{}
\penalty10000\vspace{10pt}\penalty10000\vspace{0pt}\indent Lorsque vous êtes dans l'onglet Sessions, et qu'une session est sélectionnée sur la gauche, sur la droite vous avez une fenêtre qui affiche tous les fichiers stockés dans cette session. Sous cette fenêtre, vous avez une autre fenêtre qui change en fonction du fichier que vous avez sélectionné. Lorsque vous avez sélectionné le fichier .Session, vous obtenez la fenêtre qui vous permet d'entrer les métadonnées pour cette session, avec Statut \& Stages, et Notes:\par{}\vspace{6pt}\vspace{10pt plus 2pt minus 1pt}\setbox0=\vbox{\protect\raggedright\leavevmode
\vspace*{0pt}{\XeTeXpicfile "../images/fr/Sessions_metaData_fr.png" scaled 750}\\[0pt]\protect\hypertarget{fmetasess}{}\XLingPaperaddtocontents{fmetasess}\textit{{Figure }}\textit{{6.4}}\textit{{ Session Metadata\\}}}\box0\par{}\vspace{10pt plus 2pt minus 1pt}\vspace{0pt}\indent SayMore ira de l'avant une copie dans les métadonnées du projet de localisation vers les champs correspondants ci-dessous Plus de champs. Si vous modifiez les métadonnées d'emplacement dans l'onglet Projet, il n'est pas mis à jour dans les sessions existantes. Vous pouvez également cliquer sur l'onglet Notes, puis entrer des notes sur la session en prose. Les Notes n'ont pas de caractéristiques de formatage.\par{}\vspace{6pt}\vspace{0pt}\indent Lorsque vous cliquez sur l'onglet Statut \& Stages, vous verrez:\par{}\vspace{6pt}\vspace{10pt plus 2pt minus 1pt}\setbox0=\vbox{\protect\raggedright\leavevmode
\vspace*{0pt}{\XeTeXpicfile "../images/fr/Sessions_Etat&Etapes_fr.png" scaled 750}\\[0pt]\protect\hypertarget{fetatetetapes}{}\XLingPaperaddtocontents{fetatetetapes}\textit{{Figure }}\textit{{6.5}}\textit{{ Sessions Etat \& Etapes Onglet\\}}}\box0\par{}\vspace{10pt plus 2pt minus 1pt}\vspace{0pt}\indent Le statut est l'endroit où vous pouvez faire une grande déclaration de l'endroit où l'enregistrement a été traité. Les étapes est l'endroit où vous définissez ce qui a été fait avec l'enregistrement de la session.\par{}\clearpage
\thispagestyle{bodyfirstpage}\markboth{Gestion des personnes}{Gestion des personnes}
\XLingPaperaddtocontents{cPeople}{\XLingPaperneedspace{3\baselineskip}\noindent
\fontsize{18}{21.599999999999998}\selectfont \textbf{{\centering
\raisebox{\baselineskip}[0pt]{\protect\hypertarget{cPeople}{}}\raisebox{\baselineskip}[0pt]{\pdfbookmark[1]{7 Gestion des personnes}{cPeople}}7\\}}}\par{}
\vspace{10.8pt}{\XLingPaperneedspace{3\baselineskip}\noindent
\fontsize{18}{21.599999999999998}\selectfont \textbf{{\centering
Gestion des personnes\\}}}\par{}
\vspace{21.6pt}{\XLingPaperneedspace{3\baselineskip}
\noindent\rule{\textwidth}{1pt}
{}\penalty10000\vspace{3pt}\XLingPaperneedspace{3\baselineskip}\noindent
\fontsize{12}{14.399999999999999}\selectfont \textbf{{\noindent
\raisebox{\baselineskip}[0pt]{\pdfbookmark[2]{{7.1 } Ajouter une Nouvelle Personne}{sNewPerson}}\raisebox{\baselineskip}[0pt]{\protect\hypertarget{sNewPerson}{}}{7.1 }Ajouter une Nouvelle Personne}}\markright{Ajouter une Nouvelle Personne}
\XLingPaperaddtocontents{sNewPerson}}\par{}
\penalty10000\vspace{10pt}\penalty10000\vspace{0pt}\indent Quand vous cliquez sur {\textbf{Nouveau}} dans menu {\textbf{Personne}}, vous allez voir:\par{}\vspace{6pt}\vspace{10pt plus 2pt minus 1pt}\setbox0=\vbox{\protect\raggedright\leavevmode
\vspace*{0pt}{\XeTeXpicfile "../images/en/EmptyNewPerson_en.png" scaled 750}\\[0pt]\protect\hypertarget{fnewperson}{}\XLingPaperaddtocontents{fnewperson}\textit{{Figure }}\textit{{7.1}}\textit{{ Empty New Person\\}}}\box0\par{}\vspace{10pt plus 2pt minus 1pt}\vspace{0pt}\indent Il est important de remplir autant que possible à propos de la personne\par{}{\XLingPaperneedspace{3\baselineskip}
\noindent\rule{\textwidth}{1pt}
{}\penalty10000\vspace{3pt}\XLingPaperneedspace{3\baselineskip}\noindent
\fontsize{12}{14.399999999999999}\selectfont \textbf{{\noindent
\raisebox{\baselineskip}[0pt]{\pdfbookmark[2]{{7.2 } Consentement éclairé}{sInformedConsent}}\raisebox{\baselineskip}[0pt]{\protect\hypertarget{sInformedConsent}{}}{7.2 }Consentement éclairé}}\markright{Consentement éclairé}
\XLingPaperaddtocontents{sInformedConsent}}\par{}
\penalty10000\vspace{10pt}\penalty10000\vspace{0pt}\indent Notez bien qu'il faut marquer tout le monde qui a travailler sur l'enregistrement et l'analyse, pas simplement ceux qui figurent dans l'enregistrement. Si vous ajoutez quelqu'un de qui vous n'avez pas ajouté leur consentement, c'est bien possible que ça peut empêcher le travail, la validité légale, et la publication de votre corpus. Même un des participants qui refuse de donner son accord sur l'utilisation de ces produits intellectuelle rende le produit inutilisable au niveau légale du copyright, et il aura le droit de vous portez plainte sur toute utilisation non autorisé. Aussi, les résultats de ce recherche seront scientifiquement invalide.\par{}\vspace{6pt}
\begin{mdframed}
[backgroundcolor=FTColorA,skipabove=3pt,skipbelow=3pt,innermargin=2cm,outermargin=2cm,innertopmargin=.03in,innerbottommargin=.03in,innerleftmargin=.125in,innerrightmargin=.125in,align=left]\vspace{0pt}\indent \vspace*{0pt}{\XeTeXpicfile "../images/en/WARNING_ICON.png" scaled 750} On dit {\textit{\textbf{Consentement éclairé}}} ici parce que c'est la responsabilité du chercheur à informer les participants sur leur droits du propriété intellectuels. Si une personne n'a pas compris qu'elle à la possibilité de refuser toute utilisation de son propriété intellectuel, vous ne pouvez pas, en bonne foi, utiliser les produits pour les buts académiques ni personnelles.\par{}\end{mdframed}
{\XLingPaperneedspace{3\baselineskip}
\noindent\rule{\textwidth}{1pt}
{}\penalty10000\vspace{3pt}\XLingPaperneedspace{3\baselineskip}\noindent
\fontsize{12}{14.399999999999999}\selectfont \textbf{{\noindent
\raisebox{\baselineskip}[0pt]{\pdfbookmark[2]{{7.3 } Supprimer une personne}{sDeletePerson}}\raisebox{\baselineskip}[0pt]{\protect\hypertarget{sDeletePerson}{}}{7.3 }Supprimer une personne}}\markright{Supprimer une personne}
\XLingPaperaddtocontents{sDeletePerson}}\par{}
\penalty10000\vspace{10pt}\penalty10000\vspace{0pt}\indent Vous permet de supprimer une personne d'une session. Vous pouvez simplement faire un clique droit sur la personne et sélectionner {\textbf{Supprimer une personne}}. Une boîte d'avertissement apparaîtra avant que la suppression ne soit définitive.\par{}\clearpage
{\clearpage
\XLingPaperneedspace{3\baselineskip}\noindent
\fontsize{18}{21.599999999999998}\selectfont \textbf{{\centering
\thispagestyle{empty}\raisebox{\baselineskip}[0pt]{\pdfbookmark[1]{Part III Transcription des fichiers audios et vidéos}{pTranscribe}}\raisebox{\baselineskip}[0pt]{\protect\hypertarget{pTranscribe}{}}Part III\\}}}\par{}
\vspace{10.8pt}{\XLingPaperneedspace{3\baselineskip}\noindent
\fontsize{18}{21.599999999999998}\selectfont \textbf{{\centering
Transcription des fichiers audios et vidéos\\}}}\par{}
\vspace{21.6pt}\vspace{0pt}\vspace{6pt}\clearpage
\thispagestyle{bodyfirstpage}\markboth{Survol du processus de la transcription}{Survol du processus de la transcription}
\XLingPaperaddtocontents{cTranscribe}{\XLingPaperneedspace{3\baselineskip}\noindent
\fontsize{18}{21.599999999999998}\selectfont \textbf{{\centering
\raisebox{\baselineskip}[0pt]{\protect\hypertarget{cTranscribe}{}}\raisebox{\baselineskip}[0pt]{\pdfbookmark[1]{8 Survol du processus de la transcription}{cTranscribe}}8\\}}}\par{}
\vspace{10.8pt}{\XLingPaperneedspace{3\baselineskip}\noindent
\fontsize{18}{21.599999999999998}\selectfont \textbf{{\centering
Survol du processus de la transcription\\}}}\par{}
\vspace{21.6pt}\vspace{0pt}\indent Comme des chercheurs sur le champ, vous êtes souvent le plus intéressé par vos questions de recherche. Dans des langues peu ressourcés, c'est bien possible que le corpus que vous aurez récolté sera une ressource très important pour la communauté, et en plus pour les autres chercheurs plus tard.\par{}\vspace{6pt}\vspace{0pt}\indent Donc, la recherche remplit plusieurs buts :\par{}{\parskip .5pt plus 1pt minus 1pt

\vspace{\baselineskip}

{\setlength{\XLingPapertempdim}{\XLingPaperbulletlistitemwidth+6em}\leftskip\XLingPapertempdim\relax
\interlinepenalty10000
\XLingPaperlistitem{6em}{\XLingPaperbulletlistitemwidth}{•}{Consentement éclairé}\vspace{3pt}}
{\setlength{\XLingPapertempdim}{\XLingPaperbulletlistitemwidth+6em}\leftskip\XLingPapertempdim\relax
\interlinepenalty10000
\XLingPaperlistitem{6em}{\XLingPaperbulletlistitemwidth}{•}{La récolte des données primaire sur le champ.}\vspace{3pt}}
{\setlength{\XLingPapertempdim}{\XLingPaperbulletlistitemwidth+6em}\leftskip\XLingPapertempdim\relax
\interlinepenalty10000
\XLingPaperlistitem{6em}{\XLingPaperbulletlistitemwidth}{•}{Le rassemblement des ressources trouvé sur la langue cible et les langues aux alentours.}\vspace{3pt}}
{\setlength{\XLingPapertempdim}{\XLingPaperbulletlistitemwidth+6em}\leftskip\XLingPapertempdim\relax
\interlinepenalty10000
\XLingPaperlistitem{6em}{\XLingPaperbulletlistitemwidth}{•}{Vos hypothèses et vos résultats.}}
\vspace{\baselineskip}
}\vspace{0pt}\indent Pour investigué ces questions, vous allez probablement récolté des données par enregistrement et par écrit. Si vous publiez un corpus (un ensemble) numérique de vos données primaires en addition de vos résultats, ils seront scientifiquement vérifiable par d'autres chercheurs (et vos superviseurs), et on peut aussi réutiliser vos données pour faire d'autres analyses et renforcé l'étude de ces langues.\par{}\vspace{6pt}\vspace{0pt}\indent SayMore vous aidera a organiser, analyser, archiver, et publier les données basée sur des enregistrements.\par{}\vspace{6pt}\vspace{0pt}\indent Il y a six grandes étapes a suivre dans SayMore par enregistrement pour que ces données soient utilisable pas simplement pour vous-même, mais pour les chercheurs plus tard.\par{}{\parskip .5pt plus 1pt minus 1pt
                    
\vspace{\baselineskip}

{\setlength{\XLingPapertempdim}{\XLingPapersingledigitlistitemwidth+6em}\leftskip\XLingPapertempdim\relax
\interlinepenalty10000
\XLingPaperlistitem{6em}{\XLingPapersingledigitlistitemwidth}{1.}{Enregistrement Source (Source Recording)}\vspace{3pt}}
{\setlength{\XLingPapertempdim}{\XLingPapersingledigitlistitemwidth+6em}\leftskip\XLingPapertempdim\relax
\interlinepenalty10000
\XLingPaperlistitem{6em}{\XLingPapersingledigitlistitemwidth}{2.}{Consentement Informé (Informed Consent)}\vspace{3pt}}
{\setlength{\XLingPapertempdim}{\XLingPapersingledigitlistitemwidth+6em}\leftskip\XLingPapertempdim\relax
\interlinepenalty10000
\XLingPaperlistitem{6em}{\XLingPapersingledigitlistitemwidth}{3.}{Transcription prudent orale (Careful Speech)}\vspace{3pt}}
{\setlength{\XLingPapertempdim}{\XLingPapersingledigitlistitemwidth+6em}\leftskip\XLingPapertempdim\relax
\interlinepenalty10000
\XLingPaperlistitem{6em}{\XLingPapersingledigitlistitemwidth}{4.}{Traduction orale (Oral Translation)}\vspace{3pt}}
{\setlength{\XLingPapertempdim}{\XLingPapersingledigitlistitemwidth+6em}\leftskip\XLingPapertempdim\relax
\interlinepenalty10000
\XLingPaperlistitem{6em}{\XLingPapersingledigitlistitemwidth}{5.}{Transcription écrite (Written Transcription)}\vspace{3pt}}
{\setlength{\XLingPapertempdim}{\XLingPapersingledigitlistitemwidth+6em}\leftskip\XLingPapertempdim\relax
\interlinepenalty10000
\XLingPaperlistitem{6em}{\XLingPapersingledigitlistitemwidth}{6.}{Traduction écrite (Written Translation)}}
\vspace{\baselineskip}
}\vspace{0pt}\indent Ces processus ne sont pas seulement en ordre de complétion, mais chaque étape bien fait rendra les étapes suivant plus facile. Ainsi, il faut toujours planifier plus de temps pour les étapes successives. La transcription écrite est pénible, mais très utile. Si vous êtes autochtone de la langue que vous étudiez, vous pouvez vite faire étapes 1-4, transcrire et traduire (par écrit) simplement les enregistrements le plus utile pour le mémoire, et laisser le reste du travail aux autres plus tard qui reçoit votre corpus.\par{}\vspace{6pt}\vspace{0pt}\indent Si vous avez déjà crée une session avec l'audio (ou vidéo), regarder l'onglet {\textbf{État et Étapes}}. La case á côte de {\textbf{Source Recording}} sera coché.\\\vspace*{0pt}{\XeTeXpicfile "../images/fr/Source Recording-fr.png" scaled 750}\par{}\vspace{6pt}\vspace{0pt}\indent Vous êtes prêt à commencer la saisie des métadonnées et la transcription. Alors, SayMore vous guidera pour ces processus.\par{}\vspace{6pt}{\XLingPaperneedspace{3\baselineskip}
\noindent\rule{\textwidth}{1pt}
{}\penalty10000\vspace{3pt}\XLingPaperneedspace{3\baselineskip}\noindent
\fontsize{12}{14.399999999999999}\selectfont \textbf{{\noindent
\raisebox{\baselineskip}[0pt]{\pdfbookmark[2]{{8.1 } Metadonnées du session.}{sMetaMaybe}}\raisebox{\baselineskip}[0pt]{\protect\hypertarget{sMetaMaybe}{}}{8.1 }Metadonnées du session.}}\markright{Metadonnées du session.}
\XLingPaperaddtocontents{sMetaMaybe}}\par{}
\penalty10000\vspace{10pt}\penalty10000\vspace{0pt}\indent Voir Section \hyperlink{sSessMeta1}{6.3}\par{}{\XLingPaperneedspace{3\baselineskip}
\noindent\rule{\textwidth}{1pt}
{}\penalty10000\vspace{3pt}\XLingPaperneedspace{3\baselineskip}\noindent
\fontsize{12}{14.399999999999999}\selectfont \textbf{{\noindent
\raisebox{\baselineskip}[0pt]{\pdfbookmark[2]{{8.2 } Ajouter les contributeurs de session}{sAddContributors}}\raisebox{\baselineskip}[0pt]{\protect\hypertarget{sAddContributors}{}}{8.2 }Ajouter les contributeurs de session}}\markright{Ajouter les contributeurs de session}
\XLingPaperaddtocontents{sAddContributors}}\par{}
\penalty10000\vspace{10pt}\penalty10000\vspace{0pt}\indent Avant de commencer ici, vous devriez déjà avoir ajouter toutes les personnes qui ont travaillé pour cette enregistrement, c'est-à-dire le talent (celui ou ceux qui figure dans l'enregistrement), le technicien, et les traducteurs.\par{}\vspace{6pt}\vspace{10pt plus 2pt minus 1pt}\setbox0=\vbox{\protect\raggedright\leavevmode
\vspace*{0pt}{\XeTeXpicfile "../images/fr/Sessions_metaData_Personnes_fr.png" scaled 750}\\[0pt]\protect\hypertarget{fpersonnes}{}\XLingPaperaddtocontents{fpersonnes}\textit{{Figure }}\textit{{8.1}}\textit{{ Personnes\\}}}\box0\par{}\vspace{10pt plus 2pt minus 1pt}\vspace{0pt}\indent Si on clique sur l'objet du type {\textbf{Session}}, il y a un champ pour {\textbf{Personnes}}. Cliquez sur la liste déroutante et cochez le case à côté de chaque participant. Il se peut que des personnes assisteront plus tard, et vous pouvez les ajouter ici au fuir et à mesure.\par{}\vspace{6pt}
\begin{mdframed}
[backgroundcolor=FTColorA,skipabove=3pt,skipbelow=3pt,innermargin=2cm,outermargin=2cm,innertopmargin=.03in,innerbottommargin=.03in,innerleftmargin=.125in,innerrightmargin=.125in,align=left]\vspace{0pt}\indent Notez bien qu'il faut marquer tout le monde qui a travailler sur l'enregistrement et l'analyse, pas simplement ceux qui figurent dans l'enregistrement. Si vous ajoutez quelqu'un de qui vous n'avez pas ajouté leur consentement, c'est bien possible que ça peut empêcher le travail, la validité légale, et la publication de votre corpus. Même un des participants qui refuse de donner son accord sur l'utilisation de ces produits intellectuelle rende le produit inutilisable au niveau légale du copyright, et il aura le droit de vous portez plainte sur toute utilisation non autorisé. Aussi, les résultats de ce recherche seront scientifiquement invalide. Voir section \hyperlink{sInformedConsent}{7.2}.\par{}\end{mdframed}
\vspace{0pt}\indent Une fois que vous avez ajouté des personnes avec la preuve de leur accord (soit un document signé et numérisé ou un enregistrement),\par{}\vspace{6pt}\vspace{10pt plus 2pt minus 1pt}\setbox0=\vbox{\protect\raggedright\leavevmode
\vspace*{0pt}{\XeTeXpicfile "../images/fr/Renommer_Consentement_eclare_fr.png" scaled 750}\\[0pt]\protect\hypertarget{f-NeedsALabel-.xlingpaper.1..styledPaper.1..lingPaper.1..part.3..chapter.1..section1.2..figure.4.}{}\XLingPaperaddtocontents{f-NeedsALabel-.xlingpaper.1..styledPaper.1..lingPaper.1..part.3..chapter.1..section1.2..figure.4.}\textit{{Figure }}\textit{{8.2}}\textit{{ Renommer Consentement éclairé\\}}}\box0\par{}\vspace{10pt plus 2pt minus 1pt}\vspace{0pt}\indent Regarder l'onglet {\textbf{État et Étapes}}. Le case á côte de {\textbf{Informed Consent}} sera déjà coché.\\\vspace*{0pt}{\XeTeXpicfile "../images/fr/informed consent-fr.png" scaled 750}\par{}{\XLingPaperneedspace{3\baselineskip}
\noindent\rule{\textwidth}{1pt}
{}\penalty10000\vspace{3pt}\XLingPaperneedspace{3\baselineskip}\noindent
\fontsize{12}{14.399999999999999}\selectfont \textbf{{\noindent
\raisebox{\baselineskip}[0pt]{\pdfbookmark[2]{{8.3 } Faire l'annotation}{sStartAnnot}}\raisebox{\baselineskip}[0pt]{\protect\hypertarget{sStartAnnot}{}}{8.3 }Faire l'annotation}}\markright{Faire l'annotation}
\XLingPaperaddtocontents{sStartAnnot}}\par{}
\penalty10000\vspace{10pt}\penalty10000\vspace{0pt}\indent Si votre enregistrement est en format vidéo, allez vers section \hyperlink{sVidtoAudio}{8.3.1}. Si c'est déjà en format Audio, sautez vers section \hyperlink{sStartAnnot}{8.3}\par{}\vspace{6pt}{\XLingPaperneedspace{3\baselineskip}
\noindent\rule{\textwidth}{.4pt}
{}\penalty10000\vspace{3pt}\XLingPaperneedspace{3\baselineskip}\noindent
\fontsize{10}{12}\selectfont \textbf{{\noindent
\raisebox{\baselineskip}[0pt]{\pdfbookmark[3]{{8.3.1 } Créer une version audio du vidéo.}{sVidtoAudio}}\raisebox{\baselineskip}[0pt]{\protect\hypertarget{sVidtoAudio}{}}{8.3.1 }Créer une version audio du vidéo.}}\markright{Créer une version audio du vidéo.}
\XLingPaperaddtocontents{sVidtoAudio}}\par{}
\penalty10000\vspace{10pt}\penalty10000\vspace{0pt}\indent Si vous avez importé une vidéo, il faut que SayMore copie l'audio du vidéo vers un autre ficher.\par{}{\parskip .5pt plus 1pt minus 1pt
                    
\vspace{\baselineskip}

{\setlength{\XLingPapertempdim}{\XLingPapersingledigitlistitemwidth+6em}\leftskip\XLingPapertempdim\relax
\interlinepenalty10000
\XLingPaperlistitem{6em}{\XLingPapersingledigitlistitemwidth}{1.}{Cliquez une fois sur le vidéo et encore sur l'onglet {\textbf{Commencer à annoter}}.\\{\textit{Le texte explique que SayMore créera un ficher {\XLingPaperCourierZNewFontFamily{.wav}} pour l'analyse.}}\\\vspace{10pt plus 2pt minus 1pt}\setbox0=\vbox{\protect\raggedright\leavevmode
\vspace*{0pt}{\XeTeXpicfile "../images/fr/commencer a annoter.png" scaled 750}\\[0pt]\protect\hypertarget{f-NeedsALabel-.xlingpaper.1..styledPaper.1..lingPaper.1..part.3..chapter.1..section1.3..section2.1..ol.1..li.1..figure.1.}{}\XLingPaperaddtocontents{f-NeedsALabel-.xlingpaper.1..styledPaper.1..lingPaper.1..part.3..chapter.1..section1.3..section2.1..ol.1..li.1..figure.1.}\textit{{Figure }}\textit{{8.3}}\textit{{ Commencer à annoter\\}}}\box0\par{}\vspace{10pt plus 2pt minus 1pt}}\vspace{3pt}}
{\setlength{\XLingPapertempdim}{\XLingPapersingledigitlistitemwidth+6em}\leftskip\XLingPapertempdim\relax
\interlinepenalty10000
\XLingPaperlistitem{6em}{\XLingPapersingledigitlistitemwidth}{2.}{Cliquez sur {\textbf{OK}}.\\{\textit{Après la conversion, le fichier {\XLingPaperCourierZNewFontFamily{.wav}} apparaîtra.}}}}
\vspace{\baselineskip}
}{\XLingPaperneedspace{3\baselineskip}
\noindent\rule{\textwidth}{.4pt}
{}\penalty10000\vspace{3pt}\XLingPaperneedspace{3\baselineskip}\noindent
\fontsize{10}{12}\selectfont \textbf{{\noindent
\raisebox{\baselineskip}[0pt]{\pdfbookmark[3]{{8.3.2 } Segmentation Automatique}{sAutoSeg}}\raisebox{\baselineskip}[0pt]{\protect\hypertarget{sAutoSeg}{}}{8.3.2 }Segmentation Automatique}}\markright{Segmentation Automatique}
\XLingPaperaddtocontents{sAutoSeg}}\par{}
\penalty10000\vspace{10pt}\penalty10000\vspace{0pt}\indent SayMore peut essayer a découper l'enregistrement à chaque silence pour faire des segments gerables.\par{}{\parskip .5pt plus 1pt minus 1pt
                    
\vspace{\baselineskip}

{\setlength{\XLingPapertempdim}{\XLingPapersingledigitlistitemwidth+6em}\leftskip\XLingPapertempdim\relax
\interlinepenalty10000
\XLingPaperlistitem{6em}{\XLingPapersingledigitlistitemwidth}{1.}{Cliquez sur le fichier audio \vspace*{0pt}{\XeTeXpicfile "../images/fr/micro.png" scaled 750} dans la liste des fichiers, et puis sur l'onglet {\textbf{Commencer à annoter}}.\\\vspace{10pt plus 2pt minus 1pt}\setbox0=\vbox{\protect\raggedright\leavevmode
\vspace*{0pt}{\XeTeXpicfile "../images/fr/segment automatique.png" scaled 750}\\[0pt]\protect\hypertarget{f}{}\XLingPaperaddtocontents{f}\textit{{Figure }}\textit{{8.4}}\textit{{ Utiliser le segmenteur automatique\\}}}\box0\par{}\vspace{10pt plus 2pt minus 1pt}}\vspace{3pt}}
{\setlength{\XLingPapertempdim}{\XLingPapersingledigitlistitemwidth+6em}\leftskip\XLingPapertempdim\relax
\interlinepenalty10000
\XLingPaperlistitem{6em}{\XLingPapersingledigitlistitemwidth}{2.}{Selectionnez {\textbf{Utilisez le segmenteur automatique}} et cliquez sur le bouton {\textbf{Commencer... }}.\\{\textit{SayMore découpera l'enregistrement à chaque silence et affichera un certain nombre des champs vide.}}\\\vspace{10pt plus 2pt minus 1pt}\setbox0=\vbox{\protect\raggedright\leavevmode
\vspace*{0pt}{\XeTeXpicfile "../images/fr/empty fields fr.png" scaled 750}\\[0pt]\protect\hypertarget{f}{}\XLingPaperaddtocontents{f}\textit{{Figure }}\textit{{8.5}}\textit{{ Annotations\\}}}\box0\par{}\vspace{10pt plus 2pt minus 1pt}}\vspace{3pt}}
{\setlength{\XLingPapertempdim}{\XLingPapersingledigitlistitemwidth+6em}\leftskip\XLingPapertempdim\relax
\interlinepenalty10000
\XLingPaperlistitem{6em}{\XLingPapersingledigitlistitemwidth}{3.}{Cliquez sur \vspace*{0pt}{\XeTeXpicfile "../images/fr/segment.png" scaled 750} pour voir (ou adjuster) le découpage. Il et fortement conseiller de faire les grandes changements de la segmentation AVANT de faire les transcriptions.}\vspace{3pt}}
{\setlength{\XLingPapertempdim}{\XLingPapersingledigitlistitemwidth+6em}\leftskip\XLingPapertempdim\relax
\interlinepenalty10000
\XLingPaperlistitem{6em}{\XLingPapersingledigitlistitemwidth}{4.}{S'il faut modifier la segmentation, voir section \hyperlink{}{}.}}
\vspace{\baselineskip}
}{\XLingPaperneedspace{3\baselineskip}
\noindent\rule{\textwidth}{.4pt}
{}\penalty10000\vspace{3pt}\XLingPaperneedspace{3\baselineskip}\noindent
\fontsize{10}{12}\selectfont \textbf{{\noindent
\raisebox{\baselineskip}[0pt]{\pdfbookmark[3]{{8.3.3 } Changer la Segmentation}{sAdjustSeg}}\raisebox{\baselineskip}[0pt]{\protect\hypertarget{sAdjustSeg}{}}{8.3.3 }Changer la Segmentation}}\markright{Changer la Segmentation}
\XLingPaperaddtocontents{sAdjustSeg}}\par{}
\penalty10000\vspace{10pt}\penalty10000\vspace{0pt}{\parskip .5pt plus 1pt minus 1pt
                    
\vspace{\baselineskip}

{\setlength{\XLingPapertempdim}{\XLingPapersingledigitlistitemwidth+6em}\leftskip\XLingPapertempdim\relax
\interlinepenalty10000
\XLingPaperlistitem{6em}{\XLingPapersingledigitlistitemwidth}{1.}{Cliquez sur \vspace*{0pt}{\XeTeXpicfile "../images/fr/segment.png" scaled 750} pour voir (ou ajuster) le découpage.  \\{\textit{La boîte de dialogue segmenter manuel apparaîtra.}}\\\vspace{10pt plus 2pt minus 1pt}\setbox0=\vbox{\protect\raggedright\leavevmode
\vspace*{0pt}{\XeTeXpicfile "../images/fr/resegment - fr.png" scaled 750}\\[0pt]\protect\hypertarget{f}{}\XLingPaperaddtocontents{f}\textit{{Figure }}\textit{{8.6}}\textit{{ Segmenteur Manuel\\}}}\box0\par{}\vspace{10pt plus 2pt minus 1pt}}\vspace{3pt}}
{\setlength{\XLingPapertempdim}{\XLingPapersingledigitlistitemwidth+6em}\leftskip\XLingPapertempdim\relax
\interlinepenalty10000
\XLingPaperlistitem{6em}{\XLingPapersingledigitlistitemwidth}{2.}{Vous pouvez tirer les barres oranges à gauche ou à droite avec le souris.}\vspace{3pt}}
{\setlength{\XLingPapertempdim}{\XLingPapersingledigitlistitemwidth+6em}\leftskip\XLingPapertempdim\relax
\interlinepenalty10000
\XLingPaperlistitem{6em}{\XLingPapersingledigitlistitemwidth}{3.}{Pour ajouter un segment cliquez là ou il faut couper l'audio et tapez {\textbf{Entrée}}.}\vspace{3pt}}
{\setlength{\XLingPapertempdim}{\XLingPapersingledigitlistitemwidth+6em}\leftskip\XLingPapertempdim\relax
\interlinepenalty10000
\XLingPaperlistitem{6em}{\XLingPapersingledigitlistitemwidth}{4.}{Pour ignorer les segments vides, cliquez dans le segment et cochez le case {\textbf{Ignoré}}.}}
\vspace{\baselineskip}
}{\XLingPaperneedspace{3\baselineskip}
\noindent\rule{\textwidth}{.4pt}
{}\penalty10000\vspace{3pt}\XLingPaperneedspace{3\baselineskip}\noindent
\fontsize{10}{12}\selectfont \textbf{{\noindent
\raisebox{\baselineskip}[0pt]{\pdfbookmark[3]{{8.3.4 } Segmentation Manuelle}{sManSeg}}\raisebox{\baselineskip}[0pt]{\protect\hypertarget{sManSeg}{}}{8.3.4 }Segmentation Manuelle}}\markright{Segmentation Manuelle}
\XLingPaperaddtocontents{sManSeg}}\par{}
\penalty10000\vspace{10pt}\penalty10000\vspace{0pt}{\parskip .5pt plus 1pt minus 1pt
                    
\vspace{\baselineskip}

{\setlength{\XLingPapertempdim}{\XLingPapersingledigitlistitemwidth+6em}\leftskip\XLingPapertempdim\relax
\interlinepenalty10000
\XLingPaperlistitem{6em}{\XLingPapersingledigitlistitemwidth}{1.}{Cliquez sur le fichier audio \vspace*{0pt}{\XeTeXpicfile "../images/fr/micro.png" scaled 750} dans la liste des fichiers, et puis sur l'onglet {\textbf{Commencer à annoter}}.\\ \\\vspace{10pt plus 2pt minus 1pt}\setbox0=\vbox{\protect\raggedright\leavevmode
\vspace*{0pt}{\XeTeXpicfile "../images/fr/segman.png" scaled 750}\\[0pt]\protect\hypertarget{f}{}\XLingPaperaddtocontents{f}\textit{{Figure }}\textit{{8.7}}\textit{{ Segmentation manuelle\\}}}\box0\par{}\vspace{10pt plus 2pt minus 1pt}}\vspace{3pt}}
{\setlength{\XLingPapertempdim}{\XLingPapersingledigitlistitemwidth+6em}\leftskip\XLingPapertempdim\relax
\interlinepenalty10000
\XLingPaperlistitem{6em}{\XLingPapersingledigitlistitemwidth}{2.}{Selectionnez {\textbf{Utilisez le segmenteur manuelle}} et cliquez sur le bouton {\textbf{Commencer... }}.\\{\textit{Le Segmenteur Manuelle affichera}}.\\\vspace{10pt plus 2pt minus 1pt}\setbox0=\vbox{\protect\raggedright\leavevmode
\vspace*{0pt}{\XeTeXpicfile "../images/fr/segmanwav.png" scaled 750}\\[0pt]\protect\hypertarget{f}{}\XLingPaperaddtocontents{f}\textit{{Figure }}\textit{{8.8}}\textit{{ Segmenteur\\}}}\box0\par{}\vspace{10pt plus 2pt minus 1pt}}\vspace{3pt}}
{\setlength{\XLingPapertempdim}{\XLingPapersingledigitlistitemwidth+6em}\leftskip\XLingPapertempdim\relax
\interlinepenalty10000
\XLingPaperlistitem{6em}{\XLingPapersingledigitlistitemwidth}{3.}{Tapez la barre d'{\textbf{espace}} pour jouer l'audio.}\vspace{3pt}}
{\setlength{\XLingPapertempdim}{\XLingPapersingledigitlistitemwidth+6em}\leftskip\XLingPapertempdim\relax
\interlinepenalty10000
\XLingPaperlistitem{6em}{\XLingPapersingledigitlistitemwidth}{4.}{Chaque fois qu'il faut couper l'enregistrement, tapez {\textbf{Éntrée}}.}\vspace{3pt}}
{\setlength{\XLingPapertempdim}{\XLingPapersingledigitlistitemwidth+6em}\leftskip\XLingPapertempdim\relax
\interlinepenalty10000
\XLingPaperlistitem{6em}{\XLingPapersingledigitlistitemwidth}{5.}{Continuez à tapez {\textbf{Éntrée}} pour chaque segment j'usqu'au la fin.}\vspace{3pt}}
{\setlength{\XLingPapertempdim}{\XLingPapersingledigitlistitemwidth+6em}\leftskip\XLingPapertempdim\relax
\interlinepenalty10000
\XLingPaperlistitem{6em}{\XLingPapersingledigitlistitemwidth}{6.}{Cliquez sur le bouton {\textbf{Bien}} (OK)..}}
\vspace{\baselineskip}
}{\XLingPaperneedspace{3\baselineskip}
\noindent\rule{\textwidth}{1pt}
{}\penalty10000\vspace{3pt}\XLingPaperneedspace{3\baselineskip}\noindent
\fontsize{12}{14.399999999999999}\selectfont \textbf{{\noindent
\raisebox{\baselineskip}[0pt]{\pdfbookmark[2]{{8.4 } Transcription en discours soigneux.}{sCarefulSpeech}}\raisebox{\baselineskip}[0pt]{\protect\hypertarget{sCarefulSpeech}{}}{8.4 }Transcription en discours soigneux.}}\markright{Transcription en discours soigneux.}
\XLingPaperaddtocontents{sCarefulSpeech}}\par{}
\penalty10000\vspace{10pt}\penalty10000\vspace{0pt}\indent Souvent, pendant des discours, on parle rapidement et on ne prends pas le soin de bien prononcer chaque mot. Aussi, l'enregistreur est parfois loin de l'interlocuteur. À cet effet, les enregistrements sur le champ ne sont pas idéale pour la transcription ni la traduction.\par{}\vspace{6pt}\vspace{0pt}\indent SayMore facilite la tâche de faire une transcription en discours soigneux. Un autre autochtone suivra l'enregistrement, et répète chaque phrase lentement et clairement au studio. Cet nouveau ficher sera vite fait, et il sera plus facile à transcrire.\par{}{\parskip .5pt plus 1pt minus 1pt
                    
\vspace{\baselineskip}

{\setlength{\XLingPapertempdim}{\XLingPaperdoubledigitlistitemwidth+6em}\leftskip\XLingPapertempdim\relax
\interlinepenalty10000
\XLingPaperlistitem{6em}{\XLingPaperdoubledigitlistitemwidth}{1.}{Cliquez sur le ficher {\textbf{Annotation}}.}\vspace{3pt}}
{\setlength{\XLingPapertempdim}{\XLingPaperdoubledigitlistitemwidth+6em}\leftskip\XLingPapertempdim\relax
\interlinepenalty10000
\XLingPaperlistitem{6em}{\XLingPaperdoubledigitlistitemwidth}{2.}{Dan l'onglet {\textbf{Annotations}}, cliquez sur Outils d'annotations orales, puis {\textbf{Discours soigneux...}}.\\\vspace{10pt plus 2pt minus 1pt}\setbox0=\vbox{\protect\raggedright\leavevmode
\vspace*{0pt}{\XeTeXpicfile "../images/fr/commencerdiscours.png" scaled 750}\\[0pt]\protect\hypertarget{f}{}\XLingPaperaddtocontents{f}\textit{{Figure }}\textit{{8.9}}\textit{{ Discours soigneux\\}}}\box0\par{}\vspace{10pt plus 2pt minus 1pt}}\vspace{3pt}}
{\setlength{\XLingPapertempdim}{\XLingPaperdoubledigitlistitemwidth+6em}\leftskip\XLingPapertempdim\relax
\interlinepenalty10000
\XLingPaperlistitem{6em}{\XLingPaperdoubledigitlistitemwidth}{3.}{La boîte de dialogue Enregistreur de discours soigneux s'affiche.\\\vspace{10pt plus 2pt minus 1pt}\setbox0=\vbox{\protect\raggedright\leavevmode
\vspace*{0pt}{\XeTeXpicfile "../images/fr/ecoutersource.png" scaled 750}\\[0pt]\protect\hypertarget{f}{}\XLingPaperaddtocontents{f}\textit{{Figure }}\textit{{8.10}}\textit{{ Enregistreur de discours soigneux\\}}}\box0\par{}\vspace{10pt plus 2pt minus 1pt}}\vspace{3pt}}
{\setlength{\XLingPapertempdim}{\XLingPaperdoubledigitlistitemwidth+6em}\leftskip\XLingPapertempdim\relax
\interlinepenalty10000
\XLingPaperlistitem{6em}{\XLingPaperdoubledigitlistitemwidth}{4.}{Cette boîte de dialogue se compose de plusieurs rangs de l'audio. Le premier rang et l'audio source, découpé par phrase. Le deuxième rang prendra le discours soigneux.}\vspace{3pt}}
{\setlength{\XLingPapertempdim}{\XLingPaperdoubledigitlistitemwidth+6em}\leftskip\XLingPapertempdim\relax
\interlinepenalty10000
\XLingPaperlistitem{6em}{\XLingPaperdoubledigitlistitemwidth}{5.}{Mettre le curseur dans la zone orange et cliquez sur le bouton \vspace*{0pt}{\XeTeXpicfile "../images/fr/play.png" scaled 750}.}\vspace{3pt}}
{\setlength{\XLingPapertempdim}{\XLingPaperdoubledigitlistitemwidth+6em}\leftskip\XLingPapertempdim\relax
\interlinepenalty10000
\XLingPaperlistitem{6em}{\XLingPaperdoubledigitlistitemwidth}{6.}{Rejouer l'audio plusieurs fois si c'est difficile à tout saisir.\\{\textit{Le bouton {\textbf{Prendre la parole}} s'active.}}}\vspace{3pt}}
{\setlength{\XLingPapertempdim}{\XLingPaperdoubledigitlistitemwidth+6em}\leftskip\XLingPapertempdim\relax
\interlinepenalty10000
\XLingPaperlistitem{6em}{\XLingPaperdoubledigitlistitemwidth}{7.}{Si cette section ne contienne pas des paroles à transcrire, vous pouvez l'ignorer. Mettez le curseur dans la région orange et cochez le case à côté de {\textbf{Ignorer}}.\\\vspace*{0pt}{\XeTeXpicfile "../images/fr/ignore.png" scaled 750}}\vspace{3pt}}
{\setlength{\XLingPapertempdim}{\XLingPaperdoubledigitlistitemwidth+6em}\leftskip\XLingPapertempdim\relax
\interlinepenalty10000
\XLingPaperlistitem{6em}{\XLingPaperdoubledigitlistitemwidth}{8.}{Appuyez sur le bouton {\textbf{Prendre la parole}} \vspace*{0pt}{\XeTeXpicfile "../images/fr/prendre la parole.png" scaled 750} (garder le bouton appuyé pendant que vous parlez, comme le talkie-walkie) et répétez soigneusement tous que vous venez d'entendre. \\SayMore avance au prochaine section.}\vspace{3pt}}
{\setlength{\XLingPapertempdim}{\XLingPaperdoubledigitlistitemwidth+6em}\leftskip\XLingPapertempdim\relax
\interlinepenalty10000
\XLingPaperlistitem{6em}{\XLingPaperdoubledigitlistitemwidth}{9.}{Vous pouvez rejouer ce que vous venez d'enregistrer.}{\setlength{\XLingPaperlistitemindent}{\XLingPaperdoubledigitlistitemwidth + 6em}
{\setlength{\XLingPapertempdim}{\XLingPapersingleletterlistitemwidth+\XLingPaperlistitemindent}\leftskip\XLingPapertempdim\relax
\interlinepenalty10000
\XLingPaperlistitem{\XLingPaperlistitemindent}{\XLingPapersingleletterlistitemwidth}{a.}{Si c'est bon, avancez au prochaine section.}\vspace{3pt}}
{\setlength{\XLingPapertempdim}{\XLingPapersingleletterlistitemwidth+\XLingPaperlistitemindent}\leftskip\XLingPapertempdim\relax
\interlinepenalty10000
\XLingPaperlistitem{\XLingPaperlistitemindent}{\XLingPapersingleletterlistitemwidth}{b.}{Si ce n'est pas bon, cliquez sur le bouton \vspace*{0pt}{\XeTeXpicfile "../images/fr/rerecord.png" scaled 750} et essayez-la encore.}}}\vspace{3pt}}
{\setlength{\XLingPapertempdim}{\XLingPaperdoubledigitlistitemwidth+6em}\leftskip\XLingPapertempdim\relax
\interlinepenalty10000
\XLingPaperlistitem{6em}{\XLingPaperdoubledigitlistitemwidth}{10.}{Continuez jusqu'à la fin du ficher, puis cliquez sur {\textbf{Bien}}.}}
\vspace{\baselineskip}
}\vspace{0pt}{\XLingPaperneedspace{3\baselineskip}
\noindent\rule{\textwidth}{1pt}
{}\penalty10000\vspace{3pt}\XLingPaperneedspace{3\baselineskip}\noindent
\fontsize{12}{14.399999999999999}\selectfont \textbf{{\noindent
\raisebox{\baselineskip}[0pt]{\pdfbookmark[2]{{8.5 } Transcription Orale}{sOralTransl}}\raisebox{\baselineskip}[0pt]{\protect\hypertarget{sOralTransl}{}}{8.5 }Transcription Orale}}\markright{Transcription Orale}
\XLingPaperaddtocontents{sOralTransl}}\par{}
\penalty10000\vspace{10pt}\penalty10000\vspace{0pt}\indent SayMore facilite la tâche de faire une traduction libre dans une langue véhiculaire. Un autochtone bilingue suivra l'enregistrement, et traduit chaque phrase lentement et clairement au studio. Cet nouveau ficher sera vite fait, et il sera plus facile à traduire en texte.\par{}{\parskip .5pt plus 1pt minus 1pt
                    
\vspace{\baselineskip}

{\setlength{\XLingPapertempdim}{\XLingPaperdoubledigitlistitemwidth+6em}\leftskip\XLingPapertempdim\relax
\interlinepenalty10000
\XLingPaperlistitem{6em}{\XLingPaperdoubledigitlistitemwidth}{1.}{Cliquez sur le ficher {\textbf{Annotation}}.}\vspace{3pt}}
{\setlength{\XLingPapertempdim}{\XLingPaperdoubledigitlistitemwidth+6em}\leftskip\XLingPapertempdim\relax
\interlinepenalty10000
\XLingPaperlistitem{6em}{\XLingPaperdoubledigitlistitemwidth}{2.}{Dan l'onglet {\textbf{Annotations}}, cliquez sur Outils d'annotations orales, puis {\textbf{Traduction orale...}}.\\\vspace{10pt plus 2pt minus 1pt}\setbox0=\vbox{\protect\raggedright\leavevmode
\vspace*{0pt}{\XeTeXpicfile "../images/fr/starttrans.png" scaled 750}\\[0pt]\protect\hypertarget{f}{}\XLingPaperaddtocontents{f}\textit{{Figure }}\textit{{8.11}}\textit{{ Traduction Orale\\}}}\box0\par{}\vspace{10pt plus 2pt minus 1pt}}\vspace{3pt}}
{\setlength{\XLingPapertempdim}{\XLingPaperdoubledigitlistitemwidth+6em}\leftskip\XLingPapertempdim\relax
\interlinepenalty10000
\XLingPaperlistitem{6em}{\XLingPaperdoubledigitlistitemwidth}{3.}{La boîte de dialogue Enregistreur de discours soigneux s'affiche.\\\vspace{10pt plus 2pt minus 1pt}\setbox0=\vbox{\protect\raggedright\leavevmode
\vspace*{0pt}{\XeTeXpicfile "../images/fr/oraltranswindow.png" scaled 750}\\[0pt]\protect\hypertarget{f}{}\XLingPaperaddtocontents{f}\textit{{Figure }}\textit{{8.12}}\textit{{ Enregistreur de traduction orale\\}}}\box0\par{}\vspace{10pt plus 2pt minus 1pt}}\vspace{3pt}}
{\setlength{\XLingPapertempdim}{\XLingPaperdoubledigitlistitemwidth+6em}\leftskip\XLingPapertempdim\relax
\interlinepenalty10000
\XLingPaperlistitem{6em}{\XLingPaperdoubledigitlistitemwidth}{4.}{Cette boîte de dialogue se compose de plusieurs rangs de l'audio. Le premier rang et l'audio source, découpé par phrase. Le deuxième rang prendra la traduction orale.}\vspace{3pt}}
{\setlength{\XLingPapertempdim}{\XLingPaperdoubledigitlistitemwidth+6em}\leftskip\XLingPapertempdim\relax
\interlinepenalty10000
\XLingPaperlistitem{6em}{\XLingPaperdoubledigitlistitemwidth}{5.}{Mettre le curseur dans la zone orange et cliquez sur le bouton \vspace*{0pt}{\XeTeXpicfile "../images/fr/play.png" scaled 750}.}\vspace{3pt}}
{\setlength{\XLingPapertempdim}{\XLingPaperdoubledigitlistitemwidth+6em}\leftskip\XLingPapertempdim\relax
\interlinepenalty10000
\XLingPaperlistitem{6em}{\XLingPaperdoubledigitlistitemwidth}{6.}{Rejouer l'audio plusieurs fois si c'est difficile à tout saisir.\\{\textit{Le bouton {\textbf{Prendre la parole}} s'active.}}}\vspace{3pt}}
{\setlength{\XLingPapertempdim}{\XLingPaperdoubledigitlistitemwidth+6em}\leftskip\XLingPapertempdim\relax
\interlinepenalty10000
\XLingPaperlistitem{6em}{\XLingPaperdoubledigitlistitemwidth}{7.}{Si cette section ne contienne pas des paroles à traduire, vous pouvez l'ignorer. Mettez le curseur dans la région orange et cochez le case à côté de {\textbf{Ignorer}}.\\\vspace*{0pt}{\XeTeXpicfile "../images/fr/ignore.png" scaled 750}}\vspace{3pt}}
{\setlength{\XLingPapertempdim}{\XLingPaperdoubledigitlistitemwidth+6em}\leftskip\XLingPapertempdim\relax
\interlinepenalty10000
\XLingPaperlistitem{6em}{\XLingPaperdoubledigitlistitemwidth}{8.}{Appuyez sur le bouton {\textbf{Prendre la parole}} \vspace*{0pt}{\XeTeXpicfile "../images/fr/prendre la parole.png" scaled 750} (garder le bouton appuyé pendant que vous parlez, comme le talkie-walkie) et traduire soigneusement tous que vous venez d'entendre. \\SayMore avance au prochaine section.}\vspace{3pt}}
{\setlength{\XLingPapertempdim}{\XLingPaperdoubledigitlistitemwidth+6em}\leftskip\XLingPapertempdim\relax
\interlinepenalty10000
\XLingPaperlistitem{6em}{\XLingPaperdoubledigitlistitemwidth}{9.}{Vous pouvez rejouer ce que vous venez d'enregistrer.}{\setlength{\XLingPaperlistitemindent}{\XLingPaperdoubledigitlistitemwidth + 6em}
{\setlength{\XLingPapertempdim}{\XLingPapersingleletterlistitemwidth+\XLingPaperlistitemindent}\leftskip\XLingPapertempdim\relax
\interlinepenalty10000
\XLingPaperlistitem{\XLingPaperlistitemindent}{\XLingPapersingleletterlistitemwidth}{a.}{Si c'est bon, avancez au prochaine section.}\vspace{3pt}}
{\setlength{\XLingPapertempdim}{\XLingPapersingleletterlistitemwidth+\XLingPaperlistitemindent}\leftskip\XLingPapertempdim\relax
\interlinepenalty10000
\XLingPaperlistitem{\XLingPaperlistitemindent}{\XLingPapersingleletterlistitemwidth}{b.}{Si ce n'est pas bon, cliquez sur le bouton \vspace*{0pt}{\XeTeXpicfile "../images/fr/rerecord.png" scaled 750} et essayez-la encore.}}}\vspace{3pt}}
{\setlength{\XLingPapertempdim}{\XLingPaperdoubledigitlistitemwidth+6em}\leftskip\XLingPapertempdim\relax
\interlinepenalty10000
\XLingPaperlistitem{6em}{\XLingPaperdoubledigitlistitemwidth}{10.}{Continuez jusqu'à la fin du ficher, puis cliquez sur {\textbf{Bien}}.}}
\vspace{\baselineskip}
}\vspace{0pt}{\XLingPaperneedspace{3\baselineskip}
\noindent\rule{\textwidth}{1pt}
{}\penalty10000\vspace{3pt}\XLingPaperneedspace{3\baselineskip}\noindent
\fontsize{12}{14.399999999999999}\selectfont \textbf{{\noindent
\raisebox{\baselineskip}[0pt]{\pdfbookmark[2]{{8.6 } Rédiger la Transcription}{sWrittenTrans}}\raisebox{\baselineskip}[0pt]{\protect\hypertarget{sWrittenTrans}{}}{8.6 }Rédiger la Transcription}}\markright{Rédiger la Transcription}
\XLingPaperaddtocontents{sWrittenTrans}}\par{}
\penalty10000\vspace{10pt}\penalty10000\vspace{0pt}\vspace{6pt}
\begin{mdframed}
[backgroundcolor=FTColorA,skipabove=3pt,skipbelow=3pt,innermargin=2cm,outermargin=2cm,innertopmargin=.03in,innerbottommargin=.03in,innerleftmargin=.125in,innerrightmargin=.125in,align=left]\vspace{0pt}\indent Remarque: Si vous avez déjà enregistré un discours soigneux, vous avez le choix des sons á écouter et transcrire. (Le discours soigneux sera probablement plus facile à comprendre. Cliquez sur Options à côté do Transcription, et choisir une option.\\\vspace*{0pt}{\XeTeXpicfile "../images/fr/transcription options.png" scaled 750}\par{}\end{mdframed}
{\parskip .5pt plus 1pt minus 1pt
                    
\vspace{\baselineskip}

{\setlength{\XLingPapertempdim}{\XLingPapersingledigitlistitemwidth+6em}\leftskip\XLingPapertempdim\relax
\interlinepenalty10000
\XLingPaperlistitem{6em}{\XLingPapersingledigitlistitemwidth}{1.}{Cliquez sur le fichier annotations dans la liste des fichiers pour le session\\L'onglet ci-dessous s'affiche.\\\vspace{10pt plus 2pt minus 1pt}\setbox0=\vbox{\protect\raggedright\leavevmode
\vspace*{0pt}{\XeTeXpicfile "../images/fr/empty fields fr.png" scaled 750}\\[0pt]\protect\hypertarget{f}{}\XLingPaperaddtocontents{f}\textit{{Figure }}\textit{{8.13}}\textit{{ Annotations\\}}}\box0\par{}\vspace{10pt plus 2pt minus 1pt}}\vspace{3pt}}
{\setlength{\XLingPapertempdim}{\XLingPapersingledigitlistitemwidth+6em}\leftskip\XLingPapertempdim\relax
\interlinepenalty10000
\XLingPaperlistitem{6em}{\XLingPapersingledigitlistitemwidth}{2.}{Cliquez dans le premier champ {\textbf{Transcription}}.\\{\textit{Le son jouera cinq fois.}}}\vspace{3pt}}
{\setlength{\XLingPapertempdim}{\XLingPapersingledigitlistitemwidth+6em}\leftskip\XLingPapertempdim\relax
\interlinepenalty10000
\XLingPaperlistitem{6em}{\XLingPapersingledigitlistitemwidth}{3.}{Si le son joue trop vite pour bien comprendre, choisir une autre pourcentage (\vspace*{0pt}{\XeTeXpicfile "../images/fr/percentage.png" scaled 750}) de la vitesse pour ralentir le son.}\vspace{3pt}}
{\setlength{\XLingPapertempdim}{\XLingPapersingledigitlistitemwidth+6em}\leftskip\XLingPapertempdim\relax
\interlinepenalty10000
\XLingPaperlistitem{6em}{\XLingPapersingledigitlistitemwidth}{4.}{Saisir une transcription (dans la même langue que l'enregistrement) pour ce segment. Selon vos besoins, vous pouvez le faire selon l'orthographe de la langue, ou l'Alphabet Phonétique Internationale.}\vspace{3pt}}
{\setlength{\XLingPapertempdim}{\XLingPapersingledigitlistitemwidth+6em}\leftskip\XLingPapertempdim\relax
\interlinepenalty10000
\XLingPaperlistitem{6em}{\XLingPapersingledigitlistitemwidth}{5.}{Tapez {\textbf{Entrée}}.}\vspace{3pt}}
{\setlength{\XLingPapertempdim}{\XLingPapersingledigitlistitemwidth+6em}\leftskip\XLingPapertempdim\relax
\interlinepenalty10000
\XLingPaperlistitem{6em}{\XLingPapersingledigitlistitemwidth}{6.}{Saisir le prochaine transcription.}\vspace{3pt}}
{\setlength{\XLingPapertempdim}{\XLingPapersingledigitlistitemwidth+6em}\leftskip\XLingPapertempdim\relax
\interlinepenalty10000
\XLingPaperlistitem{6em}{\XLingPapersingledigitlistitemwidth}{7.}{Répétez étapes 3 à 5.}}
\vspace{\baselineskip}
}{\XLingPaperneedspace{3\baselineskip}
\noindent\rule{\textwidth}{1pt}
{}\penalty10000\vspace{3pt}\XLingPaperneedspace{3\baselineskip}\noindent
\fontsize{12}{14.399999999999999}\selectfont \textbf{{\noindent
\raisebox{\baselineskip}[0pt]{\pdfbookmark[2]{{8.7 } Rédiger la traduction}{sWritTransl}}\raisebox{\baselineskip}[0pt]{\protect\hypertarget{sWritTransl}{}}{8.7 }Rédiger la traduction}}\markright{Rédiger la traduction}
\XLingPaperaddtocontents{sWritTransl}}\par{}
\penalty10000\vspace{10pt}\penalty10000\vspace{0pt}\vspace{6pt}
\begin{mdframed}
[backgroundcolor=FTColorA,skipabove=3pt,skipbelow=3pt,innermargin=2cm,outermargin=2cm,innertopmargin=.03in,innerbottommargin=.03in,innerleftmargin=.125in,innerrightmargin=.125in,align=left]\vspace{0pt}\indent Remarque: Si vous avez déjà enregistré une traduction libre, vous avez le choix des sons á écouter. (Cliquez sur {\textbf{Options}} à côté du {\textbf{Traduction Libre}}, et choisir une option.\\\vspace*{0pt}{\XeTeXpicfile "../images/fr/TradClick.png" scaled 750}\par{}\end{mdframed}
{\parskip .5pt plus 1pt minus 1pt
                    
\vspace{\baselineskip}

{\setlength{\XLingPapertempdim}{\XLingPapersingledigitlistitemwidth+6em}\leftskip\XLingPapertempdim\relax
\interlinepenalty10000
\XLingPaperlistitem{6em}{\XLingPapersingledigitlistitemwidth}{1.}{Cliquez sur le fichier annotations dans la liste des fichiers pour le session\\L'onglet ci-dessous s'affiche.\\\vspace{10pt plus 2pt minus 1pt}\setbox0=\vbox{\protect\raggedright\leavevmode
\vspace*{0pt}{\XeTeXpicfile "../images/fr/cliquez traduction.png" scaled 750}\\[0pt]\protect\hypertarget{f}{}\XLingPaperaddtocontents{f}\textit{{Figure }}\textit{{8.14}}\textit{{ list des fichiers\\}}}\box0\par{}\vspace{10pt plus 2pt minus 1pt}}\vspace{3pt}}
{\setlength{\XLingPapertempdim}{\XLingPapersingledigitlistitemwidth+6em}\leftskip\XLingPapertempdim\relax
\interlinepenalty10000
\XLingPaperlistitem{6em}{\XLingPapersingledigitlistitemwidth}{2.}{Cliquez dans le premier champ {\textbf{Traduction Libre}}.\\{\textit{Le son jouera cinq fois.}}}\vspace{3pt}}
{\setlength{\XLingPapertempdim}{\XLingPapersingledigitlistitemwidth+6em}\leftskip\XLingPapertempdim\relax
\interlinepenalty10000
\XLingPaperlistitem{6em}{\XLingPapersingledigitlistitemwidth}{3.}{Si le son joue trop vite pour bien comprendre le sens, choisir une autre pourcentage (\vspace*{0pt}{\XeTeXpicfile "../images/fr/percentage.png" scaled 750}) de la vitesse pour ralentir le son.}\vspace{3pt}}
{\setlength{\XLingPapertempdim}{\XLingPapersingledigitlistitemwidth+6em}\leftskip\XLingPapertempdim\relax
\interlinepenalty10000
\XLingPaperlistitem{6em}{\XLingPapersingledigitlistitemwidth}{4.}{Saisir une traduction libre (dans une langue véhiculaire comme anglais ou français) pour ce segment.}\vspace{3pt}}
{\setlength{\XLingPapertempdim}{\XLingPapersingledigitlistitemwidth+6em}\leftskip\XLingPapertempdim\relax
\interlinepenalty10000
\XLingPaperlistitem{6em}{\XLingPapersingledigitlistitemwidth}{5.}{Tapez {\textbf{Entrée}}.}\vspace{3pt}}
{\setlength{\XLingPapertempdim}{\XLingPapersingledigitlistitemwidth+6em}\leftskip\XLingPapertempdim\relax
\interlinepenalty10000
\XLingPaperlistitem{6em}{\XLingPapersingledigitlistitemwidth}{6.}{Saisir le prochaine transcription.}\vspace{3pt}}
{\setlength{\XLingPapertempdim}{\XLingPapersingledigitlistitemwidth+6em}\leftskip\XLingPapertempdim\relax
\interlinepenalty10000
\XLingPaperlistitem{6em}{\XLingPapersingledigitlistitemwidth}{7.}{Répétez étapes 3 à 5.}}
\vspace{\baselineskip}
}\clearpage
{\clearpage
\XLingPaperneedspace{3\baselineskip}\noindent
\fontsize{18}{21.599999999999998}\selectfont \textbf{{\centering
\thispagestyle{empty}\raisebox{\baselineskip}[0pt]{\pdfbookmark[1]{Part IV Archivage}{pArchiving}}\raisebox{\baselineskip}[0pt]{\protect\hypertarget{pArchiving}{}}Part IV\\}}}\par{}
\vspace{10.8pt}{\XLingPaperneedspace{3\baselineskip}\noindent
\fontsize{18}{21.599999999999998}\selectfont \textbf{{\centering
Archivage\\}}}\par{}
\vspace{21.6pt}\clearpage
\thispagestyle{bodyfirstpage}\markboth{Exportation des données}{Exportation des données}
\XLingPaperaddtocontents{cExport}{\XLingPaperneedspace{3\baselineskip}\noindent
\fontsize{18}{21.599999999999998}\selectfont \textbf{{\centering
\raisebox{\baselineskip}[0pt]{\protect\hypertarget{cExport}{}}\raisebox{\baselineskip}[0pt]{\pdfbookmark[1]{9 Exportation des données}{cExport}}9\\}}}\par{}
\vspace{10.8pt}{\XLingPaperneedspace{3\baselineskip}\noindent
\fontsize{18}{21.599999999999998}\selectfont \textbf{{\centering
Exportation des données\\}}}\par{}
\vspace{21.6pt}\vspace{0pt}\vspace{6pt}{\XLingPaperneedspace{3\baselineskip}
\noindent\rule{\textwidth}{1pt}
{}\penalty10000\vspace{3pt}\XLingPaperneedspace{3\baselineskip}\noindent
\fontsize{12}{14.399999999999999}\selectfont \textbf{{\noindent
\raisebox{\baselineskip}[0pt]{\pdfbookmark[2]{{9.1 } Exporter des sous-titres.}{sSubtitles}}\raisebox{\baselineskip}[0pt]{\protect\hypertarget{sSubtitles}{}}{9.1 }Exporter des sous-titres.}}\markright{Exporter des sous-titres.}
\XLingPaperaddtocontents{sSubtitles}}\par{}
\penalty10000\vspace{10pt}\penalty10000\vspace{0pt}{\XLingPaperneedspace{3\baselineskip}
\noindent\rule{\textwidth}{1pt}
{}\penalty10000\vspace{3pt}\XLingPaperneedspace{3\baselineskip}\noindent
\fontsize{12}{14.399999999999999}\selectfont \textbf{{\noindent
\raisebox{\baselineskip}[0pt]{\pdfbookmark[2]{{9.2 } Exporter des étiquettes Audacity.}{sAudacityLabels}}\raisebox{\baselineskip}[0pt]{\protect\hypertarget{sAudacityLabels}{}}{9.2 }Exporter des étiquettes Audacity.}}\markright{Exporter des étiquettes Audacity.}
\XLingPaperaddtocontents{sAudacityLabels}}\par{}
\penalty10000\vspace{10pt}\penalty10000\vspace{0pt}{\XLingPaperneedspace{3\baselineskip}
\noindent\rule{\textwidth}{1pt}
{}\penalty10000\vspace{3pt}\XLingPaperneedspace{3\baselineskip}\noindent
\fontsize{12}{14.399999999999999}\selectfont \textbf{{\noindent
\raisebox{\baselineskip}[0pt]{\pdfbookmark[2]{{9.3 } Exporter des sessions}{sExportSession}}\raisebox{\baselineskip}[0pt]{\protect\hypertarget{sExportSession}{}}{9.3 }Exporter des sessions}}\markright{Exporter des sessions}
\XLingPaperaddtocontents{sExportSession}}\par{}
\penalty10000\vspace{10pt}\penalty10000{\parskip .5pt plus 1pt minus 1pt
                    
{\setlength{\XLingPapertempdim}{\XLingPapersingledigitlistitemwidth+6em}\leftskip\XLingPapertempdim\relax
\interlinepenalty10000
\XLingPaperlistitem{6em}{\XLingPapersingledigitlistitemwidth}{1.}{Ouvrir un projet contenant les données des section qu'on désire exporter}\vspace{3pt}}
{\setlength{\XLingPapertempdim}{\XLingPapersingledigitlistitemwidth+6em}\leftskip\XLingPapertempdim\relax
\interlinepenalty10000
\XLingPaperlistitem{6em}{\XLingPapersingledigitlistitemwidth}{2.}{Dans le menu projet, cliquez {\textbf{Exporter les sessions.}}\\{\textit{La boîte de dialogue {\textbf{Exporter les données}} apparaît.}}\\\vspace{10pt plus 2pt minus 1pt}\setbox0=\vbox{\protect\raggedright\leavevmode
\vspace*{0pt}{\XeTeXpicfile "../images/fr/ExporterDes_fr.png" scaled 750}\\[0pt]\protect\hypertarget{f-NeedsALabel-.xlingpaper.1..styledPaper.1..lingPaper.1..part.1..chapter.2..section1.1..section2.2..ol.1..li.4..figure.1.}{}\XLingPaperaddtocontents{f-NeedsALabel-.xlingpaper.1..styledPaper.1..lingPaper.1..part.1..chapter.2..section1.1..section2.2..ol.1..li.4..figure.1.}\textit{{Figure }}\textit{{9.1}}\textit{{ La boîte de dialogue exporter des données\\}}}\box0\par{}\vspace{10pt plus 2pt minus 1pt}}\vspace{3pt}}
{\setlength{\XLingPapertempdim}{\XLingPapersingledigitlistitemwidth+6em}\leftskip\XLingPapertempdim\relax
\interlinepenalty10000
\XLingPaperlistitem{6em}{\XLingPapersingledigitlistitemwidth}{3.}{Cliquez sur {\textbf{Enregistrer}}.\\{\textit{Les données de sessions du projet seront exportés vers un fichier. Le fichier s'affichera dans le dossier}}.}\vspace{3pt}}
{\setlength{\XLingPapertempdim}{\XLingPapersingledigitlistitemwidth+6em}\leftskip\XLingPapertempdim\relax
\interlinepenalty10000
\XLingPaperlistitem{6em}{\XLingPapersingledigitlistitemwidth}{4.}{Pour ouvrir le fichier, ouvrez le dossier, et double cliquez sur fichier.\\{\textit{Le fichier exporté sera ouvert avec l'application défini pour ce genre de fichier.}}}}
\vspace{\baselineskip}
}{\XLingPaperneedspace{3\baselineskip}
\noindent\rule{\textwidth}{1pt}
{}\penalty10000\vspace{3pt}\XLingPaperneedspace{3\baselineskip}\noindent
\fontsize{12}{14.399999999999999}\selectfont \textbf{{\noindent
\raisebox{\baselineskip}[0pt]{\pdfbookmark[2]{{9.4 } Exporter des personnes}{sExportPeople}}\raisebox{\baselineskip}[0pt]{\protect\hypertarget{sExportPeople}{}}{9.4 }Exporter des personnes}}\markright{Exporter des personnes}
\XLingPaperaddtocontents{sExportPeople}}\par{}
\penalty10000\vspace{10pt}\penalty10000\vspace{0pt}\indent Pour exporter toutes les personnes et les métadonnées associées dans le projet ouvert dans un fichier, procédez comme suit:\par{}{\parskip .5pt plus 1pt minus 1pt
                    
\vspace{\baselineskip}

{\setlength{\XLingPapertempdim}{\XLingPapersingledigitlistitemwidth+6em}\leftskip\XLingPapertempdim\relax
\interlinepenalty10000
\XLingPaperlistitem{6em}{\XLingPapersingledigitlistitemwidth}{1.}{Ouvrez le projet contenant les personnes que vous souhaitez exporter}\vspace{3pt}}
{\setlength{\XLingPapertempdim}{\XLingPapersingledigitlistitemwidth+6em}\leftskip\XLingPapertempdim\relax
\interlinepenalty10000
\XLingPaperlistitem{6em}{\XLingPapersingledigitlistitemwidth}{2.}{Dans le menu {\textbf{Projet}}, cliquez {\textbf{Exporter des personnes}} \\{\textit{La boîte de dialogue {\textbf{Exporter les données}} apparaît:}}\\\vspace*{0pt}{\XeTeXpicfile "../images/fr/ExporterDes_fr.png" scaled 750}}\vspace{3pt}}
{\setlength{\XLingPapertempdim}{\XLingPapersingledigitlistitemwidth+6em}\leftskip\XLingPapertempdim\relax
\interlinepenalty10000
\XLingPaperlistitem{6em}{\XLingPapersingledigitlistitemwidth}{3.}{Dans la boîte de dialogue, cliquez sur la zone Nom de fichier, puis tapez un nom.}\vspace{3pt}}
{\setlength{\XLingPapertempdim}{\XLingPapersingledigitlistitemwidth+6em}\leftskip\XLingPapertempdim\relax
\interlinepenalty10000
\XLingPaperlistitem{6em}{\XLingPapersingledigitlistitemwidth}{4.}{Laissez la sélection Enregistrer en tant que type CSV (délimité par des virgules) (* .csv). Il est recommandé de sauvegarder le fichier dans le dossier du projet, qui est l'emplacement par défaut.}\vspace{3pt}}
{\setlength{\XLingPapertempdim}{\XLingPapersingledigitlistitemwidth+6em}\leftskip\XLingPapertempdim\relax
\interlinepenalty10000
\XLingPaperlistitem{6em}{\XLingPapersingledigitlistitemwidth}{5.}{Cliquez sur Enregistrer.\\ {\textbf{Les données de personnes dans le projet ouvert sont exportées dans un fichier. Le fichier est ensuite affiché dans le dossier.}}}\vspace{3pt}}
{\setlength{\XLingPapertempdim}{\XLingPapersingledigitlistitemwidth+6em}\leftskip\XLingPapertempdim\relax
\interlinepenalty10000
\XLingPaperlistitem{6em}{\XLingPapersingledigitlistitemwidth}{6.}{Pour ouvrir le fichier, ouvrez le dossier, puis double-cliquez sur le fichier.\\ {\textbf{Le fichier d'exportation s'ouvre avec le programme spécifié comme programme par défaut pour le type de fichier.}}}}
\vspace{\baselineskip}
}\vspace{0pt}\clearpage
\thispagestyle{bodyfirstpage}\markboth{Archivage}{Archivage}
\XLingPaperaddtocontents{cArchive}{\XLingPaperneedspace{3\baselineskip}\noindent
\fontsize{18}{21.599999999999998}\selectfont \textbf{{\centering
\raisebox{\baselineskip}[0pt]{\protect\hypertarget{cArchive}{}}\raisebox{\baselineskip}[0pt]{\pdfbookmark[1]{10 Archivage}{cArchive}}10\\}}}\par{}
\vspace{10.8pt}{\XLingPaperneedspace{3\baselineskip}\noindent
\fontsize{18}{21.599999999999998}\selectfont \textbf{{\centering
Archivage\\}}}\par{}
\vspace{21.6pt}{\XLingPaperneedspace{3\baselineskip}
\noindent\rule{\textwidth}{1pt}
{}\penalty10000\vspace{3pt}\XLingPaperneedspace{3\baselineskip}\noindent
\fontsize{12}{14.399999999999999}\selectfont \textbf{{\noindent
\raisebox{\baselineskip}[0pt]{\pdfbookmark[2]{{10.1 } Archive avec RAMP (SIL) ...}{sArchiveRamp}}\raisebox{\baselineskip}[0pt]{\protect\hypertarget{sArchiveRamp}{}}{10.1 }Archive avec RAMP (SIL) ...}}\markright{Archive avec RAMP (SIL) ...}
\XLingPaperaddtocontents{sArchiveRamp}}\par{}
\penalty10000\vspace{10pt}\penalty10000\vspace{0pt}\indent {\textbf{Archiver avec RAMP (SIL) ...Ceci vous permet d'archiver en utilisant le référentiel de protocole d'accès de SIL pour le format d'archivage et de publication électroniques. Il est conçu pour transférer toutes les méta-données dans une archive.}}\par{}{\XLingPaperneedspace{3\baselineskip}
\noindent\rule{\textwidth}{1pt}
{}\penalty10000\vspace{3pt}\XLingPaperneedspace{3\baselineskip}\noindent
\fontsize{12}{14.399999999999999}\selectfont \textbf{{\noindent
\raisebox{\baselineskip}[0pt]{\pdfbookmark[2]{{10.2 } Achiver avec IMDI}{sIMDI}}\raisebox{\baselineskip}[0pt]{\protect\hypertarget{sIMDI}{}}{10.2 }Achiver avec IMDI}}\markright{Achiver avec IMDI}
\XLingPaperaddtocontents{sIMDI}}\par{}
\penalty10000\vspace{10pt}\penalty10000\vspace{0pt}\indent Cela vous permet d'archiver la {\textbf{session}} en tant qu'initiative de métadonnées ISLE, c'est une norme utilisée pour décrire les ressources langagières multimédias et multimodales. Lorsque vous archivez avec IMDI, les fichiers dans Descriptive Documents sont exportés vers une session spéciale nommée 'Project Descriptive Documents' et les fichiers dans Other Documents sont exportés vers une session spéciale nommée 'Other Project Documents'.\par{}\vspace{6pt}\vspace{0pt}\indent Cela vous permet d'archiver la session en tant qu'initiative de métadonnées ISLE, c'est une norme utilisée pour décrire les ressources langagières multimédias et multimodales.\par{}\vspace{6pt}\vspace{0pt}\indent Cette rubrique est destinée à l'archivage à l'aide d'IMDI (vers TLA). Pour les autres archives, vous devez communiquer avec l'équipe pour apprendre ce que vous devez faire.\par{}\pagestyle{body}\XLingPaperendtableofcontents
\pagebreak\end{MainFont}
\end{document}
