\documentclass[10pt]{book}
\setlength{\paperheight}{210mm}
\setlength{\paperwidth}{148mm}
\setlength{\topmargin}{0pt}
\setlength{\voffset}{-43.817244089999996pt}
\setlength{\evensidemargin}{-43.817244089999996pt}
\setlength{\oddsidemargin}{-43.817244089999996pt}
\setlength{\textwidth}{364.195275648pt}
\setlength{\textheight}{483.69685047000013pt}
\setlength{\headheight}{12.5pt}
\setlength{\headsep}{16.45275591pt}
\setlength{\footskip}{10mm}
\DeclareTextSymbol{\textsquarebracketleft}{EU1}{91}
\DeclareTextSymbol{\textsquarebracketright}{EU1}{93}
\usepackage[framemethod=TikZ]{mdframed}
\usepackage{xltxtra}
\usepackage{setspace}
\usepackage[normalem]{ulem}
\usepackage{color}
\usepackage{colortbl}
\usepackage{tabularx}
\usepackage{longtable}
\usepackage{multirow}
\usepackage{booktabs}
\usepackage{calc}
\usepackage{fancyhdr}
\usepackage{fontspec}
\usepackage{hyperref}
\hypersetup{colorlinks=true, citecolor=black, filecolor=black, linkcolor=black, urlcolor=blue, bookmarksopen=true, pdfauthor={NGONO Louis Pascal, John Roettele}, pdfcreator={XLingPaper version 2.31.0 (www.xlingpaper.org)}, pdftitle={SayMore: Supporters Manual (English)}, pdfkeywords={Paratext 8, SIL International, United Bible Societies, Bible Translation}}
\fancypagestyle{frontmattertitle}
{\fancyhf{}
\renewcommand{\headrulewidth}{0pt}
\renewcommand{\footrulewidth}{0pt}
}\fancypagestyle{frontmatterfirstpage}
{\fancyhf{}
\fancyfoot[C]{{\XLingPaperTimesZNewZRomanFontFamily{\fontsize{9}{10.799999999999999}\selectfont \textit{\small\textit{\thepage}}}}}
\renewcommand{\headrulewidth}{0pt}
\renewcommand{\footrulewidth}{0pt}
}\fancypagestyle{frontmatter}
{\fancyhf{}
\fancyhead[LE]{{\XLingPaperTimesZNewZRomanFontFamily{\fontsize{9}{10.799999999999999}\selectfont \textit{\small\textit{\thepage}}}}}
\fancyhead[RE]{{\XLingPaperTimesZNewZRomanFontFamily{\fontsize{9}{10.799999999999999}\selectfont \textit{\small\textit{\leftmark}}}}}
\fancyhead[LO]{{\XLingPaperTimesZNewZRomanFontFamily{\fontsize{9}{10.799999999999999}\selectfont \textit{\small\textit{\leftmark}}}}}
\fancyhead[RO]{{\XLingPaperTimesZNewZRomanFontFamily{\fontsize{9}{10.799999999999999}\selectfont \textit{\small\textit{\thepage}}}}}
\renewcommand{\headrulewidth}{0pt}
\renewcommand{\footrulewidth}{0pt}
}\fancypagestyle{bodyfirstpage}
{\fancyhf{}
\fancyfoot[C]{{\XLingPaperTimesZNewZRomanFontFamily{\fontsize{10}{12}\selectfont \textit{\small\textit{\thepage}}}}}
\renewcommand{\headrulewidth}{0pt}
\renewcommand{\footrulewidth}{0pt}
}\fancypagestyle{body}
{\fancyhf{}
\fancyhead[LE]{{\XLingPaperTimesZNewZRomanFontFamily{\fontsize{10}{12}\selectfont \textit{\small\textit{\thepage}}}}}
\fancyhead[RE]{{\XLingPaperTimesZNewZRomanFontFamily{\fontsize{10}{12}\selectfont \textit{\small\textit{\leftmark}}}}}
\fancyhead[LO]{{\XLingPaperTimesZNewZRomanFontFamily{\fontsize{10}{12}\selectfont \textit{\small\textit{\rightmark}}}}}
\fancyhead[RO]{{\XLingPaperTimesZNewZRomanFontFamily{\fontsize{10}{12}\selectfont \textit{\small\textit{\thepage}}}}}
\renewcommand{\headrulewidth}{0pt}
\renewcommand{\footrulewidth}{0pt}
}\setmainfont{Times New Roman}
\font\MainFont="Times New Roman" at 10pt
\newfontfamily{\XLingPaperCharisZSILZSmallZCapsFontFamily}{Charis SIL Small Caps}
\newfontfamily{\XLingPaperCourierZNewFontFamily}{Courier New}
\newfontfamily{\XLingPaperTimesZNewZRomanFontFamily}{Times New Roman}
\definecolor{FTColorA}{HTML}{FFE6FF}
\setlength{\parindent}{0em}
\catcode`^^^^200b=\active
\def^^^^200b{\hskip0pt}
\let\origdoublepage\cleardoublepage
\newcommand{\clearemptydoublepage}{\clearpage{\pagestyle{empty}\origdoublepage}}\renewenvironment{quotation}{\list{}{\leftmargin=10mm\rightmargin=10mm}\item[]{}}{\endlist}
\clubpenalty=10000
\widowpenalty=10000
\begin{document}
\baselineskip=\glueexpr\baselineskip + 0pt plus 2pt minus 1pt\relax
\renewcommand{\footnotesize}{\fontsize{8}{9.6}\selectfont }
\newlength{\leveloneindent}
\newlength{\levelonewidth}
\newlength{\leveltwoindent}
\newlength{\leveltwowidth}
\newlength{\levelthreeindent}
\newlength{\levelthreewidth}
\newlength{\levelfourindent}
\newlength{\levelfourwidth}
\newlength{\levelfiveindent}
\newlength{\levelfivewidth}
\newlength{\levelsixindent}
\newlength{\levelsixwidth}
\newdimen\XLingPapertempdim
                \newdimen\XLingPapertempdimletter
                \newcommand{\XLingPapertableofcontents}{\immediate\openout8 = \jobname.toc\relax
\immediate\write8{<toc>}}
\newcommand{\XLingPaperaddtocontents}[1]{\immediate\write8{<tocline ref="#1" page="\thepage"/>}}
\newcommand{\XLingPaperendtableofcontents}{\immediate\write8{</toc>}\closeout8\relax
}
\newcommand{\XLingPaperdotfill}{\leaders\hbox{$\mathsurround 0pt\mkern 4.5 mu\hbox{.}\mkern 4.5 mu$}\hfill}
\newcommand{\XLingPaperdottedtocline}[4]{
\newdimen\XLingPapertempdim
\vskip0pt plus .2pt{
\leftskip#1\relax% left glue for indent
\rightskip\XLingPapertocrmarg% right glue for for right margin
\parfillskip-\rightskip% so can go into margin if need be???
\parindent#1\relax
\interlinepenalty10000
\leavevmode
\XLingPapertempdim#2\relax% numwidth
\advance\leftskip\XLingPapertempdim\null\nobreak\hskip-\leftskip{#3}\nobreak
\XLingPaperdotfill\nobreak
\hbox to\XLingPaperpnumwidth{\hfil\normalfont\normalcolor#4}
\par}}
\newlength{\XLingPaperpnumwidth}
\newlength{\XLingPapertocrmarg}
\setlength{\XLingPaperpnumwidth}{1.55em}\setlength{\XLingPapertocrmarg}{\XLingPaperpnumwidth+1em}
\newlength{\XLingPaperinterlinearsourcewidth}
\newlength{\XLingPaperinterlinearsourcegapwidth}
\settowidth{\XLingPaperinterlinearsourcegapwidth}{  }
\newlength{\XLingPaperlistinexampleindent}
\newlength{\XLingPaperisocodewidth}\setlength{\XLingPaperlistinexampleindent}{.125in+ 2.75em}
\newlength{\XLingPaperlistitemindent}
\newlength{\XLingPaperbulletlistitemwidth}\settowidth{\XLingPaperbulletlistitemwidth}{•\ }\newlength{\XLingPapersingledigitlistitemwidth}
\settowidth{\XLingPapersingledigitlistitemwidth}{8.\ }\newlength{\XLingPaperdoubledigitlistitemwidth}
\settowidth{\XLingPaperdoubledigitlistitemwidth}{88.\ }\newlength{\XLingPapertripledigitlistitemwidth}
\settowidth{\XLingPapertripledigitlistitemwidth}{888.\ }\newlength{\XLingPapersingleletterlistitemwidth}
\settowidth{\XLingPapersingleletterlistitemwidth}{m.\ }\newlength{\XLingPaperdoubleletterlistitemwidth}
\settowidth{\XLingPaperdoubleletterlistitemwidth}{mm.\ }\newlength{\XLingPapertripleletterlistitemwidth}
\settowidth{\XLingPapertripleletterlistitemwidth}{mmm.\ }\newlength{\XLingPaperromanviilistitemwidth}
\settowidth{\XLingPaperromanviilistitemwidth}{vii.\ }\newlength{\XLingPaperromanviiilistitemwidth}
\settowidth{\XLingPaperromanviiilistitemwidth}{viii.\ }\newlength{\XLingPaperromanxviiilistitemwidth}
\settowidth{\XLingPaperromanxviiilistitemwidth}{xviii.\ }\newlength{\XLingPaperspacewidth}
\settowidth{\XLingPaperspacewidth}{\ }
\newcommand{\XLingPaperneedspace}[1]{\penalty-100\begingroup
\newdimen{\XLingPaperspaceneeded}
\newdimen{\XLingPaperspaceavailable}
\setlength{\XLingPaperspaceneeded}{#1}%
\XLingPaperspaceavailable\pagegoal \advance\XLingPaperspaceavailable-\pagetotal
\ifdim \XLingPaperspaceneeded>\XLingPaperspaceavailable
\ifdim \XLingPaperspaceavailable>0pt
\vfil
\fi
\break
\fi\endgroup}
\newcommand{\XLingPaperlistitem}[4]{
\newdimen\XLingPapertempdim
\vskip0pt plus .2pt{
\leftskip#1\relax% left glue for indent
\parindent#1\relax
\interlinepenalty10000
\leavevmode
\XLingPapertempdim#2\relax% label width
\advance\leftskip\XLingPapertempdim\null\nobreak\hskip-\leftskip\hbox to\XLingPapertempdim{\hfil\normalfont\normalcolor#3\ }{#4}\nobreak
\par}}
\newcommand{\XLingPaperexample}[5]{
\newdimen\XLingPapertempdim
\vskip0pt plus .2pt{
\leftskip#1\relax% left glue for indent
\hspace*{#1}\relax
\rightskip#2\relax% right glue for indent
\interlinepenalty10000
\leavevmode
\XLingPapertempdim#3\relax% example number width
\advance\leftskip\XLingPapertempdim\null\nobreak\hskip-\leftskip\hbox to\XLingPapertempdim{\normalfont\normalcolor#4\hfil}{#5}\nobreak
\par}}
\newcommand{\XLingPaperexampleintable}[5]{
\newdimen\XLingPapertempdim
\leftskip#1\relax% left glue for indent
\hspace*{#1}\relax
\rightskip#2\relax% right glue for indent
\interlinepenalty10000
\leavevmode
\XLingPapertempdim#3\relax% example number width
\hbox to\XLingPapertempdim{\normalfont\normalcolor#4\hfil}{
\begin{tabular}
[t]{@{}l@{}}#5\end{tabular}
}\nobreak
}
\newcommand{\XLingPaperfree}[2]{\vskip0pt plus .2pt{
\leftskip#1\relax% left glue for indent
\parindent#1\relax
\interlinepenalty10000
\leavevmode{#2}\nobreak
\par}}
\newcommand{\XLingPaperlistinterlinear}[5]{\vskip0pt plus .2pt{\hspace*{#1}\hspace*{#2}
\XLingPapertempdimletter#3\relax% letter width
\advance\leftskip\XLingPapertempdimletter\null\nobreak\hskip-\leftskip\hspace*{-.3em}\hbox to\XLingPapertempdimletter{\normalfont\normalcolor#4\ \hfil}{#5}\nobreak
\par}}
\newcommand{\XLingPaperlistinterlinearintable}[5]{
\XLingPapertempdimletter#3\relax% letter width
\hspace*{-.3em}\hbox to\XLingPapertempdimletter{\normalfont\normalcolor#4\ \hfil}{
\begin{tabular}
[t]{@{}l@{}}#5\end{tabular}
}\nobreak
}

\newlength{\XLingPaperexamplefreeindent}\setlength{\XLingPaperexamplefreeindent}{-.3 em}\newskip\XLingPaperinterwordskip
\XLingPaperinterwordskip=6.66666pt plus 3.33333pt minus 2.22222pt
\def\XLingPaperintspace{\hskip\XLingPaperinterwordskip}
\def\XLingPaperraggedright{\rightskip=0pt plus1fil\pretolerance=10000}\raggedbottom
\pagestyle{fancy}
\begin{MainFont}
\XLingPapertableofcontents\pagenumbering{roman}
\pagestyle{frontmattertitle}\pagestyle{frontmattertitle}{\clearpage
\vspace*{1cm}\XLingPaperneedspace{3\baselineskip}\noindent
\fontsize{18}{21.599999999999998}\selectfont \textbf{{\centering
\vspace*{0pt}{\XeTeXpicfile "../images/SayMore5.jpg" scaled 750} \\SayMore\\}}}\par{}
{\vspace{.25in}\XLingPaperneedspace{3\baselineskip}\noindent
\fontsize{14}{16.8}\selectfont \textbf{{\centering
Supporters Manual (English)\\}}}\par{}
{\clearpage
\vspace*{1cm}\XLingPaperneedspace{3\baselineskip}\noindent
\fontsize{18}{21.599999999999998}\selectfont \textbf{{\centering
\vspace*{0pt}{\XeTeXpicfile "../images/SayMore5.jpg" scaled 750} \\SayMore\\}}}\par{}
{\vspace{.25in}\XLingPaperneedspace{3\baselineskip}\noindent
\fontsize{14}{16.8}\selectfont \textbf{{\centering
Supporters Manual (English)\\}}}\par{}
{\XLingPaperneedspace{3\baselineskip}\noindent
\textit{{\centering
NGONO Louis Pascal\\}}}\par{}
{\XLingPaperneedspace{3\baselineskip}\noindent
\textit{{\centering
John Roettele\\}}}\par{}
{\XLingPaperneedspace{3\baselineskip}\noindent
\textit{{\centering
SIL International\\}}}\par{}
{\XLingPaperneedspace{3\baselineskip}\noindent
\fontsize{10}{12}\selectfont {\centering
May. 2017\\}}\par{}
\clearpage
\pagestyle{frontmatter}\thispagestyle{frontmatterfirstpage}\thispagestyle{frontmatterfirstpage}{\vspace{12.2pt}\XLingPaperneedspace{3\baselineskip}\noindent
\fontsize{18}{21.599999999999998}\selectfont \textbf{{\centering
\raisebox{\baselineskip}[0pt]{\pdfbookmark[1]{Table of Contents}{rXLingPapContents}}\raisebox{\baselineskip}[0pt]{\protect\hypertarget{rXLingPapContents}{}}Table of Contents\\}}\markboth{Table of Contents}{Table of Contents}
\XLingPaperaddtocontents{rXLingPapContents}}\penalty10000\par{}
\vspace{10.8pt}\vspace{0pt}\hyperlink{SayMSuppMan}{\centering{Part I Introduction\\}}\hyperlink{sInstall}{\XLingPaperdottedtocline{0pt}{0pt}{1 Install SayMore}{2}
}\settowidth{\leveltwoindent}{{1 }\ }\settowidth{\leveltwowidth}{{1.1 }\thinspace\thinspace}\hyperlink{sInterface}{\XLingPaperdottedtocline{\leveltwoindent}{\leveltwowidth}{{1.1 } Change user interface language}{3}
}\hyperlink{cTour}{\XLingPaperdottedtocline{0pt}{0pt}{2 A Tour around SayMore}{4}
}\settowidth{\leveltwoindent}{{2 }\ }\settowidth{\leveltwowidth}{{2.1 }\thinspace\thinspace}\hyperlink{sProjectTab}{\XLingPaperdottedtocline{\leveltwoindent}{\leveltwowidth}{{2.1 } Project Tab}{4}
}\settowidth{\leveltwoindent}{{2 }\ }\settowidth{\leveltwowidth}{{2.2 }\thinspace\thinspace}\hyperlink{sSessionTab}{\XLingPaperdottedtocline{\leveltwoindent}{\leveltwowidth}{{2.2 } Session Tab}{5}
}\settowidth{\leveltwoindent}{{2 }\ }\settowidth{\leveltwowidth}{{2.3 }\thinspace\thinspace}\hyperlink{sPeopleTab}{\XLingPaperdottedtocline{\leveltwoindent}{\leveltwowidth}{{2.3 } People Tab}{6}
}\hyperlink{sMenus}{\XLingPaperdottedtocline{0pt}{0pt}{3 Menus}{8}
}\settowidth{\leveltwoindent}{{3 }\ }\settowidth{\leveltwowidth}{{3.1 }\thinspace\thinspace}\hyperlink{sMenuProject}{\XLingPaperdottedtocline{\leveltwoindent}{\leveltwowidth}{{3.1 } Menu: Project}{8}
}\settowidth{\leveltwoindent}{{3 }\ {3.1 }\ }\settowidth{\leveltwowidth}{{3.1.1 }\thinspace\thinspace}\hyperlink{sQuitSaymore}{\XLingPaperdottedtocline{\leveltwoindent}{\leveltwowidth}{{3.1.1 } Quit}{8}
}\settowidth{\leveltwoindent}{{3 }\ }\settowidth{\leveltwowidth}{{3.2 }\thinspace\thinspace}\hyperlink{sMenuSession}{\XLingPaperdottedtocline{\leveltwoindent}{\leveltwowidth}{{3.2 } Menu: Session}{9}
}\settowidth{\leveltwoindent}{{3 }\ }\settowidth{\leveltwowidth}{{3.3 }\thinspace\thinspace}\hyperlink{sMenuPerson}{\XLingPaperdottedtocline{\leveltwoindent}{\leveltwowidth}{{3.3 } Menu: Person}{9}
}\settowidth{\leveltwoindent}{{3 }\ }\settowidth{\leveltwowidth}{{3.4 }\thinspace\thinspace}\hyperlink{sMenuHelp}{\XLingPaperdottedtocline{\leveltwoindent}{\leveltwowidth}{{3.4 } Menu: Help}{9}
}\settowidth{\leveltwoindent}{{3 }\ {3.4 }\ }\settowidth{\leveltwowidth}{{3.4.1 }\thinspace\thinspace}\hyperlink{sHelp}{\XLingPaperdottedtocline{\leveltwoindent}{\leveltwowidth}{{3.4.1 } Help...}{10}
}\settowidth{\leveltwoindent}{{3 }\ {3.4 }\ }\settowidth{\leveltwowidth}{{3.4.2 }\thinspace\thinspace}\hyperlink{sAbout}{\XLingPaperdottedtocline{\leveltwoindent}{\leveltwowidth}{{3.4.2 } About...}{10}
}\vspace{0pt}\hyperlink{pGestion}{\centering{Part II Project, Session, and User Management\\}}\hyperlink{cCreateProjectOver}{\XLingPaperdottedtocline{0pt}{0pt}{4 Create or Open a Project}{12}
}\settowidth{\leveltwoindent}{{4 }\ }\settowidth{\leveltwowidth}{{4.1 }\thinspace\thinspace}\hyperlink{sCreateProject}{\XLingPaperdottedtocline{\leveltwoindent}{\leveltwowidth}{{4.1 } Create a Project}{12}
}\settowidth{\leveltwoindent}{{4 }\ }\settowidth{\leveltwowidth}{{4.2 }\thinspace\thinspace}\hyperlink{sOpenProject}{\XLingPaperdottedtocline{\leveltwoindent}{\leveltwowidth}{{4.2 } Open an Existing Project}{13}
}\settowidth{\leveltwoindent}{{4 }\ }\settowidth{\leveltwowidth}{{4.3 }\thinspace\thinspace}\hyperlink{sOpenOtherProject}{\XLingPaperdottedtocline{\leveltwoindent}{\leveltwowidth}{{4.3 } Open a different Project}{14}
}\hyperlink{cProjectMeta}{\XLingPaperdottedtocline{0pt}{0pt}{5 Managing a Project}{15}
}\settowidth{\leveltwoindent}{{5 }\ }\settowidth{\leveltwowidth}{{5.1 }\thinspace\thinspace}\hyperlink{s}{\XLingPaperdottedtocline{\leveltwoindent}{\leveltwowidth}{{5.1 } About This Project}{15}
}\settowidth{\leveltwoindent}{{5 }\ }\settowidth{\leveltwowidth}{{5.2 }\thinspace\thinspace}\hyperlink{sAccessProt}{\XLingPaperdottedtocline{\leveltwoindent}{\leveltwowidth}{{5.2 } Access Protocol}{15}
}\settowidth{\leveltwoindent}{{5 }\ }\settowidth{\leveltwowidth}{{5.3 }\thinspace\thinspace}\hyperlink{s}{\XLingPaperdottedtocline{\leveltwoindent}{\leveltwowidth}{{5.3 } Project Description Documents}{15}
}\settowidth{\leveltwoindent}{{5 }\ }\settowidth{\leveltwowidth}{{5.4 }\thinspace\thinspace}\hyperlink{sOtherDocs}{\XLingPaperdottedtocline{\leveltwoindent}{\leveltwowidth}{{5.4 } Other Documents}{16}
}\settowidth{\leveltwoindent}{{5 }\ }\settowidth{\leveltwowidth}{{5.5 }\thinspace\thinspace}\hyperlink{s}{\XLingPaperdottedtocline{\leveltwoindent}{\leveltwowidth}{{5.5 } Project Progress}{15}
}\hyperlink{cSessions}{\XLingPaperdottedtocline{0pt}{0pt}{6 Managing Sessions}{18}
}\settowidth{\leveltwoindent}{{6 }\ }\settowidth{\leveltwowidth}{{6.1 }\thinspace\thinspace}\hyperlink{sCreateSession}{\XLingPaperdottedtocline{\leveltwoindent}{\leveltwowidth}{{6.1 } Create a new Session}{18}
}\settowidth{\leveltwoindent}{{6 }\ {6.1 }\ }\settowidth{\leveltwowidth}{{6.1.1 }\thinspace\thinspace}\hyperlink{sNewFromDevice}{\XLingPaperdottedtocline{\leveltwoindent}{\leveltwowidth}{{6.1.1 } New From Device...}{18}
}\settowidth{\leveltwoindent}{{6 }\ {6.1 }\ }\settowidth{\leveltwowidth}{{6.1.2 }\thinspace\thinspace}\hyperlink{sNewFromRecording}{\XLingPaperdottedtocline{\leveltwoindent}{\leveltwowidth}{{6.1.2 } New From Recording...}{19}
}\settowidth{\leveltwoindent}{{6 }\ }\settowidth{\leveltwowidth}{{6.2 }\thinspace\thinspace}\hyperlink{sDeleteSession}{\XLingPaperdottedtocline{\leveltwoindent}{\leveltwowidth}{{6.2 } Delete Session...}{20}
}\settowidth{\leveltwoindent}{{6 }\ }\settowidth{\leveltwowidth}{{6.3 }\thinspace\thinspace}\hyperlink{sSessMeta1}{\XLingPaperdottedtocline{\leveltwoindent}{\leveltwowidth}{{6.3 } Session Metadata}{20}
}\hyperlink{cPeople}{\XLingPaperdottedtocline{0pt}{0pt}{7 Managing People}{22}
}\settowidth{\leveltwoindent}{{7 }\ }\settowidth{\leveltwowidth}{{7.1 }\thinspace\thinspace}\hyperlink{sNewPerson}{\XLingPaperdottedtocline{\leveltwoindent}{\leveltwowidth}{{7.1 } Add a New Person}{22}
}\settowidth{\leveltwoindent}{{7 }\ }\settowidth{\leveltwowidth}{{7.2 }\thinspace\thinspace}\hyperlink{sInformedConsent}{\XLingPaperdottedtocline{\leveltwoindent}{\leveltwowidth}{{7.2 } Informed Consent }{22}
}\settowidth{\leveltwoindent}{{7 }\ }\settowidth{\leveltwowidth}{{7.3 }\thinspace\thinspace}\hyperlink{sDeletePerson}{\XLingPaperdottedtocline{\leveltwoindent}{\leveltwowidth}{{7.3 } Delete Person...}{23}
}\vspace{0pt}\hyperlink{pTranscribe}{\centering{Part III Transcribing Audio and Video files\\}}\hyperlink{cTranscribe}{\XLingPaperdottedtocline{0pt}{0pt}{8 Overview of Transcription Process}{25}
}\settowidth{\leveltwoindent}{{8 }\ }\settowidth{\leveltwowidth}{{8.1 }\thinspace\thinspace}\hyperlink{sMetaMaybe}{\XLingPaperdottedtocline{\leveltwoindent}{\leveltwowidth}{{8.1 } Session Metadata}{26}
}\settowidth{\leveltwoindent}{{8 }\ }\settowidth{\leveltwowidth}{{8.2 }\thinspace\thinspace}\hyperlink{sAddContributors}{\XLingPaperdottedtocline{\leveltwoindent}{\leveltwowidth}{{8.2 } Add Session Contributors}{26}
}\settowidth{\leveltwoindent}{{8 }\ }\settowidth{\leveltwowidth}{{8.3 }\thinspace\thinspace}\hyperlink{sStartAnnot}{\XLingPaperdottedtocline{\leveltwoindent}{\leveltwowidth}{{8.3 } Start Annotating}{26}
}\settowidth{\leveltwoindent}{{8 }\ {8.3 }\ }\settowidth{\leveltwowidth}{{8.3.1 }\thinspace\thinspace}\hyperlink{sVidtoAudio}{\XLingPaperdottedtocline{\leveltwoindent}{\leveltwowidth}{{8.3.1 } Convert Video to Audio}{27}
}\settowidth{\leveltwoindent}{{8 }\ {8.3 }\ }\settowidth{\leveltwowidth}{{8.3.2 }\thinspace\thinspace}\hyperlink{sStartAnnot}{\XLingPaperdottedtocline{\leveltwoindent}{\leveltwowidth}{{8.3.2 } Automatic Segmentation}{26}
}\settowidth{\leveltwoindent}{{8 }\ }\settowidth{\leveltwowidth}{{8.4 }\thinspace\thinspace}\hyperlink{sChangeSegmentation}{\XLingPaperdottedtocline{\leveltwoindent}{\leveltwowidth}{{8.4 } Adjust Segmentation}{29}
}\settowidth{\leveltwoindent}{{8 }\ }\settowidth{\leveltwowidth}{{8.5 }\thinspace\thinspace}\hyperlink{sSegMan}{\XLingPaperdottedtocline{\leveltwoindent}{\leveltwowidth}{{8.5 } Manual Segmentation}{29}
}\settowidth{\leveltwoindent}{{8 }\ }\settowidth{\leveltwowidth}{{8.6 }\thinspace\thinspace}\hyperlink{sCarefulSpeech}{\XLingPaperdottedtocline{\leveltwoindent}{\leveltwowidth}{{8.6 } Careful Speech Transcription}{31}
}\settowidth{\leveltwoindent}{{8 }\ }\settowidth{\leveltwowidth}{{8.7 }\thinspace\thinspace}\hyperlink{sOralTransl}{\XLingPaperdottedtocline{\leveltwoindent}{\leveltwowidth}{{8.7 } Oral Translation}{31}
}\settowidth{\leveltwoindent}{{8 }\ }\settowidth{\leveltwowidth}{{8.8 }\thinspace\thinspace}\hyperlink{sWrittenTrans}{\XLingPaperdottedtocline{\leveltwoindent}{\leveltwowidth}{{8.8 } Written Transcription}{31}
}\settowidth{\leveltwoindent}{{8 }\ }\settowidth{\leveltwowidth}{{8.9 }\thinspace\thinspace}\hyperlink{sWritTransl}{\XLingPaperdottedtocline{\leveltwoindent}{\leveltwowidth}{{8.9 } Written Translation}{33}
}\vspace{0pt}\hyperlink{pArchiving}{\centering{Part IV Archiving\\}}\hyperlink{cExport}{\XLingPaperdottedtocline{0pt}{0pt}{9 Exporting Data}{35}
}\settowidth{\leveltwoindent}{{9 }\ }\settowidth{\leveltwowidth}{{9.1 }\thinspace\thinspace}\hyperlink{sSubtitles}{\XLingPaperdottedtocline{\leveltwoindent}{\leveltwowidth}{{9.1 } Export Subtitles}{35}
}\settowidth{\leveltwoindent}{{9 }\ }\settowidth{\leveltwowidth}{{9.2 }\thinspace\thinspace}\hyperlink{sSubtitles}{\XLingPaperdottedtocline{\leveltwoindent}{\leveltwowidth}{{9.2 } Export Audacity Labels}{35}
}\settowidth{\leveltwoindent}{{9 }\ }\settowidth{\leveltwowidth}{{9.3 }\thinspace\thinspace}\hyperlink{sExportSession}{\XLingPaperdottedtocline{\leveltwoindent}{\leveltwowidth}{{9.3 } Export Sessions}{35}
}\settowidth{\leveltwoindent}{{9 }\ }\settowidth{\leveltwowidth}{{9.4 }\thinspace\thinspace}\hyperlink{sExportPeople}{\XLingPaperdottedtocline{\leveltwoindent}{\leveltwowidth}{{9.4 } Export People}{36}
}\hyperlink{cArchive}{\XLingPaperdottedtocline{0pt}{0pt}{10 Archiving Overview}{37}
}\settowidth{\leveltwoindent}{{10 }\ }\settowidth{\leveltwowidth}{{10.1 }\thinspace\thinspace}\hyperlink{sArchiveRamp}{\XLingPaperdottedtocline{\leveltwoindent}{\leveltwowidth}{{10.1 } Archive with RAMP (SIL)...}{37}
}\clearpage
\pagenumbering{arabic}{\clearpage
\XLingPaperneedspace{3\baselineskip}\noindent
\fontsize{18}{21.599999999999998}\selectfont \textbf{{\centering
\thispagestyle{empty}\raisebox{\baselineskip}[0pt]{\pdfbookmark[1]{Part I Introduction}{SayMSuppMan}}\raisebox{\baselineskip}[0pt]{\protect\hypertarget{SayMSuppMan}{}}Part I\\}}}\par{}
\vspace{10.8pt}{\XLingPaperneedspace{3\baselineskip}\noindent
\fontsize{18}{21.599999999999998}\selectfont \textbf{{\centering
Introduction\\}}}\par{}
\vspace{21.6pt}\vspace{0pt}\indent SayMore is a tool you use to organize video, audio, image, and various additional files with appropriate meta data. SayMore also helps you keep track of your recording progress. You can easily see which sessions are In progress, Incoming, or skipped. When you want to use your data elsewhere, SayMore allows you to export to Audacity, ELAN, FLEx, Toolbox, or YouTube. When you are ready, SayMore can help archive your project or session in IMDI format, ready to use with ARBIL or other IMDI-compatible utilities. SIL members can do the same to quickly submit a package of all the relevant files and meta data using SIL's RAMP archive-submission application.\par{}\vspace{6pt}\clearpage
\thispagestyle{bodyfirstpage}\markboth{Install SayMore}{Install SayMore}
\XLingPaperaddtocontents{sInstall}{\XLingPaperneedspace{3\baselineskip}\noindent
\fontsize{18}{21.599999999999998}\selectfont \textbf{{\centering
\raisebox{\baselineskip}[0pt]{\protect\hypertarget{sInstall}{}}\raisebox{\baselineskip}[0pt]{\pdfbookmark[1]{1 Install SayMore}{sInstall}}1\\}}}\par{}
\vspace{10.8pt}{\XLingPaperneedspace{3\baselineskip}\noindent
\fontsize{18}{21.599999999999998}\selectfont \textbf{{\centering
Install SayMore\\}}}\par{}
\vspace{21.6pt}\vspace{0pt}\indent SayMore can be downloaded from \href{https://software.sil.org/saymore/download/ }{\textcolor[rgb]{0,0,1}{\uline{https://software.sil.org/saymore/download/ }}} . Pay attention to the technical requirements (at time of writing, Saymore requires Windows 7, 8, or 10, a PDF reader, and Microsoft .Net 4.6).\par{}\vspace{6pt}\vspace{0pt}\indent If you are going to be away from fast Internet, go ahead and download {\textbf{FFmpeg}}. {\textbf{FFMpeg}} is the open source tool SayMore uses to convert media files from the ones your device outputs to what you want for archiving.\\ \\Also if you are holding a workshop go ahead and download {\textbf{.Net 4.6 Standalone Installer}} and take it with you in case some participants don't have it.\par{}\vspace{6pt}\vspace{0pt}\indent After installation, open the SayMore application. You will see this screen:\par{}\vspace{6pt}\vspace{10pt plus 2pt minus 1pt}\setbox0=\vbox{\protect\raggedright\leavevmode
\vspace*{0pt}{\XeTeXpicfile "../images/en/Open_Create_en.png" scaled 750}\\[0pt]\protect\hypertarget{fSplashSaymore}{}\XLingPaperaddtocontents{fSplashSaymore}\textit{{Figure }}\textit{{1.1}}\textit{{ First time opening of SayMore\\}}}\box0\par{}\vspace{10pt plus 2pt minus 1pt}\vspace{0pt}\indent Here you can re-open previous projects, browse for projects that might be stored on a USB device, and Create new, blank project.\par{}\vspace{6pt}{\XLingPaperneedspace{3\baselineskip}
\noindent\rule{\textwidth}{1pt}
{}\penalty10000\vspace{3pt}\XLingPaperneedspace{3\baselineskip}\noindent
\fontsize{12}{14.399999999999999}\selectfont \textbf{{\noindent
\raisebox{\baselineskip}[0pt]{\pdfbookmark[2]{{1.1 } Change user interface language}{sInterface}}\raisebox{\baselineskip}[0pt]{\protect\hypertarget{sInterface}{}}{1.1 }Change user interface language}}\markright{Change user interface language}
\XLingPaperaddtocontents{sInterface}}\par{}
\penalty10000\vspace{10pt}\penalty10000\vspace{0pt}\indent This allows you to change the Language interface. \\\par{}\vspace{6pt}\vspace{10pt plus 2pt minus 1pt}\setbox0=\vbox{\protect\raggedright\leavevmode
\vspace*{0pt}{\XeTeXpicfile "../images/en/UserInterfaceLang_en.png" scaled 750}\\[0pt]\protect\hypertarget{f-NeedsALabel-.xlingpaper.1..styledPaper.1..lingPaper.1..part.1..chapter.2..section1.1..section2.5..figure.1.}{}\XLingPaperaddtocontents{f-NeedsALabel-.xlingpaper.1..styledPaper.1..lingPaper.1..part.1..chapter.2..section1.1..section2.5..figure.1.}\textit{{Figure }}\textit{{1.2}}\textit{{ User Interface Language Drop down\\}}}\box0\par{}\vspace{10pt plus 2pt minus 1pt}\vspace{0pt}\indent If the language you need is not listed, or you need to edit translated UI content:\par{}{\parskip .5pt plus 1pt minus 1pt

\vspace{\baselineskip}

{\setlength{\XLingPapertempdim}{\XLingPaperbulletlistitemwidth+6em}\leftskip\XLingPapertempdim\relax
\interlinepenalty10000
\XLingPaperlistitem{6em}{\XLingPaperbulletlistitemwidth}{•}{Click the {\textup{\textmd{\textcolor[rgb]{0,0.2,0.8}{\uline{I want to localize SayMore for another language}}}}} hyperlink. The Localize User Interface dialog box opens so you can use it.}\vspace{3pt}}
{\setlength{\XLingPapertempdim}{\XLingPaperbulletlistitemwidth+6em}\leftskip\XLingPapertempdim\relax
\interlinepenalty10000
\XLingPaperlistitem{6em}{\XLingPaperbulletlistitemwidth}{•}{Optionally, get a shared UI language file and a translated Access Protocol file from another user.}}
\vspace{\baselineskip}
}\vspace{0pt}\vspace{6pt}
\begin{mdframed}
[backgroundcolor=FTColorA,skipabove=3pt,skipbelow=3pt,innermargin=2cm,outermargin=2cm,innertopmargin=.03in,innerbottommargin=.03in,innerleftmargin=.125in,innerrightmargin=.125in,align=left]\vspace{0pt}\indent If not all words are changed to the selected language, just close and re-open SayMore\par{}\end{mdframed}
\clearpage
\thispagestyle{bodyfirstpage}\markboth{A Tour around SayMore}{A Tour around SayMore}
\XLingPaperaddtocontents{cTour}{\XLingPaperneedspace{3\baselineskip}\noindent
\fontsize{18}{21.599999999999998}\selectfont \textbf{{\centering
\raisebox{\baselineskip}[0pt]{\protect\hypertarget{cTour}{}}\raisebox{\baselineskip}[0pt]{\pdfbookmark[1]{2 A Tour around SayMore}{cTour}}2\\}}}\par{}
\vspace{10.8pt}{\XLingPaperneedspace{3\baselineskip}\noindent
\fontsize{18}{21.599999999999998}\selectfont \textbf{{\centering
A Tour around SayMore\\}}}\par{}
\vspace{21.6pt}\vspace{0pt}\indent SayMore has only four menu items and three tabs. The menu {\textbf{Session}} is greyed out until you clique on the {\textbf{Sessions}} tab. This is the same for menu {\textbf{Person}} and tab {\textbf{People}}.\par{}\vspace{6pt}\vspace{10pt plus 2pt minus 1pt}\setbox0=\vbox{\protect\raggedright\leavevmode
\vspace*{0pt}{\XeTeXpicfile "../images/en/MenusAndTabs.png" scaled 750}\\[0pt]\protect\hypertarget{f-NeedsALabel-.xlingpaper.1..styledPaper.1..lingPaper.1..part.1..chapter.2..figure.1.}{}\XLingPaperaddtocontents{f-NeedsALabel-.xlingpaper.1..styledPaper.1..lingPaper.1..part.1..chapter.2..figure.1.}\textit{{Figure }}\textit{{2.1}}\textit{{ Menu and Tab items\\}}}\box0\par{}\vspace{10pt plus 2pt minus 1pt}\vspace{0pt}\indent The next sections will go over what you will find in the Tab layouts.\\\par{}\vspace{6pt}{\XLingPaperneedspace{3\baselineskip}
\noindent\rule{\textwidth}{1pt}
{}\penalty10000\vspace{3pt}\XLingPaperneedspace{3\baselineskip}\noindent
\fontsize{12}{14.399999999999999}\selectfont \textbf{{\noindent
\raisebox{\baselineskip}[0pt]{\pdfbookmark[2]{{2.1 } Project Tab}{sProjectTab}}\raisebox{\baselineskip}[0pt]{\protect\hypertarget{sProjectTab}{}}{2.1 }Project Tab}}\markright{Project Tab}
\XLingPaperaddtocontents{sProjectTab}}\par{}
\penalty10000\vspace{10pt}\penalty10000\vspace{0pt}\indent The project tab is where you are able to specify and keep track of what the project is about.\par{}\vspace{6pt}\vspace{10pt plus 2pt minus 1pt}\setbox0=\vbox{\protect\raggedright\leavevmode
\vspace*{0pt}{\XeTeXpicfile "../images/en/ProjectTab_en.png" scaled 750}\\[0pt]\protect\hypertarget{f-NeedsALabel-.xlingpaper.1..styledPaper.1..lingPaper.1..part.1..chapter.2..section1.1..figure.1.}{}\XLingPaperaddtocontents{f-NeedsALabel-.xlingpaper.1..styledPaper.1..lingPaper.1..part.1..chapter.2..section1.1..figure.1.}\textit{{Figure }}\textit{{2.2}}\textit{{ Project Tab window\\}}}\box0\par{}\vspace{10pt plus 2pt minus 1pt}\vspace{0pt}\indent There is a small menu to the left of the window:\\\par{}\vspace{6pt}\vspace{10pt plus 2pt minus 1pt}\setbox0=\vbox{\protect\centering \leavevmode
\vspace*{0pt}{\XeTeXpicfile "../images/en/tabProject_contextmenu_en.png" scaled 750}\\[0pt]\protect\hypertarget{f-NeedsALabel-.xlingpaper.1..styledPaper.1..lingPaper.1..part.1..chapter.2..section1.1..figure.3.}{}\XLingPaperaddtocontents{f-NeedsALabel-.xlingpaper.1..styledPaper.1..lingPaper.1..part.1..chapter.2..section1.1..figure.3.}\textit{{Figure }}\textit{{2.3}}\textit{{ Project Tab context menu\\}}}\box0\par{}\vspace{10pt plus 2pt minus 1pt}\vspace{0pt}{\parskip .5pt plus 1pt minus 1pt

\vspace{\baselineskip}

{\setlength{\XLingPapertempdim}{\XLingPaperbulletlistitemwidth+6em}\leftskip\XLingPapertempdim\relax
\interlinepenalty10000
\XLingPaperlistitem{6em}{\XLingPaperbulletlistitemwidth}{•}{{\textbf{About This Project}} \\ {\textit{This is where you fill out the information pertaining to your project such as a Title, Description, Location, etc}}}\vspace{3pt}}
{\setlength{\XLingPapertempdim}{\XLingPaperbulletlistitemwidth+6em}\leftskip\XLingPapertempdim\relax
\interlinepenalty10000
\XLingPaperlistitem{6em}{\XLingPaperbulletlistitemwidth}{•}{{\textbf{Access Protocol }} \\ {\textit{This is where you set the access protocol used by this project. This determines who has permission to get or see your archived data.}}}\vspace{3pt}}
{\setlength{\XLingPapertempdim}{\XLingPaperbulletlistitemwidth+6em}\leftskip\XLingPapertempdim\relax
\interlinepenalty10000
\XLingPaperlistitem{6em}{\XLingPaperbulletlistitemwidth}{•}{{\textbf{Description Documents}}\\ {\textit{This is where you add files pertaining to the project and the whole body of work for a language.}}}\vspace{3pt}}
{\setlength{\XLingPapertempdim}{\XLingPaperbulletlistitemwidth+6em}\leftskip\XLingPapertempdim\relax
\interlinepenalty10000
\XLingPaperlistitem{6em}{\XLingPaperbulletlistitemwidth}{•}{{\textbf{Other Documents}} \\{\textit{This is where you add documents that don't seem to fit anywhere else, like how the project was funded.}}}\vspace{3pt}}
{\setlength{\XLingPapertempdim}{\XLingPaperbulletlistitemwidth+6em}\leftskip\XLingPapertempdim\relax
\interlinepenalty10000
\XLingPaperlistitem{6em}{\XLingPaperbulletlistitemwidth}{•}{{\textbf{Progress}} \\ {\textit{This is where you can see the progress being made for the open project. You can even Copy, Save or Print the information found here.}}}}
\vspace{\baselineskip}
}{\XLingPaperneedspace{3\baselineskip}
\noindent\rule{\textwidth}{1pt}
{}\penalty10000\vspace{3pt}\XLingPaperneedspace{3\baselineskip}\noindent
\fontsize{12}{14.399999999999999}\selectfont \textbf{{\noindent
\raisebox{\baselineskip}[0pt]{\pdfbookmark[2]{{2.2 } Session Tab}{sSessionTab}}\raisebox{\baselineskip}[0pt]{\protect\hypertarget{sSessionTab}{}}{2.2 }Session Tab}}\markright{Session Tab}
\XLingPaperaddtocontents{sSessionTab}}\par{}
\penalty10000\vspace{10pt}\penalty10000\vspace{0pt}\indent The Sessions Tab is where you will find the information pertaining to each session as well as what was done in each session. The Session Tab window:\\\par{}\vspace{6pt}\vspace{10pt plus 2pt minus 1pt}\setbox0=\vbox{\protect\centering \leavevmode
\vspace*{0pt}{\XeTeXpicfile "../images/en/TAB_Sessions_withExamples_en.png" scaled 750}\\[0pt]\protect\hypertarget{f-NeedsALabel-.xlingpaper.1..styledPaper.1..lingPaper.1..part.1..chapter.2..section1.2..figure.1.}{}\XLingPaperaddtocontents{f-NeedsALabel-.xlingpaper.1..styledPaper.1..lingPaper.1..part.1..chapter.2..section1.2..figure.1.}\textit{{Figure }}\textit{{2.4}}\textit{{ Example Sessions\\}}}\box0\par{}\vspace{10pt plus 2pt minus 1pt}\vspace{0pt}\indent The left side of the window is where you will see your sessions. There are columns to show the sessions ID, Title, Stages, and Status. There are other optional columns you can turn on (Date, Genre, Location) by clicking \vspace*{0pt}{\XeTeXpicfile "../images/en/TAB_Sessions_ADDCOLUMNS.png" scaled 750} and selecting which ones you want to view.\par{}\vspace{6pt}\vspace{0pt}\indent The right upper window shows what files make up that session. The bottom right window displays information and content to be filled in depending on what you have selected in the upper right window. If you wish to add some more documentation or add the audio file for this session, click \vspace*{0pt}{\XeTeXpicfile "../images/en/AddFilesButton_small_en.png" scaled 750} and browse to the desired file.\par{}{\XLingPaperneedspace{3\baselineskip}
\noindent\rule{\textwidth}{1pt}
{}\penalty10000\vspace{3pt}\XLingPaperneedspace{3\baselineskip}\noindent
\fontsize{12}{14.399999999999999}\selectfont \textbf{{\noindent
\raisebox{\baselineskip}[0pt]{\pdfbookmark[2]{{2.3 } People Tab}{sPeopleTab}}\raisebox{\baselineskip}[0pt]{\protect\hypertarget{sPeopleTab}{}}{2.3 }People Tab}}\markright{People Tab}
\XLingPaperaddtocontents{sPeopleTab}}\par{}
\penalty10000\vspace{10pt}\penalty10000\vspace{0pt}\indent The People tab is where you keep track of all your participants. Here is where you will document their biological information, contributions, and notes:\\\par{}\vspace{6pt}\vspace{10pt plus 2pt minus 1pt}\setbox0=\vbox{\protect\raggedright\leavevmode
\vspace*{0pt}{\XeTeXpicfile "../images/en/TAB_People_withExample_en.png" scaled 750}\\[0pt]\protect\hypertarget{f-NeedsALabel-.xlingpaper.1..styledPaper.1..lingPaper.1..part.1..chapter.2..section1.3..figure.1.}{}\XLingPaperaddtocontents{f-NeedsALabel-.xlingpaper.1..styledPaper.1..lingPaper.1..part.1..chapter.2..section1.3..figure.1.}\textit{{Figure }}\textit{{2.5}}\textit{{ People Tab with example\\}}}\box0\par{}\vspace{10pt plus 2pt minus 1pt}\vspace{0pt}\indent Also you can add any files you need here also by clicking \vspace*{0pt}{\XeTeXpicfile "../images/en/AddFilesButton_small_en.png" scaled 750} and browsing to the desired files.  \\\par{}\vspace{6pt}\vspace{0pt}\clearpage
\thispagestyle{bodyfirstpage}\markboth{Menus}{Menus}
\XLingPaperaddtocontents{sMenus}{\XLingPaperneedspace{3\baselineskip}\noindent
\fontsize{18}{21.599999999999998}\selectfont \textbf{{\centering
\raisebox{\baselineskip}[0pt]{\protect\hypertarget{sMenus}{}}\raisebox{\baselineskip}[0pt]{\pdfbookmark[1]{3 Menus}{sMenus}}3\\}}}\par{}
\vspace{10.8pt}{\XLingPaperneedspace{3\baselineskip}\noindent
\fontsize{18}{21.599999999999998}\selectfont \textbf{{\centering
Menus\\}}}\par{}
\vspace{21.6pt}\vspace{0pt}\vspace{6pt}{\XLingPaperneedspace{3\baselineskip}
\noindent\rule{\textwidth}{1pt}
{}\penalty10000\vspace{3pt}\XLingPaperneedspace{3\baselineskip}\noindent
\fontsize{12}{14.399999999999999}\selectfont \textbf{{\noindent
\raisebox{\baselineskip}[0pt]{\pdfbookmark[2]{{3.1 } Menu: Project}{sMenuProject}}\raisebox{\baselineskip}[0pt]{\protect\hypertarget{sMenuProject}{}}{3.1 }Menu: Project}}\markright{Menu: Project}
\XLingPaperaddtocontents{sMenuProject}}\par{}
\penalty10000\vspace{10pt}\penalty10000\vspace{0pt}\indent When you clique on {\textbf{Project}} you will see:\par{}\vspace{6pt}\vspace{10pt plus 2pt minus 1pt}\setbox0=\vbox{\protect\centering \leavevmode
\vspace*{0pt}{\XeTeXpicfile "../images/en/mnuProject_en.png" scaled 750}\\[0pt]\protect\hypertarget{fProjectMenu}{}\XLingPaperaddtocontents{fProjectMenu}\textit{{Figure }}\textit{{3.1}}\textit{{ Project Menu\\}}}\box0\par{}\vspace{10pt plus 2pt minus 1pt}
\begin{mdframed}
[backgroundcolor=FTColorA,skipabove=3pt,skipbelow=3pt,innermargin=2cm,outermargin=2cm,innertopmargin=.03in,innerbottommargin=.03in,innerleftmargin=.125in,innerrightmargin=.125in,align=left]\vspace{0pt}\indent Note:\par{}{\parskip .5pt plus 1pt minus 1pt

\vspace{\baselineskip}

{\setlength{\XLingPapertempdim}{\XLingPaperbulletlistitemwidth+6em}\leftskip\XLingPapertempdim\relax
\interlinepenalty10000
\XLingPaperlistitem{6em}{\XLingPaperbulletlistitemwidth}{•}{As you can see there is no Save option. This is because SayMore saves for you.}\vspace{3pt}}
{\setlength{\XLingPapertempdim}{\XLingPaperbulletlistitemwidth+6em}\leftskip\XLingPapertempdim\relax
\interlinepenalty10000
\XLingPaperlistitem{6em}{\XLingPaperbulletlistitemwidth}{•}{There is no Delete command that deletes a project. Instead, you can delete the project folder in your Windows Explorer (File Explorer).}}
\vspace{\baselineskip}
}\end{mdframed}
{\XLingPaperneedspace{3\baselineskip}
\noindent\rule{\textwidth}{.4pt}
{}\penalty10000\vspace{3pt}\XLingPaperneedspace{3\baselineskip}\noindent
\fontsize{10}{12}\selectfont \textbf{{\noindent
\raisebox{\baselineskip}[0pt]{\pdfbookmark[3]{{3.1.1 } Quit}{sQuitSaymore}}\raisebox{\baselineskip}[0pt]{\protect\hypertarget{sQuitSaymore}{}}{3.1.1 }Quit}}\markright{Quit}
\XLingPaperaddtocontents{sQuitSaymore}}\par{}
\penalty10000\vspace{10pt}\penalty10000\vspace{0pt}\indent Closes the application.\par{}{\XLingPaperneedspace{3\baselineskip}
\noindent\rule{\textwidth}{1pt}
{}\penalty10000\vspace{3pt}\XLingPaperneedspace{3\baselineskip}\noindent
\fontsize{12}{14.399999999999999}\selectfont \textbf{{\noindent
\raisebox{\baselineskip}[0pt]{\pdfbookmark[2]{{3.2 } Menu: Session}{sMenuSession}}\raisebox{\baselineskip}[0pt]{\protect\hypertarget{sMenuSession}{}}{3.2 }Menu: Session}}\markright{Menu: Session}
\XLingPaperaddtocontents{sMenuSession}}\par{}
\penalty10000\vspace{10pt}\penalty10000\vspace{0pt}\indent When you clique on the Sessions tab you have a new window displayed and also the Menu option Session is now available.\\clicking on Menu Session you will see:\par{}\vspace{6pt}\vspace{10pt plus 2pt minus 1pt}\setbox0=\vbox{\protect\centering \leavevmode
\vspace*{0pt}{\XeTeXpicfile "../images/en/mnuSessions.png" scaled 750}\\[0pt]\protect\hypertarget{fsessionmenu}{}\XLingPaperaddtocontents{fsessionmenu}\textit{{Figure }}\textit{{3.2}}\textit{{ Menu Session\\}}}\box0\par{}\vspace{10pt plus 2pt minus 1pt}\vspace{0pt}{\XLingPaperneedspace{3\baselineskip}
\noindent\rule{\textwidth}{1pt}
{}\penalty10000\vspace{3pt}\XLingPaperneedspace{3\baselineskip}\noindent
\fontsize{12}{14.399999999999999}\selectfont \textbf{{\noindent
\raisebox{\baselineskip}[0pt]{\pdfbookmark[2]{{3.3 } Menu: Person}{sMenuPerson}}\raisebox{\baselineskip}[0pt]{\protect\hypertarget{sMenuPerson}{}}{3.3 }Menu: Person}}\markright{Menu: Person}
\XLingPaperaddtocontents{sMenuPerson}}\par{}
\penalty10000\vspace{10pt}\penalty10000\vspace{0pt}\indent When you clique the {\textbf{People}} Tab you can then select the Menu {\textbf{Person}}.\par{}\vspace{6pt}\vspace{10pt plus 2pt minus 1pt}\setbox0=\vbox{\protect\centering \leavevmode
\vspace*{0pt}{\XeTeXpicfile "../images/en/mnuPerson_en.png" scaled 750}\\[0pt]\protect\hypertarget{fPersonMenu}{}\XLingPaperaddtocontents{fPersonMenu}\textit{{Figure }}\textit{{3.3}}\textit{{ Menu Person\\}}}\box0\par{}\vspace{10pt plus 2pt minus 1pt}{\XLingPaperneedspace{3\baselineskip}
\noindent\rule{\textwidth}{1pt}
{}\penalty10000\vspace{3pt}\XLingPaperneedspace{3\baselineskip}\noindent
\fontsize{12}{14.399999999999999}\selectfont \textbf{{\noindent
\raisebox{\baselineskip}[0pt]{\pdfbookmark[2]{{3.4 } Menu: Help}{sMenuHelp}}\raisebox{\baselineskip}[0pt]{\protect\hypertarget{sMenuHelp}{}}{3.4 }Menu: Help}}\markright{Menu: Help}
\XLingPaperaddtocontents{sMenuHelp}}\par{}
\penalty10000\vspace{10pt}\penalty10000\vspace{0pt}\indent This is the place to find help while using SayMore. When you clique on menu Help you will see:\par{}\vspace{6pt}\vspace{10pt plus 2pt minus 1pt}\setbox0=\vbox{\protect\centering \leavevmode
\vspace*{0pt}{\XeTeXpicfile "../images/en/mnuHelp_en.png" scaled 750}\\[0pt]\protect\hypertarget{f-NeedsALabel-.xlingpaper.1..styledPaper.1..lingPaper.1..part.1..chapter.2..section1.4..figure.1.}{}\XLingPaperaddtocontents{f-NeedsALabel-.xlingpaper.1..styledPaper.1..lingPaper.1..part.1..chapter.2..section1.4..figure.1.}\textit{{Figure }}\textit{{3.4}}\textit{{ Menu Help\\}}}\box0\par{}\vspace{10pt plus 2pt minus 1pt}\vspace{0pt}\vspace{6pt}{\XLingPaperneedspace{3\baselineskip}
\noindent\rule{\textwidth}{.4pt}
{}\penalty10000\vspace{3pt}\XLingPaperneedspace{3\baselineskip}\noindent
\fontsize{10}{12}\selectfont \textbf{{\noindent
\raisebox{\baselineskip}[0pt]{\pdfbookmark[3]{{3.4.1 } Help...}{sHelp}}\raisebox{\baselineskip}[0pt]{\protect\hypertarget{sHelp}{}}{3.4.1 }Help...}}\markright{Help...}
\XLingPaperaddtocontents{sHelp}}\par{}
\penalty10000\vspace{10pt}\penalty10000\vspace{0pt}\indent When you clique {\textbf{Help...}} it will open a new window:\par{}\vspace{6pt}\vspace{10pt plus 2pt minus 1pt}\setbox0=\vbox{\protect\raggedright\leavevmode
\vspace*{0pt}{\XeTeXpicfile "../images/en/SayMore_HelpWindow_en.png" scaled 750}\\[0pt]\protect\hypertarget{f-NeedsALabel-.xlingpaper.1..styledPaper.1..lingPaper.1..part.1..chapter.2..section1.4..section2.2..figure.1.}{}\XLingPaperaddtocontents{f-NeedsALabel-.xlingpaper.1..styledPaper.1..lingPaper.1..part.1..chapter.2..section1.4..section2.2..figure.1.}\textit{{Figure }}\textit{{3.5}}\textit{{ SayMore Help Window\\}}}\box0\par{}\vspace{10pt plus 2pt minus 1pt}\vspace{0pt}\indent From this window you can search by using the Content Tree, Index (Search by Keyword), and Search (search for any word).\par{}{\XLingPaperneedspace{3\baselineskip}
\noindent\rule{\textwidth}{.4pt}
{}\penalty10000\vspace{3pt}\XLingPaperneedspace{3\baselineskip}\noindent
\fontsize{10}{12}\selectfont \textbf{{\noindent
\raisebox{\baselineskip}[0pt]{\pdfbookmark[3]{{3.4.2 } About...}{sAbout}}\raisebox{\baselineskip}[0pt]{\protect\hypertarget{sAbout}{}}{3.4.2 }About...}}\markright{About...}
\XLingPaperaddtocontents{sAbout}}\par{}
\penalty10000\vspace{10pt}\penalty10000\vspace{0pt}\indent When you clique About... a window opens detailing who was behind making SayMore and thanks to those who helped make SayMore Possible. It also has links to the Open Source Components/Libraries used in SayMore.\par{}\clearpage
{\clearpage
\XLingPaperneedspace{3\baselineskip}\noindent
\fontsize{18}{21.599999999999998}\selectfont \textbf{{\centering
\thispagestyle{empty}\raisebox{\baselineskip}[0pt]{\pdfbookmark[1]{Part II Project, Session, and User Management}{pGestion}}\raisebox{\baselineskip}[0pt]{\protect\hypertarget{pGestion}{}}Part II\\}}}\par{}
\vspace{10.8pt}{\XLingPaperneedspace{3\baselineskip}\noindent
\fontsize{18}{21.599999999999998}\selectfont \textbf{{\centering
Project, Session, and User Management\\}}}\par{}
\vspace{21.6pt}\vspace{0pt}\vspace{6pt}\clearpage
\thispagestyle{bodyfirstpage}\markboth{Create or Open a Project}{Create or Open a Project}
\XLingPaperaddtocontents{cCreateProjectOver}{\XLingPaperneedspace{3\baselineskip}\noindent
\fontsize{18}{21.599999999999998}\selectfont \textbf{{\centering
\raisebox{\baselineskip}[0pt]{\protect\hypertarget{cCreateProjectOver}{}}\raisebox{\baselineskip}[0pt]{\pdfbookmark[1]{4 Create or Open a Project}{cCreateProjectOver}}4\\}}}\par{}
\vspace{10.8pt}{\XLingPaperneedspace{3\baselineskip}\noindent
\fontsize{18}{21.599999999999998}\selectfont \textbf{{\centering
Create or Open a Project\\}}}\par{}
\vspace{21.6pt}\vspace{0pt}\indent Before working in SayMore, you must create a project. You should create a separate project for each language you are analysing, but you do not need to create a new project for every file you analyse.\par{}\vspace{6pt}{\XLingPaperneedspace{3\baselineskip}
\noindent\rule{\textwidth}{1pt}
{}\penalty10000\vspace{3pt}\XLingPaperneedspace{3\baselineskip}\noindent
\fontsize{12}{14.399999999999999}\selectfont \textbf{{\noindent
\raisebox{\baselineskip}[0pt]{\pdfbookmark[2]{{4.1 } Create a Project}{sCreateProject}}\raisebox{\baselineskip}[0pt]{\protect\hypertarget{sCreateProject}{}}{4.1 }Create a Project}}\markright{Create a Project}
\XLingPaperaddtocontents{sCreateProject}}\par{}
\penalty10000\vspace{10pt}\penalty10000\vspace{0pt}\indent When you clique on {\textbf{Open / Create Project}} you will see:\par{}\vspace{6pt}\vspace{10pt plus 2pt minus 1pt}\setbox0=\vbox{\protect\raggedright\leavevmode
\vspace*{0pt}{\XeTeXpicfile "../images/en/Open_Create_en.png" scaled 750}\\[0pt]\protect\hypertarget{f-NeedsALabel-.xlingpaper.1..styledPaper.1..lingPaper.1..part.1..chapter.2..section1.1..section2.1..figure.1.}{}\XLingPaperaddtocontents{f-NeedsALabel-.xlingpaper.1..styledPaper.1..lingPaper.1..part.1..chapter.2..section1.1..section2.1..figure.1.}\textit{{Figure }}\textit{{4.1}}\textit{{ Open / Create Project\\}}}\box0\par{}\vspace{10pt plus 2pt minus 1pt}\vspace{0pt}\indent When clicking on {\textbf{Create new, Blank project...}} you will see a dialogue box asking you for a name. Fill in the blank and press {\textbf{Okay}}. Below the name you have chosen you will see the path to where the project will be saved.\par{}\vspace{6pt}\vspace{10pt plus 2pt minus 1pt}\setbox0=\vbox{\protect\centering \leavevmode
\vspace*{0pt}{\XeTeXpicfile "../images/en/NewProjectName_en.png" scaled 750}\\[0pt]\protect\hypertarget{fNewPro}{}\XLingPaperaddtocontents{fNewPro}\textit{{Figure }}\textit{{4.2}}\textit{{ New Project Name\\}}}\box0\par{}\vspace{10pt plus 2pt minus 1pt}\vspace{0pt}\indent Once you click okay, SayMore will open with a new blank project. To continue go to:\\ \hyperlink{cProjectMeta}{5} \hyperlink{cProjectMeta}{Managing a Project}\par{}{\XLingPaperneedspace{3\baselineskip}
\noindent\rule{\textwidth}{1pt}
{}\penalty10000\vspace{3pt}\XLingPaperneedspace{3\baselineskip}\noindent
\fontsize{12}{14.399999999999999}\selectfont \textbf{{\noindent
\raisebox{\baselineskip}[0pt]{\pdfbookmark[2]{{4.2 } Open an Existing Project}{sOpenProject}}\raisebox{\baselineskip}[0pt]{\protect\hypertarget{sOpenProject}{}}{4.2 }Open an Existing Project}}\markright{Open an Existing Project}
\XLingPaperaddtocontents{sOpenProject}}\par{}
\penalty10000\vspace{10pt}\penalty10000\vspace{0pt}\indent When you clique on {\textbf{Open / Create Project}} you will see:\par{}\vspace{6pt}\vspace{10pt plus 2pt minus 1pt}\setbox0=\vbox{\protect\raggedright\leavevmode
\vspace*{0pt}{\XeTeXpicfile "../images/en/Open_Create_en.png" scaled 750}\\[0pt]\protect\hypertarget{fOpenExisting}{}\XLingPaperaddtocontents{fOpenExisting}\textit{{Figure }}\textit{{4.3}}\textit{{ Open / Create Project\\}}}\box0\par{}\vspace{10pt plus 2pt minus 1pt}\vspace{0pt}\indent Click on the name of your project under 'Open' to open it.\\{\textit{The project will open.}}\par{}{\XLingPaperneedspace{3\baselineskip}
\noindent\rule{\textwidth}{1pt}
{}\penalty10000\vspace{3pt}\XLingPaperneedspace{3\baselineskip}\noindent
\fontsize{12}{14.399999999999999}\selectfont \textbf{{\noindent
\raisebox{\baselineskip}[0pt]{\pdfbookmark[2]{{4.3 } Open a different Project}{sOpenOtherProject}}\raisebox{\baselineskip}[0pt]{\protect\hypertarget{sOpenOtherProject}{}}{4.3 }Open a different Project}}\markright{Open a different Project}
\XLingPaperaddtocontents{sOpenOtherProject}}\par{}
\penalty10000\vspace{10pt}\penalty10000\vspace{0pt}\indent If you want to open another project from a USB or other storage device, follow these steps:\par{}\vspace{6pt}\vspace{0pt}\indent From the {\textbf{Project}} menu, clique on {\textbf{Open / Create Project}} you will see:\par{}\vspace{6pt}\vspace{10pt plus 2pt minus 1pt}\setbox0=\vbox{\protect\raggedright\leavevmode
\vspace*{0pt}{\XeTeXpicfile "../images/en/Open_Create_en.png" scaled 750}\\[0pt]\protect\hypertarget{fOpenOther}{}\XLingPaperaddtocontents{fOpenOther}\textit{{Figure }}\textit{{4.4}}\textit{{ Open / Create Project\\}}}\box0\par{}\vspace{10pt plus 2pt minus 1pt}\vspace{0pt}\indent Click on 'browse for project' the name of your project to open it.\\ \\\par{}\clearpage
\thispagestyle{bodyfirstpage}\markboth{Managing a Project}{Managing a Project}
\XLingPaperaddtocontents{cProjectMeta}{\XLingPaperneedspace{3\baselineskip}\noindent
\fontsize{18}{21.599999999999998}\selectfont \textbf{{\centering
\raisebox{\baselineskip}[0pt]{\protect\hypertarget{cProjectMeta}{}}\raisebox{\baselineskip}[0pt]{\pdfbookmark[1]{5 Managing a Project}{cProjectMeta}}5\\}}}\par{}
\vspace{10.8pt}{\XLingPaperneedspace{3\baselineskip}\noindent
\fontsize{18}{21.599999999999998}\selectfont \textbf{{\centering
Managing a Project\\}}}\par{}
\vspace{21.6pt}\vspace{0pt}\indent When you create a new project or even opening an existing project it is important to make sure the Project Tab elements are filled out as much as possible. This will also be the first tab open when you create a new project. The Project Tab is designed to collect metadata that concerns the entire project. These metadata and documents are critical for the discoverability and use of the archive.\par{}\vspace{6pt}{\XLingPaperneedspace{3\baselineskip}
\noindent\rule{\textwidth}{1pt}
{}\penalty10000\vspace{3pt}\XLingPaperneedspace{3\baselineskip}\noindent
\fontsize{12}{14.399999999999999}\selectfont \textbf{{\noindent
\raisebox{\baselineskip}[0pt]{\pdfbookmark[2]{{5.1 } About This Project}{s}}\raisebox{\baselineskip}[0pt]{\protect\hypertarget{s}{}}{5.1 }About This Project}}\markright{About This Project}
\XLingPaperaddtocontents{s}}\par{}
\penalty10000\vspace{10pt}\penalty10000\vspace{0pt}\indent Below describes what should be placed in the fields in this window:\par{}{\parskip .5pt plus 1pt minus 1pt

\vspace{\baselineskip}

{\setlength{\XLingPapertempdim}{\XLingPaperbulletlistitemwidth+6em}\leftskip\XLingPapertempdim\relax
\interlinepenalty10000
\XLingPaperlistitem{6em}{\XLingPaperbulletlistitemwidth}{•}{{\textbf{Title}} - The full title for the project. Initially, the project name appears here. You can edit it, such as to make it more formal (longer). This will not change the project name, or any file or folder names.}\vspace{3pt}}
{\setlength{\XLingPapertempdim}{\XLingPaperbulletlistitemwidth+6em}\leftskip\XLingPapertempdim\relax
\interlinepenalty10000
\XLingPaperlistitem{6em}{\XLingPaperbulletlistitemwidth}{•}{{\textbf{Description}} - Information about the scope and goals of the project.}\vspace{3pt}}
{\setlength{\XLingPapertempdim}{\XLingPaperbulletlistitemwidth+6em}\leftskip\XLingPapertempdim\relax
\interlinepenalty10000
\XLingPaperlistitem{6em}{\XLingPaperbulletlistitemwidth}{•}{{\textbf{Vernacular}} - The ISO-639 name of the language used in the sessions.}\vspace{3pt}}
{\setlength{\XLingPapertempdim}{\XLingPaperbulletlistitemwidth+6em}\leftskip\XLingPapertempdim\relax
\interlinepenalty10000
\XLingPaperlistitem{6em}{\XLingPaperbulletlistitemwidth}{•}{{\textbf{Location/Address}} - The location, such as the village or town, or address where the session was recorded or originated.}\vspace{3pt}}
{\setlength{\XLingPapertempdim}{\XLingPaperbulletlistitemwidth+6em}\leftskip\XLingPapertempdim\relax
\interlinepenalty10000
\XLingPaperlistitem{6em}{\XLingPaperbulletlistitemwidth}{•}{{\textbf{Region}} - The region or sub-region that has the location or address.}\vspace{3pt}}
{\setlength{\XLingPapertempdim}{\XLingPaperbulletlistitemwidth+6em}\leftskip\XLingPapertempdim\relax
\interlinepenalty10000
\XLingPaperlistitem{6em}{\XLingPaperbulletlistitemwidth}{•}{{\textbf{Country}} - A list which is used to identify the country where the location/address and region are located.}\vspace{3pt}}
{\setlength{\XLingPapertempdim}{\XLingPaperbulletlistitemwidth+6em}\leftskip\XLingPapertempdim\relax
\interlinepenalty10000
\XLingPaperlistitem{6em}{\XLingPaperbulletlistitemwidth}{•}{{\textbf{Continent}} - A list which is used to identify the continent that has the country.}\vspace{3pt}}
{\setlength{\XLingPapertempdim}{\XLingPaperbulletlistitemwidth+6em}\leftskip\XLingPapertempdim\relax
\interlinepenalty10000
\XLingPaperlistitem{6em}{\XLingPaperbulletlistitemwidth}{•}{{\textbf{Contact Person}} - Contact information about the person or institution responsible for the project.}\vspace{3pt}}
{\setlength{\XLingPapertempdim}{\XLingPaperbulletlistitemwidth+6em}\leftskip\XLingPapertempdim\relax
\interlinepenalty10000
\XLingPaperlistitem{6em}{\XLingPaperbulletlistitemwidth}{•}{{\textbf{Funding Project Title}} - A title that matches associated project funding documentation.}\vspace{3pt}}
{\setlength{\XLingPapertempdim}{\XLingPaperbulletlistitemwidth+6em}\leftskip\XLingPapertempdim\relax
\interlinepenalty10000
\XLingPaperlistitem{6em}{\XLingPaperbulletlistitemwidth}{•}{{\textbf{Date Available}} - The date of that this project was or will be archived.}\vspace{3pt}}
{\setlength{\XLingPapertempdim}{\XLingPaperbulletlistitemwidth+6em}\leftskip\XLingPapertempdim\relax
\interlinepenalty10000
\XLingPaperlistitem{6em}{\XLingPaperbulletlistitemwidth}{•}{{\textbf{Rights Holder}} - The copyright symbol, year, organization name, and statement regarding copyright.}\vspace{3pt}}
{\setlength{\XLingPapertempdim}{\XLingPaperbulletlistitemwidth+6em}\leftskip\XLingPapertempdim\relax
\interlinepenalty10000
\XLingPaperlistitem{6em}{\XLingPaperbulletlistitemwidth}{•}{{\textbf{Depositor}} - The person responsible for depositing the resource in an archive.}}
\vspace{\baselineskip}
}\vspace{0pt}{\XLingPaperneedspace{3\baselineskip}
\noindent\rule{\textwidth}{1pt}
{}\penalty10000\vspace{3pt}\XLingPaperneedspace{3\baselineskip}\noindent
\fontsize{12}{14.399999999999999}\selectfont \textbf{{\noindent
\raisebox{\baselineskip}[0pt]{\pdfbookmark[2]{{5.2 } Access Protocol}{sAccessProt}}\raisebox{\baselineskip}[0pt]{\protect\hypertarget{sAccessProt}{}}{5.2 }Access Protocol}}\markright{Access Protocol}
\XLingPaperaddtocontents{sAccessProt}}\par{}
\penalty10000\vspace{10pt}\penalty10000\vspace{0pt}\indent Access is how you determine who has permission to get or see your archived data. Access Protocol is used to determine what access to assign to an item in archives. Here is where you choose the access protocol for the project.\par{}{\parskip .5pt plus 1pt minus 1pt

\vspace{\baselineskip}

{\setlength{\XLingPapertempdim}{\XLingPaperbulletlistitemwidth+6em}\leftskip\XLingPapertempdim\relax
\interlinepenalty10000
\XLingPaperlistitem{6em}{\XLingPaperbulletlistitemwidth}{•}{Unless your organization instructs you differently:\\SIL members or others with permission to use REAP/RAMP, choose:}{\setlength{\XLingPaperlistitemindent}{\XLingPaperbulletlistitemwidth + 6em}
{\setlength{\XLingPapertempdim}{\XLingPaperbulletlistitemwidth+\XLingPaperlistitemindent}\leftskip\XLingPapertempdim\relax
\interlinepenalty10000
\XLingPaperlistitem{\XLingPaperlistitemindent}{\XLingPaperbulletlistitemwidth}{•}{REAP}}}\vspace{3pt}}
{\setlength{\XLingPapertempdim}{\XLingPaperbulletlistitemwidth+6em}\leftskip\XLingPapertempdim\relax
\interlinepenalty10000
\XLingPaperlistitem{6em}{\XLingPaperbulletlistitemwidth}{•}{All others, choose one of the following:}{\setlength{\XLingPaperlistitemindent}{\XLingPaperbulletlistitemwidth + 6em}
{\setlength{\XLingPapertempdim}{\XLingPaperbulletlistitemwidth+\XLingPaperlistitemindent}\leftskip\XLingPapertempdim\relax
\interlinepenalty10000
\XLingPaperlistitem{\XLingPaperlistitemindent}{\XLingPaperbulletlistitemwidth}{•}{AILCA}\vspace{3pt}}
{\setlength{\XLingPapertempdim}{\XLingPaperbulletlistitemwidth+\XLingPaperlistitemindent}\leftskip\XLingPapertempdim\relax
\interlinepenalty10000
\XLingPaperlistitem{\XLingPaperlistitemindent}{\XLingPaperbulletlistitemwidth}{•}{AILLA}\vspace{3pt}}
{\setlength{\XLingPapertempdim}{\XLingPaperbulletlistitemwidth+\XLingPaperlistitemindent}\leftskip\XLingPapertempdim\relax
\interlinepenalty10000
\XLingPaperlistitem{\XLingPaperlistitemindent}{\XLingPaperbulletlistitemwidth}{•}{ANLA}\vspace{3pt}}
{\setlength{\XLingPapertempdim}{\XLingPaperbulletlistitemwidth+\XLingPaperlistitemindent}\leftskip\XLingPapertempdim\relax
\interlinepenalty10000
\XLingPaperlistitem{\XLingPaperlistitemindent}{\XLingPaperbulletlistitemwidth}{•}{ELAR}\vspace{3pt}}
{\setlength{\XLingPapertempdim}{\XLingPaperbulletlistitemwidth+\XLingPaperlistitemindent}\leftskip\XLingPapertempdim\relax
\interlinepenalty10000
\XLingPaperlistitem{\XLingPaperlistitemindent}{\XLingPaperbulletlistitemwidth}{•}{TLA}\vspace{3pt}}
{\setlength{\XLingPapertempdim}{\XLingPaperbulletlistitemwidth+\XLingPaperlistitemindent}\leftskip\XLingPapertempdim\relax
\interlinepenalty10000
\XLingPaperlistitem{\XLingPaperlistitemindent}{\XLingPaperbulletlistitemwidth}{•}{Custom}}}}
\vspace{\baselineskip}
}\vspace{0pt}\indent If you choose 'Custom' you are presented with an empty box that you can type in custom access choices separated by commas. This allows you to set different recorded sessions to different access protocols.\par{}{\XLingPaperneedspace{3\baselineskip}
\noindent\rule{\textwidth}{1pt}
{}\penalty10000\vspace{3pt}\XLingPaperneedspace{3\baselineskip}\noindent
\fontsize{12}{14.399999999999999}\selectfont \textbf{{\noindent
\raisebox{\baselineskip}[0pt]{\pdfbookmark[2]{{5.3 } Project Description Documents}{s}}\raisebox{\baselineskip}[0pt]{\protect\hypertarget{s}{}}{5.3 }Project Description Documents}}\markright{Project Description Documents}
\XLingPaperaddtocontents{s}}\par{}
\penalty10000\vspace{10pt}\penalty10000\vspace{0pt}\indent This is where you add documents that describe the project and corpus. When the project is archived with IMDI the files here are exported to a special session named 'Project Descriptive Documents'.\par{}{\XLingPaperneedspace{3\baselineskip}
\noindent\rule{\textwidth}{1pt}
{}\penalty10000\vspace{3pt}\XLingPaperneedspace{3\baselineskip}\noindent
\fontsize{12}{14.399999999999999}\selectfont \textbf{{\noindent
\raisebox{\baselineskip}[0pt]{\pdfbookmark[2]{{5.4 } Other Documents}{sOtherDocs}}\raisebox{\baselineskip}[0pt]{\protect\hypertarget{sOtherDocs}{}}{5.4 }Other Documents}}\markright{Other Documents}
\XLingPaperaddtocontents{sOtherDocs}}\par{}
\penalty10000\vspace{10pt}\penalty10000\vspace{0pt}\indent This is where you add any other project-level documents that do not belong in Description Documents. When the project is archived with IMDI the files here are exported to a special session named 'Other Project Documents'.\par{}{\XLingPaperneedspace{3\baselineskip}
\noindent\rule{\textwidth}{1pt}
{}\penalty10000\vspace{3pt}\XLingPaperneedspace{3\baselineskip}\noindent
\fontsize{12}{14.399999999999999}\selectfont \textbf{{\noindent
\raisebox{\baselineskip}[0pt]{\pdfbookmark[2]{{5.5 } Project Progress}{s}}\raisebox{\baselineskip}[0pt]{\protect\hypertarget{s}{}}{5.5 }Project Progress}}\markright{Project Progress}
\XLingPaperaddtocontents{s}}\par{}
\penalty10000\vspace{10pt}\penalty10000\vspace{0pt}\indent This window is where you can track the progress of the projects:\par{}\vspace{6pt}\vspace{10pt plus 2pt minus 1pt}\setbox0=\vbox{\protect\raggedright\leavevmode
\vspace*{0pt}{\XeTeXpicfile "../images/en/Progress_en.png" scaled 750}\\[0pt]\protect\hypertarget{f-NeedsALabel-.xlingpaper.1..styledPaper.1..lingPaper.1..part.2..chapter.2..section1.5..figure.1.}{}\XLingPaperaddtocontents{f-NeedsALabel-.xlingpaper.1..styledPaper.1..lingPaper.1..part.2..chapter.2..section1.5..figure.1.}\textit{{Figure }}\textit{{5.1}}\textit{{ Progress\\}}}\box0\par{}\vspace{10pt plus 2pt minus 1pt}\vspace{0pt}\indent As you can see in the image above, this is a very useful graphic that can be copied, to be pasted into a report, saved as an individual html file, or printed.\par{}\vspace{6pt}\vspace{0pt}\clearpage
\thispagestyle{bodyfirstpage}\markboth{Managing Sessions}{Managing Sessions}
\XLingPaperaddtocontents{cSessions}{\XLingPaperneedspace{3\baselineskip}\noindent
\fontsize{18}{21.599999999999998}\selectfont \textbf{{\centering
\raisebox{\baselineskip}[0pt]{\protect\hypertarget{cSessions}{}}\raisebox{\baselineskip}[0pt]{\pdfbookmark[1]{6 Managing Sessions}{cSessions}}6\\}}}\par{}
\vspace{10.8pt}{\XLingPaperneedspace{3\baselineskip}\noindent
\fontsize{18}{21.599999999999998}\selectfont \textbf{{\centering
Managing Sessions\\}}}\par{}
\vspace{21.6pt}\vspace{0pt}\indent If you have just created a new project, the sessions tab will be empty. Otherwise you will see all of your previous sessions and what was recorded in each session.\par{}\vspace{6pt}{\XLingPaperneedspace{3\baselineskip}
\noindent\rule{\textwidth}{1pt}
{}\penalty10000\vspace{3pt}\XLingPaperneedspace{3\baselineskip}\noindent
\fontsize{12}{14.399999999999999}\selectfont \textbf{{\noindent
\raisebox{\baselineskip}[0pt]{\pdfbookmark[2]{{6.1 } Create a new Session}{sCreateSession}}\raisebox{\baselineskip}[0pt]{\protect\hypertarget{sCreateSession}{}}{6.1 }Create a new Session}}\markright{Create a new Session}
\XLingPaperaddtocontents{sCreateSession}}\par{}
\penalty10000\vspace{10pt}\penalty10000\vspace{0pt}\indent {\textbf{New}} Allows you to create a blank new Session. You then have to manually add all the files you wish to associate with the new session. This is done on the right hand side of the Sessions Tab window. To create a new Session you can use the menu Session drop down or on the bottom of the Session Tab window there are buttons:\par{}\vspace{6pt}\vspace{10pt plus 2pt minus 1pt}\setbox0=\vbox{\protect\raggedright\leavevmode
\vspace*{0pt}{\XeTeXpicfile "../images/en/TAB_Sessions_buttons_en.png" scaled 750}\\[0pt]\protect\hypertarget{f-NeedsALabel-.xlingpaper.1..styledPaper.1..lingPaper.1..part.2..chapter.3..section1.1..figure.1.}{}\XLingPaperaddtocontents{f-NeedsALabel-.xlingpaper.1..styledPaper.1..lingPaper.1..part.2..chapter.3..section1.1..figure.1.}\textit{{Figure }}\textit{{6.1}}\textit{{ Session window buttons\\}}}\box0\par{}\vspace{10pt plus 2pt minus 1pt}\vspace{0pt}\vspace{6pt}
\begin{mdframed}
[backgroundcolor=FTColorA,skipabove=3pt,skipbelow=3pt,innermargin=2cm,outermargin=2cm,innertopmargin=.03in,innerbottommargin=.03in,innerleftmargin=.125in,innerrightmargin=.125in,align=left]\vspace{0pt}\indent Note:\par{}{\parskip .5pt plus 1pt minus 1pt

\vspace{\baselineskip}

{\setlength{\XLingPapertempdim}{\XLingPaperbulletlistitemwidth+6em}\leftskip\XLingPapertempdim\relax
\interlinepenalty10000
\XLingPaperlistitem{6em}{\XLingPaperbulletlistitemwidth}{•}{{\textbf{Add Files...}} You can add files to any Session, not just a New session.}}
\vspace{\baselineskip}
}\end{mdframed}
{\XLingPaperneedspace{3\baselineskip}
\noindent\rule{\textwidth}{.4pt}
{}\penalty10000\vspace{3pt}\XLingPaperneedspace{3\baselineskip}\noindent
\fontsize{10}{12}\selectfont \textbf{{\noindent
\raisebox{\baselineskip}[0pt]{\pdfbookmark[3]{{6.1.1 } New From Device...}{sNewFromDevice}}\raisebox{\baselineskip}[0pt]{\protect\hypertarget{sNewFromDevice}{}}{6.1.1 }New From Device...}}\markright{New From Device...}
\XLingPaperaddtocontents{sNewFromDevice}}\par{}
\penalty10000\vspace{10pt}\penalty10000\vspace{0pt}\indent {\textbf{New From Device..}} Allows you to create a new Session and Opens:\par{}\vspace{6pt}\vspace{10pt plus 2pt minus 1pt}\setbox0=\vbox{\protect\raggedright\leavevmode
\vspace*{0pt}{\XeTeXpicfile "../images/en/NewSessionFromDeviceWindow_en.png" scaled 750}\\[0pt]\protect\hypertarget{fNewFromDevice}{}\XLingPaperaddtocontents{fNewFromDevice}\textit{{Figure }}\textit{{6.2}}\textit{{ New From Device Window\\}}}\box0\par{}\vspace{10pt plus 2pt minus 1pt}\vspace{0pt}\indent When you have selected a folder you will see:\par{}\vspace{6pt}\vspace{10pt plus 2pt minus 1pt}\setbox0=\vbox{\protect\raggedright\leavevmode
\vspace*{0pt}{\XeTeXpicfile "../images/en/NewSessionsFromDevice_Examples_en.png" scaled 750}\\[0pt]\protect\hypertarget{f-NeedsALabel-.xlingpaper.1..styledPaper.1..lingPaper.1..part.2..chapter.3..section1.1..section2.1..figure.2.}{}\XLingPaperaddtocontents{f-NeedsALabel-.xlingpaper.1..styledPaper.1..lingPaper.1..part.2..chapter.3..section1.1..section2.1..figure.2.}\textit{{Figure }}\textit{{6.3}}\textit{{ New Sessions From Device\\}}}\box0\par{}\vspace{10pt plus 2pt minus 1pt}\vspace{0pt}\indent Here you can select all or some of the Audio files. With the media box you can play the highlighted file to make sure it is the desired file. SayMore will remember the path to this file so the next time you click on {\textbf{New Sessions From Device}} you will see this folder still selected. Each Audio file selected will be made into its own session. As you can see in {\textit{Figure}} \hyperlink{f}{} there are three files selected thus there will be 3 new sessions created.\par{}\vspace{6pt}
\begin{mdframed}
[backgroundcolor=FTColorA,skipabove=3pt,skipbelow=3pt,innermargin=2cm,outermargin=2cm,innertopmargin=.03in,innerbottommargin=.03in,innerleftmargin=.125in,innerrightmargin=.125in,align=left]\vspace{0pt}\indent If SayMore cannot access the folder where files were located the last time you used this feature, a warning \vspace*{0pt}{\XeTeXpicfile "../images/en/WARNING_ICON.png" scaled 750} message appears. Simply browse to the desired folder again.\par{}\end{mdframed}
{\XLingPaperneedspace{3\baselineskip}
\noindent\rule{\textwidth}{.4pt}
{}\penalty10000\vspace{3pt}\XLingPaperneedspace{3\baselineskip}\noindent
\fontsize{10}{12}\selectfont \textbf{{\noindent
\raisebox{\baselineskip}[0pt]{\pdfbookmark[3]{{6.1.2 } New From Recording...}{sNewFromRecording}}\raisebox{\baselineskip}[0pt]{\protect\hypertarget{sNewFromRecording}{}}{6.1.2 }New From Recording...}}\markright{New From Recording...}
\XLingPaperaddtocontents{sNewFromRecording}}\par{}
\penalty10000\vspace{10pt}\penalty10000\vspace{0pt}\indent {\textbf{New From Recording...}} Allows you to use the Computer Microphone to record a session. If you have an external microphone connected it will record from there. Click the Record button and when finished click the Stop button. You can immediately hear the playback by pressing Play Back Recording. If you are happy with the recording click OK.\par{}\vspace{6pt}\vspace{10pt plus 2pt minus 1pt}\setbox0=\vbox{\protect\centering \leavevmode
\vspace*{0pt}{\XeTeXpicfile "../images/en/NewFromRecording_en.png" scaled 750}\\[0pt]\protect\hypertarget{fRecord}{}\XLingPaperaddtocontents{fRecord}\textit{{Figure }}\textit{{6.4}}\textit{{ New From Recording Window\\}}}\box0\par{}\vspace{10pt plus 2pt minus 1pt}\vspace{0pt}\vspace{6pt}
\begin{mdframed}
[backgroundcolor=FTColorA,skipabove=3pt,skipbelow=3pt,innermargin=2cm,outermargin=2cm,innertopmargin=.03in,innerbottommargin=.03in,innerleftmargin=.125in,innerrightmargin=.125in,align=left]\vspace{0pt}\indent From SayMore Developers:\\{\textit{\textbf{"We plead with you to avoid using your laptop's built-in microphone; if you are willing to give up usefulness for future phonetic research, an OK USB headset can be had for US\textdollar{}35 or less in many countries. Note that the very popular Zoom H2's can also be plugged in and used as a microphone (in a lower quality mode)."}}}\par{}\end{mdframed}
{\XLingPaperneedspace{3\baselineskip}
\noindent\rule{\textwidth}{1pt}
{}\penalty10000\vspace{3pt}\XLingPaperneedspace{3\baselineskip}\noindent
\fontsize{12}{14.399999999999999}\selectfont \textbf{{\noindent
\raisebox{\baselineskip}[0pt]{\pdfbookmark[2]{{6.2 } Delete Session...}{sDeleteSession}}\raisebox{\baselineskip}[0pt]{\protect\hypertarget{sDeleteSession}{}}{6.2 }Delete Session...}}\markright{Delete Session...}
\XLingPaperaddtocontents{sDeleteSession}}\par{}
\penalty10000\vspace{10pt}\penalty10000\vspace{0pt}\indent {\textbf{Delete Session...}} Allows you to delete a session you have selected in the {\textbf{Sessions tab}}\par{}{\XLingPaperneedspace{3\baselineskip}
\noindent\rule{\textwidth}{1pt}
{}\penalty10000\vspace{3pt}\XLingPaperneedspace{3\baselineskip}\noindent
\fontsize{12}{14.399999999999999}\selectfont \textbf{{\noindent
\raisebox{\baselineskip}[0pt]{\pdfbookmark[2]{{6.3 } Session Metadata}{sSessMeta1}}\raisebox{\baselineskip}[0pt]{\protect\hypertarget{sSessMeta1}{}}{6.3 }Session Metadata}}\markright{Session Metadata}
\XLingPaperaddtocontents{sSessMeta1}}\par{}
\penalty10000\vspace{10pt}\penalty10000\vspace{0pt}\indent When you are in the Sessions Tab, and a session is selected on the left, on the right you have a window showing all the files stored in this session. Below this window you have another window that changes depending on what file you have selected. When you have the .session file selected you will get the window that allows you to enter the metadata for that session, along with Status \& Stages, and Notes:\par{}\vspace{6pt}\vspace{10pt plus 2pt minus 1pt}\setbox0=\vbox{\protect\raggedright\leavevmode
\vspace*{0pt}{\XeTeXpicfile "../images/en/Sessions_MetaData_en.png" scaled 750}\\[0pt]\protect\hypertarget{f-NeedsALabel-.xlingpaper.1..styledPaper.1..lingPaper.1..part.2..chapter.3..section1.3..figure.1.}{}\XLingPaperaddtocontents{f-NeedsALabel-.xlingpaper.1..styledPaper.1..lingPaper.1..part.2..chapter.3..section1.3..figure.1.}\textit{{Figure }}\textit{{6.5}}\textit{{ \\}}}\box0\par{}\vspace{10pt plus 2pt minus 1pt}\vspace{0pt}\indent SayMore will go ahead an copy in location project metadata to corresponding fields below More Fields. If you change the location metadata in the Project tab it is not updated in existing sessions. Also you can click on the Notes Tab and then enter notes there about the session in prose. The Notes has no formating features.\par{}\vspace{6pt}\vspace{0pt}\indent When you click on the Status \& Stages Tab you will see:\par{}\vspace{6pt}\vspace{10pt plus 2pt minus 1pt}\setbox0=\vbox{\protect\raggedright\leavevmode
\vspace*{0pt}{\XeTeXpicfile "../images/en/Sessions_Status&Stages_en.png" scaled 750}\\[0pt]\protect\hypertarget{f-NeedsALabel-.xlingpaper.1..styledPaper.1..lingPaper.1..part.2..chapter.3..section1.3..figure.3.}{}\XLingPaperaddtocontents{f-NeedsALabel-.xlingpaper.1..styledPaper.1..lingPaper.1..part.2..chapter.3..section1.3..figure.3.}\textit{{Figure }}\textit{{6.6}}\textit{{ Session Status \& Stages Tab\\}}}\box0\par{}\vspace{10pt plus 2pt minus 1pt}\vspace{0pt}\indent The status is where you can make a big picture statement as to where the recording has been dealt with. The Stages is where you set what has been done with the session recording.\par{}\clearpage
\thispagestyle{bodyfirstpage}\markboth{Managing People}{Managing People}
\XLingPaperaddtocontents{cPeople}{\XLingPaperneedspace{3\baselineskip}\noindent
\fontsize{18}{21.599999999999998}\selectfont \textbf{{\centering
\raisebox{\baselineskip}[0pt]{\protect\hypertarget{cPeople}{}}\raisebox{\baselineskip}[0pt]{\pdfbookmark[1]{7 Managing People}{cPeople}}7\\}}}\par{}
\vspace{10.8pt}{\XLingPaperneedspace{3\baselineskip}\noindent
\fontsize{18}{21.599999999999998}\selectfont \textbf{{\centering
Managing People\\}}}\par{}
\vspace{21.6pt}{\XLingPaperneedspace{3\baselineskip}
\noindent\rule{\textwidth}{1pt}
{}\penalty10000\vspace{3pt}\XLingPaperneedspace{3\baselineskip}\noindent
\fontsize{12}{14.399999999999999}\selectfont \textbf{{\noindent
\raisebox{\baselineskip}[0pt]{\pdfbookmark[2]{{7.1 } Add a New Person}{sNewPerson}}\raisebox{\baselineskip}[0pt]{\protect\hypertarget{sNewPerson}{}}{7.1 }Add a New Person}}\markright{Add a New Person}
\XLingPaperaddtocontents{sNewPerson}}\par{}
\penalty10000\vspace{10pt}\penalty10000\vspace{0pt}\indent When you clique on {\textbf{New}} in Menu {\textbf{Person}}, you will see:\par{}\vspace{6pt}\vspace{10pt plus 2pt minus 1pt}\setbox0=\vbox{\protect\raggedright\leavevmode
\vspace*{0pt}{\XeTeXpicfile "../images/en/EmptyNewPerson_en.png" scaled 750}\\[0pt]\protect\hypertarget{f-NeedsALabel-.xlingpaper.1..styledPaper.1..lingPaper.1..part.1..chapter.2..section1.3..section2.1..figure.1.}{}\XLingPaperaddtocontents{f-NeedsALabel-.xlingpaper.1..styledPaper.1..lingPaper.1..part.1..chapter.2..section1.3..section2.1..figure.1.}\textit{{Figure }}\textit{{7.1}}\textit{{ Empty New Person\\}}}\box0\par{}\vspace{10pt plus 2pt minus 1pt}\vspace{0pt}\indent It is important to try to fill out as much as possible about the Person.\par{}{\XLingPaperneedspace{3\baselineskip}
\noindent\rule{\textwidth}{1pt}
{}\penalty10000\vspace{3pt}\XLingPaperneedspace{3\baselineskip}\noindent
\fontsize{12}{14.399999999999999}\selectfont \textbf{{\noindent
\raisebox{\baselineskip}[0pt]{\pdfbookmark[2]{{7.2 } Informed Consent }{sInformedConsent}}\raisebox{\baselineskip}[0pt]{\protect\hypertarget{sInformedConsent}{}}{7.2 }Informed Consent }}\markright{Informed Consent }
\XLingPaperaddtocontents{sInformedConsent}}\par{}
\penalty10000\vspace{10pt}\penalty10000\vspace{0pt}\indent It is very important to remember to make note of everyone who has participated in the project, not just those making recordings. If you add someone to the project and don't have their consent, it is possible that it may prevent the work to proceed or even be distributed when finished due to questions about the legality of the body of work gathered. So before anyone contributes to the project, get their informed consent finished.\par{}\vspace{6pt}
\begin{mdframed}
[backgroundcolor=FTColorA,skipabove=3pt,skipbelow=3pt,innermargin=2cm,outermargin=2cm,innertopmargin=.03in,innerbottommargin=.03in,innerleftmargin=.125in,innerrightmargin=.125in,align=left]\vspace{0pt}\indent \vspace*{0pt}{\XeTeXpicfile "../images/en/WARNING_ICON.png" scaled 750} It says {\textit{\textbf{Informed Consent}}} here because it is the responsibility of the researcher to inform the participants of their intellectual property rights. If the person does not understand that they have the option to refuse any use of their intellectual property, you may not, even in good faith, use the products for academic or personal purposes.\par{}\end{mdframed}
{\XLingPaperneedspace{3\baselineskip}
\noindent\rule{\textwidth}{1pt}
{}\penalty10000\vspace{3pt}\XLingPaperneedspace{3\baselineskip}\noindent
\fontsize{12}{14.399999999999999}\selectfont \textbf{{\noindent
\raisebox{\baselineskip}[0pt]{\pdfbookmark[2]{{7.3 } Delete Person...}{sDeletePerson}}\raisebox{\baselineskip}[0pt]{\protect\hypertarget{sDeletePerson}{}}{7.3 }Delete Person...}}\markright{Delete Person...}
\XLingPaperaddtocontents{sDeletePerson}}\par{}
\penalty10000\vspace{10pt}\penalty10000\vspace{0pt}\indent Allows you to remove a person from a session. You can also just right clique on the person and select {\textbf{Delete Person}}. A warning box will appear before deletion is final.\par{}\clearpage
{\clearpage
\XLingPaperneedspace{3\baselineskip}\noindent
\fontsize{18}{21.599999999999998}\selectfont \textbf{{\centering
\thispagestyle{empty}\raisebox{\baselineskip}[0pt]{\pdfbookmark[1]{Part III Transcribing Audio and Video files}{pTranscribe}}\raisebox{\baselineskip}[0pt]{\protect\hypertarget{pTranscribe}{}}Part III\\}}}\par{}
\vspace{10.8pt}{\XLingPaperneedspace{3\baselineskip}\noindent
\fontsize{18}{21.599999999999998}\selectfont \textbf{{\centering
Transcribing Audio and Video files\\}}}\par{}
\vspace{21.6pt}\vspace{0pt}\vspace{6pt}\clearpage
\thispagestyle{bodyfirstpage}\markboth{Overview of Transcription Process}{Overview of Transcription Process}
\XLingPaperaddtocontents{cTranscribe}{\XLingPaperneedspace{3\baselineskip}\noindent
\fontsize{18}{21.599999999999998}\selectfont \textbf{{\centering
\raisebox{\baselineskip}[0pt]{\protect\hypertarget{cTranscribe}{}}\raisebox{\baselineskip}[0pt]{\pdfbookmark[1]{8 Overview of Transcription Process}{cTranscribe}}8\\}}}\par{}
\vspace{10.8pt}{\XLingPaperneedspace{3\baselineskip}\noindent
\fontsize{18}{21.599999999999998}\selectfont \textbf{{\centering
Overview of Transcription Process\\}}}\par{}
\vspace{21.6pt}\vspace{0pt}\vspace{6pt}\vspace{0pt}{\parskip .5pt plus 1pt minus 1pt

\vspace{\baselineskip}

{\setlength{\XLingPapertempdim}{\XLingPaperbulletlistitemwidth+6em}\leftskip\XLingPapertempdim\relax
\interlinepenalty10000
\XLingPaperlistitem{6em}{\XLingPaperbulletlistitemwidth}{•}{}\vspace{3pt}}
{\setlength{\XLingPapertempdim}{\XLingPaperbulletlistitemwidth+6em}\leftskip\XLingPapertempdim\relax
\interlinepenalty10000
\XLingPaperlistitem{6em}{\XLingPaperbulletlistitemwidth}{•}{}\vspace{3pt}}
{\setlength{\XLingPapertempdim}{\XLingPaperbulletlistitemwidth+6em}\leftskip\XLingPapertempdim\relax
\interlinepenalty10000
\XLingPaperlistitem{6em}{\XLingPaperbulletlistitemwidth}{•}{}}
\vspace{\baselineskip}
}\vspace{0pt}\vspace{6pt}\vspace{0pt}\vspace{6pt}\vspace{0pt}{\parskip .5pt plus 1pt minus 1pt
                    
\vspace{\baselineskip}

{\setlength{\XLingPapertempdim}{\XLingPapersingledigitlistitemwidth+6em}\leftskip\XLingPapertempdim\relax
\interlinepenalty10000
\XLingPaperlistitem{6em}{\XLingPapersingledigitlistitemwidth}{1.}{Source Recording}\vspace{3pt}}
{\setlength{\XLingPapertempdim}{\XLingPapersingledigitlistitemwidth+6em}\leftskip\XLingPapertempdim\relax
\interlinepenalty10000
\XLingPaperlistitem{6em}{\XLingPapersingledigitlistitemwidth}{2.}{Informed Consent}\vspace{3pt}}
{\setlength{\XLingPapertempdim}{\XLingPapersingledigitlistitemwidth+6em}\leftskip\XLingPapertempdim\relax
\interlinepenalty10000
\XLingPaperlistitem{6em}{\XLingPapersingledigitlistitemwidth}{3.}{Careful Speech}\vspace{3pt}}
{\setlength{\XLingPapertempdim}{\XLingPapersingledigitlistitemwidth+6em}\leftskip\XLingPapertempdim\relax
\interlinepenalty10000
\XLingPaperlistitem{6em}{\XLingPapersingledigitlistitemwidth}{4.}{Oral Transcription}\vspace{3pt}}
{\setlength{\XLingPapertempdim}{\XLingPapersingledigitlistitemwidth+6em}\leftskip\XLingPapertempdim\relax
\interlinepenalty10000
\XLingPaperlistitem{6em}{\XLingPapersingledigitlistitemwidth}{5.}{Written Transcription}\vspace{3pt}}
{\setlength{\XLingPapertempdim}{\XLingPapersingledigitlistitemwidth+6em}\leftskip\XLingPapertempdim\relax
\interlinepenalty10000
\XLingPaperlistitem{6em}{\XLingPapersingledigitlistitemwidth}{6.}{Written Translation}}
\vspace{\baselineskip}
}\vspace{0pt}\vspace{6pt}\vspace{0pt}\vspace{6pt}\vspace{0pt}\vspace{6pt}{\XLingPaperneedspace{3\baselineskip}
\noindent\rule{\textwidth}{1pt}
{}\penalty10000\vspace{3pt}\XLingPaperneedspace{3\baselineskip}\noindent
\fontsize{12}{14.399999999999999}\selectfont \textbf{{\noindent
\raisebox{\baselineskip}[0pt]{\pdfbookmark[2]{{8.1 } Session Metadata}{sMetaMaybe}}\raisebox{\baselineskip}[0pt]{\protect\hypertarget{sMetaMaybe}{}}{8.1 }Session Metadata}}\markright{Session Metadata}
\XLingPaperaddtocontents{sMetaMaybe}}\par{}
\penalty10000\vspace{10pt}\penalty10000\vspace{0pt}\indent See Section \hyperlink{sSessMeta1}{6.3}\par{}{\XLingPaperneedspace{3\baselineskip}
\noindent\rule{\textwidth}{1pt}
{}\penalty10000\vspace{3pt}\XLingPaperneedspace{3\baselineskip}\noindent
\fontsize{12}{14.399999999999999}\selectfont \textbf{{\noindent
\raisebox{\baselineskip}[0pt]{\pdfbookmark[2]{{8.2 } Add Session Contributors}{sAddContributors}}\raisebox{\baselineskip}[0pt]{\protect\hypertarget{sAddContributors}{}}{8.2 }Add Session Contributors}}\markright{Add Session Contributors}
\XLingPaperaddtocontents{sAddContributors}}\par{}
\penalty10000\vspace{10pt}\penalty10000{\XLingPaperneedspace{3\baselineskip}
\noindent\rule{\textwidth}{1pt}
{}\penalty10000\vspace{3pt}\XLingPaperneedspace{3\baselineskip}\noindent
\fontsize{12}{14.399999999999999}\selectfont \textbf{{\noindent
\raisebox{\baselineskip}[0pt]{\pdfbookmark[2]{{8.3 } Start Annotating}{sStartAnnot}}\raisebox{\baselineskip}[0pt]{\protect\hypertarget{sStartAnnot}{}}{8.3 }Start Annotating}}\markright{Start Annotating}
\XLingPaperaddtocontents{sStartAnnot}}\par{}
\penalty10000\vspace{10pt}\penalty10000\vspace{0pt}\vspace{6pt}{\XLingPaperneedspace{3\baselineskip}
\noindent\rule{\textwidth}{.4pt}
{}\penalty10000\vspace{3pt}\XLingPaperneedspace{3\baselineskip}\noindent
\fontsize{10}{12}\selectfont \textbf{{\noindent
\raisebox{\baselineskip}[0pt]{\pdfbookmark[3]{{8.3.1 } Convert Video to Audio}{sVidtoAudio}}\raisebox{\baselineskip}[0pt]{\protect\hypertarget{sVidtoAudio}{}}{8.3.1 }Convert Video to Audio}}\markright{Convert Video to Audio}
\XLingPaperaddtocontents{sVidtoAudio}}\par{}
\penalty10000\vspace{10pt}\penalty10000\vspace{0pt}{\parskip .5pt plus 1pt minus 1pt
                    
\vspace{\baselineskip}

{\setlength{\XLingPapertempdim}{\XLingPapersingledigitlistitemwidth+6em}\leftskip\XLingPapertempdim\relax
\interlinepenalty10000
\XLingPaperlistitem{6em}{\XLingPapersingledigitlistitemwidth}{1.}{}\vspace{3pt}}
{\setlength{\XLingPapertempdim}{\XLingPapersingledigitlistitemwidth+6em}\leftskip\XLingPapertempdim\relax
\interlinepenalty10000
\XLingPaperlistitem{6em}{\XLingPapersingledigitlistitemwidth}{2.}{}}
\vspace{\baselineskip}
}{\XLingPaperneedspace{3\baselineskip}
\noindent\rule{\textwidth}{.4pt}
{}\penalty10000\vspace{3pt}\XLingPaperneedspace{3\baselineskip}\noindent
\fontsize{10}{12}\selectfont \textbf{{\noindent
\raisebox{\baselineskip}[0pt]{\pdfbookmark[3]{{8.3.2 } Automatic Segmentation}{sStartAnnot}}\raisebox{\baselineskip}[0pt]{\protect\hypertarget{sStartAnnot}{}}{8.3.2 }Automatic Segmentation}}\markright{Automatic Segmentation}
\XLingPaperaddtocontents{sStartAnnot}}\par{}
\penalty10000\vspace{10pt}\penalty10000{\XLingPaperneedspace{3\baselineskip}
\noindent\rule{\textwidth}{1pt}
{}\penalty10000\vspace{3pt}\XLingPaperneedspace{3\baselineskip}\noindent
\fontsize{12}{14.399999999999999}\selectfont \textbf{{\noindent
\raisebox{\baselineskip}[0pt]{\pdfbookmark[2]{{8.4 } Adjust Segmentation}{sChangeSegmentation}}\raisebox{\baselineskip}[0pt]{\protect\hypertarget{sChangeSegmentation}{}}{8.4 }Adjust Segmentation}}\markright{Adjust Segmentation}
\XLingPaperaddtocontents{sChangeSegmentation}}\par{}
\penalty10000\vspace{10pt}\penalty10000\vspace{0pt}{\XLingPaperneedspace{3\baselineskip}
\noindent\rule{\textwidth}{1pt}
{}\penalty10000\vspace{3pt}\XLingPaperneedspace{3\baselineskip}\noindent
\fontsize{12}{14.399999999999999}\selectfont \textbf{{\noindent
\raisebox{\baselineskip}[0pt]{\pdfbookmark[2]{{8.5 } Manual Segmentation}{sSegMan}}\raisebox{\baselineskip}[0pt]{\protect\hypertarget{sSegMan}{}}{8.5 }Manual Segmentation}}\markright{Manual Segmentation}
\XLingPaperaddtocontents{sSegMan}}\par{}
\penalty10000\vspace{10pt}\penalty10000\vspace{0pt}{\XLingPaperneedspace{3\baselineskip}
\noindent\rule{\textwidth}{1pt}
{}\penalty10000\vspace{3pt}\XLingPaperneedspace{3\baselineskip}\noindent
\fontsize{12}{14.399999999999999}\selectfont \textbf{{\noindent
\raisebox{\baselineskip}[0pt]{\pdfbookmark[2]{{8.6 } Careful Speech Transcription}{sCarefulSpeech}}\raisebox{\baselineskip}[0pt]{\protect\hypertarget{sCarefulSpeech}{}}{8.6 }Careful Speech Transcription}}\markright{Careful Speech Transcription}
\XLingPaperaddtocontents{sCarefulSpeech}}\par{}
\penalty10000\vspace{10pt}\penalty10000\vspace{0pt}{\XLingPaperneedspace{3\baselineskip}
\noindent\rule{\textwidth}{1pt}
{}\penalty10000\vspace{3pt}\XLingPaperneedspace{3\baselineskip}\noindent
\fontsize{12}{14.399999999999999}\selectfont \textbf{{\noindent
\raisebox{\baselineskip}[0pt]{\pdfbookmark[2]{{8.7 } Oral Translation}{sOralTransl}}\raisebox{\baselineskip}[0pt]{\protect\hypertarget{sOralTransl}{}}{8.7 }Oral Translation}}\markright{Oral Translation}
\XLingPaperaddtocontents{sOralTransl}}\par{}
\penalty10000\vspace{10pt}\penalty10000\vspace{0pt}{\XLingPaperneedspace{3\baselineskip}
\noindent\rule{\textwidth}{1pt}
{}\penalty10000\vspace{3pt}\XLingPaperneedspace{3\baselineskip}\noindent
\fontsize{12}{14.399999999999999}\selectfont \textbf{{\noindent
\raisebox{\baselineskip}[0pt]{\pdfbookmark[2]{{8.8 } Written Transcription}{sWrittenTrans}}\raisebox{\baselineskip}[0pt]{\protect\hypertarget{sWrittenTrans}{}}{8.8 }Written Transcription}}\markright{Written Transcription}
\XLingPaperaddtocontents{sWrittenTrans}}\par{}
\penalty10000\vspace{10pt}\penalty10000\vspace{0pt}\vspace{6pt}
\begin{mdframed}
[backgroundcolor=FTColorA,skipabove=3pt,skipbelow=3pt,innermargin=2cm,outermargin=2cm,innertopmargin=.03in,innerbottommargin=.03in,innerleftmargin=.125in,innerrightmargin=.125in,align=left]\vspace{0pt}\end{mdframed}
{\XLingPaperneedspace{3\baselineskip}
\noindent\rule{\textwidth}{1pt}
{}\penalty10000\vspace{3pt}\XLingPaperneedspace{3\baselineskip}\noindent
\fontsize{12}{14.399999999999999}\selectfont \textbf{{\noindent
\raisebox{\baselineskip}[0pt]{\pdfbookmark[2]{{8.9 } Written Translation}{sWritTransl}}\raisebox{\baselineskip}[0pt]{\protect\hypertarget{sWritTransl}{}}{8.9 }Written Translation}}\markright{Written Translation}
\XLingPaperaddtocontents{sWritTransl}}\par{}
\penalty10000\vspace{10pt}\penalty10000\vspace{0pt}\vspace{6pt}
\begin{mdframed}
[backgroundcolor=FTColorA,skipabove=3pt,skipbelow=3pt,innermargin=2cm,outermargin=2cm,innertopmargin=.03in,innerbottommargin=.03in,innerleftmargin=.125in,innerrightmargin=.125in,align=left]\vspace{0pt}\end{mdframed}
\clearpage
{\clearpage
\XLingPaperneedspace{3\baselineskip}\noindent
\fontsize{18}{21.599999999999998}\selectfont \textbf{{\centering
\thispagestyle{empty}\raisebox{\baselineskip}[0pt]{\pdfbookmark[1]{Part IV Archiving}{pArchiving}}\raisebox{\baselineskip}[0pt]{\protect\hypertarget{pArchiving}{}}Part IV\\}}}\par{}
\vspace{10.8pt}{\XLingPaperneedspace{3\baselineskip}\noindent
\fontsize{18}{21.599999999999998}\selectfont \textbf{{\centering
Archiving\\}}}\par{}
\vspace{21.6pt}\clearpage
\thispagestyle{bodyfirstpage}\markboth{Exporting Data}{Exporting Data}
\XLingPaperaddtocontents{cExport}{\XLingPaperneedspace{3\baselineskip}\noindent
\fontsize{18}{21.599999999999998}\selectfont \textbf{{\centering
\raisebox{\baselineskip}[0pt]{\protect\hypertarget{cExport}{}}\raisebox{\baselineskip}[0pt]{\pdfbookmark[1]{9 Exporting Data}{cExport}}9\\}}}\par{}
\vspace{10.8pt}{\XLingPaperneedspace{3\baselineskip}\noindent
\fontsize{18}{21.599999999999998}\selectfont \textbf{{\centering
Exporting Data\\}}}\par{}
\vspace{21.6pt}\vspace{0pt}\vspace{6pt}{\XLingPaperneedspace{3\baselineskip}
\noindent\rule{\textwidth}{1pt}
{}\penalty10000\vspace{3pt}\XLingPaperneedspace{3\baselineskip}\noindent
\fontsize{12}{14.399999999999999}\selectfont \textbf{{\noindent
\raisebox{\baselineskip}[0pt]{\pdfbookmark[2]{{9.1 } Export Subtitles}{sSubtitles}}\raisebox{\baselineskip}[0pt]{\protect\hypertarget{sSubtitles}{}}{9.1 }Export Subtitles}}\markright{Export Subtitles}
\XLingPaperaddtocontents{sSubtitles}}\par{}
\penalty10000\vspace{10pt}\penalty10000\vspace{0pt}{\XLingPaperneedspace{3\baselineskip}
\noindent\rule{\textwidth}{1pt}
{}\penalty10000\vspace{3pt}\XLingPaperneedspace{3\baselineskip}\noindent
\fontsize{12}{14.399999999999999}\selectfont \textbf{{\noindent
\raisebox{\baselineskip}[0pt]{\pdfbookmark[2]{{9.2 } Export Audacity Labels}{sSubtitles}}\raisebox{\baselineskip}[0pt]{\protect\hypertarget{sSubtitles}{}}{9.2 }Export Audacity Labels}}\markright{Export Audacity Labels}
\XLingPaperaddtocontents{sSubtitles}}\par{}
\penalty10000\vspace{10pt}\penalty10000\vspace{0pt}{\XLingPaperneedspace{3\baselineskip}
\noindent\rule{\textwidth}{1pt}
{}\penalty10000\vspace{3pt}\XLingPaperneedspace{3\baselineskip}\noindent
\fontsize{12}{14.399999999999999}\selectfont \textbf{{\noindent
\raisebox{\baselineskip}[0pt]{\pdfbookmark[2]{{9.3 } Export Sessions}{sExportSession}}\raisebox{\baselineskip}[0pt]{\protect\hypertarget{sExportSession}{}}{9.3 }Export Sessions}}\markright{Export Sessions}
\XLingPaperaddtocontents{sExportSession}}\par{}
\penalty10000\vspace{10pt}\penalty10000{\parskip .5pt plus 1pt minus 1pt
                    
{\setlength{\XLingPapertempdim}{\XLingPapersingledigitlistitemwidth+6em}\leftskip\XLingPapertempdim\relax
\interlinepenalty10000
\XLingPaperlistitem{6em}{\XLingPapersingledigitlistitemwidth}{1.}{Open the project that has the session data you want to export.}\vspace{3pt}}
{\setlength{\XLingPapertempdim}{\XLingPapersingledigitlistitemwidth+6em}\leftskip\XLingPapertempdim\relax
\interlinepenalty10000
\XLingPaperlistitem{6em}{\XLingPapersingledigitlistitemwidth}{2.}{On the Project menu, clique Export Sessions.\\{\textit{The Export Data dialog box appears:}}\\\vspace{10pt plus 2pt minus 1pt}\setbox0=\vbox{\protect\raggedright\leavevmode
\vspace*{0pt}{\XeTeXpicfile "../images/en/ExportSessions_en.png" scaled 750}\\[0pt]\protect\hypertarget{f-NeedsALabel-.xlingpaper.1..styledPaper.1..lingPaper.1..part.1..chapter.2..section1.1..section2.2..ol.1..li.2..figure.1.}{}\XLingPaperaddtocontents{f-NeedsALabel-.xlingpaper.1..styledPaper.1..lingPaper.1..part.1..chapter.2..section1.1..section2.2..ol.1..li.2..figure.1.}\textit{{Figure }}\textit{{9.1}}\textit{{ Export Sessions Save Dialogue\\}}}\box0\par{}\vspace{10pt plus 2pt minus 1pt}}\vspace{3pt}}
{\setlength{\XLingPapertempdim}{\XLingPapersingledigitlistitemwidth+6em}\leftskip\XLingPapertempdim\relax
\interlinepenalty10000
\XLingPaperlistitem{6em}{\XLingPapersingledigitlistitemwidth}{3.}{Click {\textbf{Save}}. \\{\textit{The session data in the open project is exported to a file. The file is then shown in the folder.}}}\vspace{3pt}}
{\setlength{\XLingPapertempdim}{\XLingPapersingledigitlistitemwidth+6em}\leftskip\XLingPapertempdim\relax
\interlinepenalty10000
\XLingPaperlistitem{6em}{\XLingPapersingledigitlistitemwidth}{4.}{To open the file, open the folder, and then double-click the file. \\{\textit{The export file opens with the program specified as the default program for the file type. }}}}
\vspace{\baselineskip}
}{\XLingPaperneedspace{3\baselineskip}
\noindent\rule{\textwidth}{1pt}
{}\penalty10000\vspace{3pt}\XLingPaperneedspace{3\baselineskip}\noindent
\fontsize{12}{14.399999999999999}\selectfont \textbf{{\noindent
\raisebox{\baselineskip}[0pt]{\pdfbookmark[2]{{9.4 } Export People}{sExportPeople}}\raisebox{\baselineskip}[0pt]{\protect\hypertarget{sExportPeople}{}}{9.4 }Export People}}\markright{Export People}
\XLingPaperaddtocontents{sExportPeople}}\par{}
\penalty10000\vspace{10pt}\penalty10000\vspace{0pt}\indent To export all of the people and associated metadata in the open project to a file, do the following steps:\par{}{\parskip .5pt plus 1pt minus 1pt
                    
\vspace{\baselineskip}

{\setlength{\XLingPapertempdim}{\XLingPapersingledigitlistitemwidth+6em}\leftskip\XLingPapertempdim\relax
\interlinepenalty10000
\XLingPaperlistitem{6em}{\XLingPapersingledigitlistitemwidth}{1.}{Open the project that has the people you want to export.}\vspace{3pt}}
{\setlength{\XLingPapertempdim}{\XLingPapersingledigitlistitemwidth+6em}\leftskip\XLingPapertempdim\relax
\interlinepenalty10000
\XLingPaperlistitem{6em}{\XLingPapersingledigitlistitemwidth}{2.}{On the Project menu, clique Export People.\\The Export Data dialog box appears:\vspace{10pt plus 2pt minus 1pt}\setbox0=\vbox{\protect\raggedright\leavevmode
\vspace*{0pt}{\XeTeXpicfile "../images/en/ExportPeople_en.png" scaled 750}\\[0pt]\protect\hypertarget{f-NeedsALabel-.xlingpaper.1..styledPaper.1..lingPaper.1..part.1..chapter.2..section1.1..section2.3..ol.1..li.2..figure.1.}{}\XLingPaperaddtocontents{f-NeedsALabel-.xlingpaper.1..styledPaper.1..lingPaper.1..part.1..chapter.2..section1.1..section2.3..ol.1..li.2..figure.1.}\textit{{Figure }}\textit{{9.2}}\textit{{ Export People Save Dialogue\\}}}\box0\par{}\vspace{10pt plus 2pt minus 1pt}}\vspace{3pt}}
{\setlength{\XLingPapertempdim}{\XLingPapersingledigitlistitemwidth+6em}\leftskip\XLingPapertempdim\relax
\interlinepenalty10000
\XLingPaperlistitem{6em}{\XLingPapersingledigitlistitemwidth}{3.}{In the dialog box, clique the File name box, and then type a name. \\Leave the Save as type selection as CSV (Comma delimited)(*.csv). \\Also, it is recommended that you save the file in the project folder, which is the default location.}\vspace{3pt}}
{\setlength{\XLingPapertempdim}{\XLingPapersingledigitlistitemwidth+6em}\leftskip\XLingPapertempdim\relax
\interlinepenalty10000
\XLingPaperlistitem{6em}{\XLingPapersingledigitlistitemwidth}{4.}{Click Save. \\{\textit{The people data in the open project is exported to a file. The file is then shown in the folder.}}}\vspace{3pt}}
{\setlength{\XLingPapertempdim}{\XLingPapersingledigitlistitemwidth+6em}\leftskip\XLingPapertempdim\relax
\interlinepenalty10000
\XLingPaperlistitem{6em}{\XLingPapersingledigitlistitemwidth}{5.}{To open the file, open the folder, and then double-click the file. \\{\textit{The export file opens with the program specified as the default program for the file type.}}}}
\vspace{\baselineskip}
}\clearpage
\thispagestyle{bodyfirstpage}\markboth{Archiving Overview}{Archiving Overview}
\XLingPaperaddtocontents{cArchive}{\XLingPaperneedspace{3\baselineskip}\noindent
\fontsize{18}{21.599999999999998}\selectfont \textbf{{\centering
\raisebox{\baselineskip}[0pt]{\protect\hypertarget{cArchive}{}}\raisebox{\baselineskip}[0pt]{\pdfbookmark[1]{10 Archiving Overview}{cArchive}}10\\}}}\par{}
\vspace{10.8pt}{\XLingPaperneedspace{3\baselineskip}\noindent
\fontsize{18}{21.599999999999998}\selectfont \textbf{{\centering
Archiving Overview\\}}}\par{}
\vspace{21.6pt}{\XLingPaperneedspace{3\baselineskip}
\noindent\rule{\textwidth}{1pt}
{}\penalty10000\vspace{3pt}\XLingPaperneedspace{3\baselineskip}\noindent
\fontsize{12}{14.399999999999999}\selectfont \textbf{{\noindent
\raisebox{\baselineskip}[0pt]{\pdfbookmark[2]{{10.1 } Archive with RAMP (SIL)...}{sArchiveRamp}}\raisebox{\baselineskip}[0pt]{\protect\hypertarget{sArchiveRamp}{}}{10.1 }Archive with RAMP (SIL)...}}\markright{Archive with RAMP (SIL)...}
\XLingPaperaddtocontents{sArchiveRamp}}\par{}
\penalty10000\vspace{10pt}\penalty10000\vspace{0pt}\indent {\textbf{Archive with RAMP (SIL)...}} This allows you to archive using SIL's access protocol Repository for Electronic Archiving and Publishing format. It is designed to carry over all meta-data to an archive.\par{}\pagestyle{body}\XLingPaperendtableofcontents
\pagebreak\end{MainFont}
\end{document}
